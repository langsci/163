%% -*- coding:utf-8 -*-
\title{A lexicalist account of argument structure}
\renewcommand{\lsCoverTitleFont}[1]{\sffamily\addfontfeatures{Scale=MatchUppercase}\fontsize{42pt}{16.75mm}\selectfont
  #1}

\subtitle{Template-based phrasal LFG approaches and a lexical HPSG alternative}
\author{Stefan Müller}
\typesetter{Stefan Müller}
\proofreader{%
}

\openreviewer{%
}

\BackTitle{A lexicalist account of argument structure: Template-based phrasal LFG approaches and a
  lexical HPSG alternative}

\BackBody{There are two prominent schools in linguistics: Minimalism (Chomsky) and Construction
  Grammar (Goldberg, Tomasello). Minimalism comes with the claim that our linguistic capabilities
  consist of an abstract, binary combinatorial operation (Merge) and a lexicon. Most versions of
  Construction Grammar assume that language consists of flat phrasal schemata that contribute their
  own meaning and may license additional arguments. This book examines a variant of Lexical
  Functional Grammar, which is lexical in principle but was augmented by tools that allow for the
  description of phrasal constructions in the Construction Grammar sense. These new tools include
  templates that can be used to model inheritance hierarchies and a resource driven semantics. The
  resource driven semantics makes it possible to reach the effects that lexical rules had, for
  example remapping of arguments, by semantic means. The semantic constraints can be evaluated in
  the syntactic component, which is basically similar to the delayed execution of lexical rules. So
  this is a new formalization that might be suitable to provide solutions to longstanding problems
  that are not available for other formalizations. 

While the authors suggest a lexical treatment of many phenomena and only assume phrasal
constructions for selected phenomena like benefactive and resultative constructions in English, it
can be shown that even these two constructions should not be treated phrasally in English and that
the analysis would not extend to other languages as for instance German. I show that the new formal
tools do not really improve the situation and many of the basic conceptual problems remain. Since
this specific proposal fails for two constructions, it follows that proposals (in the same
framework) that assume phrasal analyses for all constructions are not appropriate either. 

The conclusion is that lexical models are needed and this entails that the schemata that combine
syntactic objects are rather abstract (as in Categorial Grammar, Minimalism, HPSG and standard
LFG). On the other hand there are constructions that should be treated by very specific, phrasal
schemata as in Construction Grammar and LFG and HPSG. So the conclusion is that both schools are
right (and wrong) and that a combination of ideas from both camps is needed.} 
\dedication{For Friederike}
%% \renewcommand{\lsISBNdigital}{978-3-944675-21-3}
%% \renewcommand{\lsISBNhardcoverOne}{978-3-946234-29-6}
%% \renewcommand{\lsISBNhardcoverTwo}{978-3-946234-40-1}
%% \renewcommand{\lsISBNsoftcoverOne}{978-3-946234-30-2}
%% \renewcommand{\lsISBNsoftcoverTwo}{978-3-946234-41-8}
%% \renewcommand{\lsISBNsoftcoverusOne}{978-1-530465-62-0}
%% \renewcommand{\lsISBNsoftcoverusTwo}{978-1-523743-82-7}
%% \renewcommand{\lsSeries}{tbls} % use lowercase acronym, e.g. sidl, eotms, tgdi
\renewcommand{\lsSeries}{cfls} % use lowercase acronym, e.g. sidl, eotms, tgdi
\renewcommand{\lsSeriesNumber}{3} %will be assigned when the book enters the proofreading stage
%% \renewcommand{\lsURL}{http://langsci-press.org/catalog/book/25} % contact the coordinator for the right number


%      <!-- Local IspellDict: en_US-w_accents -->
