%% -*- coding:utf-8 -*-
%% 130

%% -*- coding:utf-8 -*-
\title{A lexicalist account of argument structure}
\renewcommand{\lsCoverTitleFont}[1]{\sffamily\addfontfeatures{Scale=MatchUppercase}\fontsize{42pt}{16.75mm}\selectfont
  #1}

\subtitle{Template-based phrasal LFG approaches and a lexical HPSG alternative}
\author{Stefan Müller}
\typesetter{Stefan Müller}
\proofreader{%
Gerald Delahunty,
Andreas Hölzl, 
Timm Lichte, 
Annie Zaenen}

\openreviewer{%
}

\BackTitle{A lexicalist account of argument structure: Template-based phrasal LFG approaches and a
  lexical HPSG alternative}

\BackBody{There are two prominent schools in linguistics: Minimalism (Chomsky) and Construction
  Grammar (Goldberg, Tomasello). Minimalism comes with the claim that our linguistic capabilities
  consist of an abstract, binary combinatorial operation (Merge) and a lexicon. Most versions of
  Construction Grammar assume that language consists of flat phrasal schemata that contribute their
  own meaning and may license additional arguments. This book examines a variant of Lexical
  Functional Grammar, which is lexical in principle but was augmented by tools that allow for the
  description of phrasal constructions in the Construction Grammar sense. These new tools include
  templates that can be used to model inheritance hierarchies and a resource driven semantics. The
  resource driven semantics makes it possible to reach the effects that lexical rules had, for
  example remapping of arguments, by semantic means. The semantic constraints can be evaluated in
  the syntactic component, which is basically similar to the delayed execution of lexical rules. So
  this is a new formalization that might be suitable to provide solutions to longstanding problems
  that are not available for other formalizations. 

While the authors suggest a lexical treatment of many phenomena and only assume phrasal
constructions for selected phenomena like benefactive and resultative constructions in English, it
can be shown that even these two constructions should not be treated phrasally in English and that
the analysis would not extend to other languages as for instance German. I show that the new formal
tools do not really improve the situation and many of the basic conceptual problems remain. Since
this specific proposal fails for two constructions, it follows that proposals (in the same
framework) that assume phrasal analyses for all constructions are not appropriate either. 

The conclusion is that lexical models are needed and this entails that the schemata that combine
syntactic objects are rather abstract (as in Categorial Grammar, Minimalism, HPSG and standard
LFG). On the other hand there are constructions that should be treated by very specific, phrasal
schemata as in Construction Grammar and LFG and HPSG. So the conclusion is that both schools are
right (and wrong) and that a combination of ideas from both camps is needed.} 
\dedication{For Friederike}
%% \renewcommand{\lsISBNdigital}{978-3-944675-21-3}
%% \renewcommand{\lsISBNhardcoverOne}{978-3-946234-29-6}
%% \renewcommand{\lsISBNhardcoverTwo}{978-3-946234-40-1}
%% \renewcommand{\lsISBNsoftcoverOne}{978-3-946234-30-2}
%% \renewcommand{\lsISBNsoftcoverTwo}{978-3-946234-41-8}
%% \renewcommand{\lsISBNsoftcoverusOne}{978-1-530465-62-0}
%% \renewcommand{\lsISBNsoftcoverusTwo}{978-1-523743-82-7}
%% \renewcommand{\lsSeries}{tbls} % use lowercase acronym, e.g. sidl, eotms, tgdi
\renewcommand{\lsSeries}{cfls} % use lowercase acronym, e.g. sidl, eotms, tgdi
\renewcommand{\lsSeriesNumber}{3} %will be assigned when the book enters the proofreading stage
%% \renewcommand{\lsURL}{http://langsci-press.org/catalog/book/25} % contact the coordinator for the right number


%      <!-- Local IspellDict: en_US-w_accents -->

%% -*- coding:utf-8 -*-
\usepackage{langsci-forest-setup}

\usepackage{german}
\selectlanguage{USenglish}

\usepackage{float,graphicx}

\usepackage{./styles/article-ex,./styles/makros.2e,./styles/eng-date,./styles/my-theorems,./styles/oneline}

\usepackage[figuresright]{rotating}

\usepackage[hyphens]{url}


\usepackage{dalrymple}

\usepackage{jambox}

% Ash:
\usepackage{./styles/trees}
\usepackage{./styles/glue}
\usepackage{./styles/prooftree}
\usepackage{./styles/semantics}
%\usepackage{ndproof}

% relies on pstricks

\makeatletter
\@ifpackagelater{pstricks}{2015/11/15}{%
    % Package is new enough
    \PackageError{pstricks}{We need texlive 2017 with pstricks 2015/11/14}%
}{%
%    \PackageError{<your package or document name>}{Package <name> is to old <...>}%
}
\makeatother

\usepackage{avm}

\usepackage{ash}


% loaded last, because of \dh
\usepackage{mycommands,abbrev}



\usepackage{todonotes}
\newcommand{\todostefan}[1]{\todo[color=green!40]{\footnotesize #1}\xspace}
\newcommand{\iw}[1]{}

\usepackage{./langsci/styles/langsci-gb4e}




\usepackage{./styles/eng-hyp-utf8}


\usepackage{bookmark}

\usepackage{soul}

\usepackage{chngcntr}

\counterwithout*{equation}{chapter}

%% -*- coding:utf-8 -*-
\let\citew=\citealp

\newcommand{\wtrans}[1]{(#1)}
\newcommand{\page}{}

\newcommand{\argzero}{\feat{arg$_0$}\xspace}
\newcommand{\argone}{\feat{arg$_1$}\xspace}
\newcommand{\argtwo}{\feat{arg$_2$}\xspace}
\newcommand{\argthree}{\feat{arg$_3$}\xspace}



\newcommand{\nom}{\textsc{nom}{}\xspace}
\newcommand{\gen}{\textsc{gen}{}\xspace}
\newcommand{\dat}{\textsc{dat}{}\xspace}
\newcommand{\acc}{\textsc{acc}{}\xspace}


%\bibliography{phrasal-lfg}


\begin{document}
%\setcounter{page}{9001}

\frontmatter
\maketitle

\tableofcontents

%\include{chapters/preface}


\chapter{Preface}

This book is part of my efforts to convince Construction Grammarians and people working in related
frameworks that lexical approaches to argument structure are the only ones possible within a certain
set of basic assumptions. I started this discussion with Kerstin Fischer and Anatol Stefanowitch 15
years ago in Bremen and continued it with friends and colleagues in the DFG-financed Construction
Grammar network. Several publications grew out of this work
\citep{Mueller2006d,Mueller2007d,MuellerPersian,MWArgSt,MWArgStReply,MW2014a,MuellerSatztypen,MuellerDefaults}. Usually
the proposals I argued against were not formalized and/or the phenomena I pointed out as problematic
were not covered in theoretical work so far. This is different for the present book: the
constructional proposals I discuss are formulated in Lexical Functional Grammar. Most of the
phenomena are covered and one can clearly see consequences of the proposals I discuss. The book does
not only discuss a constructional LFG analysis of benefactive constructions, it also provides an
alternative lexical HPSG analysis that also shows how interactions of benefactives with resultative
constructions and passive and derivational morphology can be covered in a way that allows for
cross"=linguistic generalizations. The HPSG analysis is implemented in the TRALE system
\citep*{MPR2002a-u,Penn2004a-u} as part of the CoreGram project \citep{MuellerCoreGram} and will be part of the Grammix Virtual Machine
\citep{MuellerGrammix}.

\section*{Acknowledgments}

I thank Ash Asudeh for a really detailed discussion of an earlier version of this book and of
\citew{AGT2014a}. I thank Ida Toivonen for discussion of \citew{Toivonen2013a} and \citew{AGT2014a}
via email. I am very grateful for the \LaTeX{} sources Ash provided for the proofs and figures
that I quoted from their paper. This saved me a lot of time! I want to also thank Elizabeth Christie
for providing the \LaTeX{} code for a lexical item.

I thank Steve Wechsler for discussion of an earlier version of this book. Thanks to Jonas Kuhn for
discussion of the attachment of constraints to c-structures and Economy of Expression.

I also want to thank the participants of HeadLex 2016, the joint conference on LFG and HPSG, for (intense) discussion. Miriam Butt, Mary Dalrymple,
 Ron Kaplan, and Anna Kibort deserve special mention.

Thanks also goes to Martin Haspelmath and Adam Przepiórkowski for comments on an earlier version of this book.

Furthermore, I thank Gert Webelhuth, Gerald Penn, Tom Wasow, Paul Kay, %Antwort auf HPSG liste
Adele Goldberg, Jean-Pierre Koenig, Doug Arnold, Aaron Broadwell, and Berthold Crysmann for various
comments and pointers to relevant literature.

A five page abstract was submitted to HeadLex 2016. I am
grateful to the reviewers of this abstract and the reviewers of a revised 20 page version. I also
thank Miriam Butt and Tracy Holloway King for comments on and discussion of the proceedings version.  The
comments helped a lot to shape and improve this book. Thanks!  

This book underwent community proofreading and I want to thank the proofreaders
(\makeatletter\@proofreader\makeatother) for their careful work: you did an amazing job and Annie
Zanen and Timm Lichte even commented on content. Thanks!


~\medskip

\noindent
Berlin, \today\hfill Stefan Müller


\mainmatter


%% \begin{abstract}
%% This paper is a contribution to the general discussion of the question whether argument structure
%% constructions should be treated with reference to phrasal patterns as suggested by
%% \citet{Goldberg95a,Goldberg2006a}, \citet{CJ2005a} and others or whether lexical approaches like
%% Categorial Grammar \citep{Ajdukiewicz35a-u,Steedman2000a-u}, LFG \citep{Bresnan82a}, HPSG
%% \citep{ps2,Sag97a} and Sign-based Construction Grammar \citep{SBK2012a,Sag2012a} are more
%% appropriate. In the absence of any fully worked out formalized phrasal proposals in Construction
%% Grammar many arguments against phrasal approaches have a hypothetical character (see for instance
%% \citew{Mueller2006d}). This paper discusses recent approaches to two specific argument structure
%% phenomena by \citet{Christie2010a} and \citet*{AGT2014a} in the framework of LFG, which can be seen
%% as formalizations of phrasal constructionist approaches. The authors argue that certain arguments in
%% resultative and benefactive constructions in English are licensed in phrasal constructions rather
%% than lexically. I show that the constructions under consideration are flexible as far as extraction,
%% passivization and coordination is concerned. Applying an old argument by Dowty and Bresnan to
%% resultative constructions, I show that data involving derivational morphology suggests that valence
%% information is visible at the lexical level and hence should not be introduced at the phrasal
%% level. The conclusion is that analyses like the classical lexical analysis of resultative
%% constructions by \citet{Simpson83a} and lexical analyses of the benefactive
%% \citep{Cook2006a-u,Toivonen2013a} are the only option for lexicalist theories like LFG and HPSG.

%% Furthermore I discuss active/""passive alternations in template-based approaches and point out that
%% either the phrase structure component (c"=structure) does not constrain anything or generalizations
%% regarding c"=structure are missing in the same way as they are missing in simple phrase structure
%% grammar. Such missing generalizations motivated Harris and Chomsky to introduce transformations, but
%% transformations are not used in LFG and hence the generalizations regarding active/passive
%% c"=structure pairs cannot be captured. Furthermore, I show that cross"=linguistic generalizations
%% cannot be captured with reference to phrasal configurations since languages differ in the way they
%% actually realize resultative and benefactive constructions. It is shown that languages like German
%% that allow much freer constituent order than English and partial verb phrases are incompatible
%% with phrasal views of argument structure.

%% In a third part I develop a lexical account of German and English resultatives and benefactives in
%% the framework of HPSG and show how this account captures the commonalities between German and
%% English despite the superficial dissimilarities between the two languages. It is also shown how
%% restrictions on the benefactive construction in English can be captured in a lexical model and how
%% these restrictions are propagated in coordinations of lexical material.
%% \end{abstract}


\chapter{Introduction}

This book argues that argument structure should be treated lexically rather than as fixed phrasal
configurations. This is discussed with respect to the benefactive construction and the resultative
construction. It is shown that both constructions are more flexible than claimed in previous
publications and that generalizations about the construction cannot be captured language internally
and cross"=linguistically in phrasal approaches. This first chapter is intended to introduce the
reader to the history and current form of the phrasal/lexical debate.

Currently, there are two big camps in grammatical theory: the Chomskyan research tradition
\citep{Chomsky81a,Chomsky95a-u} going back to earlier work by \citet{Chomsky57a} and the
more recent framework of Construction Grammar (CxG,
\citealt*{FKoC88a,Goldberg95a,Goldberg2006a,Tomasello2003a}).\footnote{%
  The series editors asked me to modify this sentence since I would run the risk of annoying my
  readers on the first page of my book by stating that there are just two big camps in grammatical theory. I decided to leave the statement as is
  since I think it is the truth. I believe that I can make such a statement since I am working in a
  minority framework myself (Head-Driven Phrase Structure Grammar). I discussed various theoretical frameworks (Categorial Grammar,
  Generalized Phrase Structure Grammar, Lexical Functional Grammar, Tree Adjoining Grammar,
  Dependency Grammar) in \citew{MuellerGT-Eng2}. Mainstream Generative Grammar (GB/Minimalism) and Construction Grammar
  differ from all other frameworks discussed in the book and smaller ones that could not be discussed
  in having various journals and book series exclusively dedicated to research within GB/Minimalism
  and CxG and in the number and size of conferences. A further difference is the number of chairs world wide and
  the number of grant applications per framework.
% Rivista di Grammatica Generativa, Syntax, Linguistic Inquiry, NLLT, 
% LI Monographs, Studies in Generative Grammar
% Constructions and Frames
% Constructional approaches to Language, Benjamins
}
Within the Chomskyan research tradition, \emph{Lectures on Government \& Binding} was very
influential \citep{Chomsky81a}. It initiated a lot of research, both 
in syntax and in language acquisition. Starting with \citet{Chomsky73a} and \citet{Jackendoff77a},
restrictive models of constituent structure were assumed stating that all constituents that are
licensed by a core grammar have the format determined by \xbar schemata. It was argued that there is
a Poverty of the Stimulus from which it follows that there has to be innate domain"=specific
knowledge about linguistics (Universal Grammar, UG). The part of the grammar that is acquired with
the help of this UG is called the core grammar. The rest being the so-called periphery. The \xbar
schemata are rather abstract rules that state that a head combines with its complements to form an
intermediate projection (\mex{1}a) to which adjuncts may be added (\mex{1}b). When a specifier is added a maximal
projection (a complete phrase = XP) results (\mex{1}c).
\eal
\ex XP $\to$ UP \xbar
\ex \xbar $\to$ \xbar YP
\ex \xbar $\to$ X ZP
\zl
In addition to such abstract rules, general principles were assumed. The principles were formulated
in a way that was general enough to make them work for all languages. The differences between
languages were explained with references to parameters that could be set appropriately for a given
language or language class. The parameters were assumed to be each responsible for a variety of
phenomena so that the fixation of one parameter helped children to infer a lot of properties in one
go and hence make language acquisition possible despite the alleged Poverty of the Stimulus.
This general framework was very fruitful and inspired a lot of comparative work. However, it was
realized pretty soon that switch-like parameters cannot be found: it is not the case that a abundance of
phenomena is connected crosslinguistically (Haider \citeyear{Haider94c-u}, \citeyear[Section~2.2]{Haider2001a}; \citealp[Section~16.1]{MuellerGT-Eng1}). There are tendencies, for sure, but no hard switch-like
parameters that work exceptionless for all languages. Furthermore, it has been pointed out that there
are no abrupt changes in language acquisition, something that would be expected if language
acquisition would involve setting binary parameters (\citealp[\page 731]{Bloom93a}; \citealp[\page
  6]{Haider93a}; \citealp[\page 3]{Abney96a}; \citealp[Section~9.1]{AW98a};
\citealp{Tomasello2000a,Tomasello2003a}). 

Another problem with the GB conception of Principles \& Parameters is that the assumed UG is quite
rich: it contains the principles (Case Assignment, Empty Category Principle, Extended Projection
Principle, Subjacency) and on top of this grammatical categories and features, which have to be part
of UG since the principles or the parameters refer to such information. Chomsky's Minimalist Program
addressed the question of how information about such a rich UG is supposed to become part of the
human genome and it was suggested that what is really part of the human genome is the ability to
form recursive structures \citep*{HCF2002a}. There have been several modifications to the rules and
the basic machinery that is assumed and currently there are two basic operations left: External and
Internal Merge \citep{Chomsky2001b-u}. External Merge combines a head with an argument and Internal Merge deletes a
constituent in an existing tree and attaches it at the left periphery.

The Chomskyan division of Core and Periphery was criticized by proponents of Construction Grammar
and the related Simpler Syntax since it was pointed out that a large part of our linguistic
knowledge would be assigned to the Periphery. Now, if we are able to acquire the Periphery, which is
by definition the irregular part of our linguistic systems, why shouldn't we be able to acquire the
more regular parts of the Core? And, indeed, recent advances in statistical methods show that
input-based learning is very likely to be sufficient for language acquisition: statistics-based
determination of part of speech information is quite successful and \citet{Bod2009a} showed how
syntactic structure and in particular auxiliary inversion, Chomsky's standard example in Poverty of the Stimulus
discussions, can be learned from data without running in any Poverty of the Stimulus problems. The
simulations by the group around Freudenthal yielded better explanations of language acquisition phenomena than
earlier generative accounts \citep{FPAG2007a}.

So, Construction Grammarians assume an input-based acquisition of language and reject the assumption
of innate language"=specific knowledge. It is assumed that language acquisition works via
generalization over patterns. For instance, \citet{Tomasello2003a} assumes a transitive construction
consisting of a subject, verb, and object:
\ea
{}[Subj TrVerb Obj]
\z
This can be seen as the generalization over various usage events involving transitive verbs like
those in (\mex{1}):
\eal
\label{Beispiele-fuer-Transitivkonstruktion}
\ex {}[\sub{S} [\sub{NP} The man/the woman] sees  [\sub{NP} the dog/the rabbit/it]].
\ex {}[\sub{S} [\sub{NP} The man/the woman] likes [\sub{NP} the dog/the rabbit/it]].
\ex {}[\sub{S} [\sub{NP} The man/the woman] kicks [\sub{NP} the dog/the rabbit/it]].
\zl
While researchers like \citet{Croft2001a} and \citet{Tomasello2003a} see the pattern in (\mex{-1}) as the result of the
generalization process other researchers assign more structure to sentences with transitive verbs
and assume a VP. Nevertheless, it is obvious that Construction Grammar analyses are rather close to
observable data and that most CxG analyses assume phrasal schemata like (\mex{-1}). 

The following figures show the analysis of (\mex{1}) in Minimalism and in Construction Grammar. 
\ea
Anna reads the book.
\z
The analysis in Figure~\ref{fig-Anna-reads-the-book-flat-CxG} is a completely flat structure as
assumed by Croft and the one in Figure~\ref{fig-Anna-reads-the-book-VP} is an 
analysis with VP as it is assumed in Sign-Based Construction Grammar and the analysis in
Figure~\ref{fig-Anna-reads-the-book-minimalism-TP} is the Minimalist analysis in the version of \citet{Adger2003a}.
\begin{figure}
\centering
\begin{forest}
sm edges
[S
  [NP [Anna]]
  [V  [reads]]
  [NP [the book,roof]]]
\end{forest}
\caption{\label{fig-Anna-reads-the-book-flat-CxG}Analysis of \emph{Anna reads the
    book.} in CxG according to \citet{Croft2001a}}
\end{figure}%
\begin{figure}
\centering
\begin{forest}
sm edges
[S
  [NP [Anna]]
  [VP 
    [V  [reads]]
    [NP [the book,roof]]]]
\end{forest}
\caption{\label{fig-Anna-reads-the-book-VP}Analysis of \emph{Anna reads the
    book.} in Sign-Based CxG according to \citet{Sag2012a}}
\end{figure}%
\begin{figure}
\centering
\begin{forest}
baseline
[TP
 [\textit{Anna} {[D, \st{nom}]}]
 [\tbar{[\st{\textit{u}D*}, \st{nom}]}
   [T{[pres]}]
   [\vP
     [\phonliste{ Anna }]
     [\littlevbar~{[\st{\textit{u}D}]}
       [\textit{v}
         [\textit{read}] [\textit{v} {[\st{acc}]}]]
       [VP
         [\phonliste{ read } {[V, \st{\textit{u}D}]}]
         [DP{[\st{acc}]} [\textit{the book}, roof]]]]]]]
\end{forest}
\caption{\label{fig-Anna-reads-the-book-minimalism-TP}Minimalist analysis of \emph{Anna reads the
    book.} according to \citet{Adger2003a}}
\end{figure}%
As is obvious, the Minimalist analysis is much more complex. It involves additional categories like T
and \textit{v}. On the other hand the combinatorical operations (Merge) are very simple: two
constituents are combined. Which elements are possible in such binary combinations is determined by
features. For instance, verbs have features that correspond to the valence information known from
other theories (\eg LFG and HPSG). 

The general debate is whether such structures can be learned or whether flat or flatter structures
have to be assumed. Another issue is whether syntax is something involving abstract algorithmic
rules like Move and Merge or whether syntax is a set of construction"=specific rules that are
combined with meaning. Semantics plays an important role in language acquisition. Work
in GB and Minimalism usually deals with syntax only and ignores semantics, an exception being work
in the Cartographic tradition of \citet{Rizzi97a-u}. In the latter type of work, information of all levels
is syntactified, and we find semantic categories like Agent and Patient and information structure
categories like Topic and Focus as node labels in syntactic trees. In Construction Grammar, on the
other hand, there is the claim that every construction comes with a certain meaning. Therefore,
syntax and semantics are often treated simultaneously. This is also true for related theories like
Head-Driven Phrase Structure Grammar (HPSG, \citew{ps,ps2,Sag97a}), which I am assuming here.
Construction Grammar assumes that grammar is basically a set of form"=meaning pairs. Lexical items,
phrasal schemata, lexical rules are all form"=meaning pairs. A special case of construction are
so-called \emph{argument structure constructions}\footnote{%
The term \emph{argument structure construction} is an established term in Construction Grammar
  research. See, for instance, some of the paper and book titles in the list of references.
}. The term \emph{argument structure construction} refers to some theoretical entity (lexical item,
lexical rule, phrase structure rule or schema) that licenses arguments. Depending on the authors,
argument structure constructions can be lexical or phrasal constructions. This book is a
contribution to the question of how argument structure constructions should be treated. While
Minimalism assumes that heads select for arguments and abstract rules combine heads with arguments,
most researchers working in Construction Grammar assume that there are very specific constructions
that contribute meaning and license arguments. In what follows, I will introduce the specific topic
of this book in a bit more detail. As I will show, the question is not just Minimalism
vs.\ Construction Grammar since there are other theories that differ considerably from Minimalism,
but nevertheless assume rich lexical items and very abstract combinatorical schemata. So the
question of how arguments and heads should be represented and combined is a very central one that
affects many linguistic frameworks.

\citet{Goldberg95a,Goldberg2006a}, \citet{Tomasello2003a} and others argue for a phrasal view of argument structure
constructions: lexical entries for verbs come with minimal specifications as to which arguments are
required by a verb, but they come with a specification of argument roles. Verbs can be inserted into phrasal
constructions, and these constructions may express the arguments that belong to a verb semantically
or even add further arguments. A frequently discussed example is the one in (\mex{1}):


\ea
\label{ex-runshoes-threadbare}
He runs his sneakers threadbare.
\z
\emph{run} is an intransitive verb, but, in (\mex{0}), it enters the resultative construction,
which licenses an additional argument (\emph{his sneakers}) and a result predicate
(\emph{threadbare}). The resultative semantics is said to be contributed by the whole phrasal
pattern rather than by one of its elements (for instance, Goldberg, \citeyear[\page
  88--89]{Goldberg91a}; \citeyear{Goldberg95a}; \citew[\page 533]{GJ2004a}). The lexical approach assumes that there are several lexical items for verbs
like \emph{run}. There is the lexical item that is needed to analyze simple sentences with the
intransitive verb and its subject, and there is a further lexical item that is used in the analysis
of sentences like (\mex{0}). The latter lexical item selects for a subject, an object and a result
predicate and contributes the resultative semantics. Both lexical items are usually related by a lexical
rule. See \citew{Simpson83a}, \citew{Verspoor97a}, \citew{Wechsler97a}, \citew{WN2001a},
Wunderlich \citeyear[\page 45]{Wunderlich92a-u-kopiert}; \citeyear[\page 120--126]{Wunderlich97c}, \citew{KW98a},
\citew[Chapter~5]{Mueller2002b}, and \citew{Christie2015a-u}
%and \citew{Kay2005a} macht Dative
for lexical analyses in several frameworks.

Lexical approaches usually assume abstract rules or schemata for the combination of lexical
items. For instance, Categorial Grammar assumes functional application and Minimalism assumes
Merge. Head-Driven Phrase Structure Grammar has a Head-Complement Schema and a Specifier-Head
Schema. These abstract schemata are assumed to provide minimal semantic information (functional
application) but do not add any construction"=specific semantics. Construction Grammar proposals like
the one of Tomasello and the one of Goldberg come with strong claims about the non-existence of such
abstract rules \citep[\page 99]{Tomasello2003a}. They argue with respect to language acquisition
that all constructions are phrasal and that what is acquired is phrasal patterns. As is shown
in \citew[Section~11.11.8.1]{MuellerGTBuch1}, \citew{MuellerGT-Eng1} and \citew[Section~9.1]{MWArgSt},
phrasal constructions cannot be the result of language acquisition, it is rather dependencies that
are important for the characterization of the linguistic knowledge of competent speakers. This book
argues that both phrasal constructions in the sense of Construction Grammar and abstract schemata in
the sense of Categorial Grammar, HPSG and Minimalism are needed. Hence, it argues for a broader view
on language that incorporates insights from both schools and fuses them into a new, unified
framework.

The question, whether constructions like (\mex{0}) should be treated as lexical or as
phrasal constructions, has been discussed in the literature in several papers
\citep{GJ2004a,Mueller2006d,Goldberg2013b,MWArgSt} but since most Construction Grammar publications
(intentionally, see \citew[Section~10.4]{Goldberg2006a}) are not formalized, the discussion of aspects not treated
in the original proposal (\eg interaction with morphology, application of the approach to
non"=configurational languages like German, partial verb phrase fronting) was rather
hypothetical. There have been Construction Grammar-inspired proposals in HPSG
\citep{Haugereid2007a,Haugereid2009a} and Simpler Syntax \citep{CJ2005a} and these were shown to
have empirical problems, to make wrong predictions or to be not extendable to other languages
\citep{MuellerUnifying,MuellerGT-Eng1}. Formal CxG proposals (\citealp{BC2005a};
\citealp{vanTrijp2011a}) are discussed in \citew[Chapter~10.6.3]{MuellerGT-Eng1} and
\citew{MuellerFCG}.\footnote{%
  Sign-Based Construction Grammar (SBCG) is also formalized, but SBCG assumes a lexical approach to
  argument structure constructions. \citet*{SBK2012a} are very explicit about this being a
  fundamental property of SBCG and they cite \citew{Mueller2006d} and \citew{MuellerPersian} on
  this. SBCG is a HPSG variant (\citew[486]{Sag2010b}; \citew[Section~10.6.2]{MuellerGT-Eng1}) and
  hence it is no surprise that it is fully compatible with what is argued for in this book.
} Recently, several articles
have been published suggesting a template-based phrasal approach in LFG that makes use of glue semantics, a resource-driven semantic
theory \citep*{Christie2010a,AGT2014a}. While these proposals seem to avoid many of the challenges
that earlier proposals faced, they, in fact, have many of the problems that were discussed with respect
to hypothetical extensions of non"=formal proposals in Construction Grammar. Fortunately, the LFG proposals are worked out in
detail and are embedded in a formal theory that provides formalized analyses of the languages and
phenomena under discussion. It is, therefore, possible to show what the new template-based theories
predict and to pin down exactly the phenomena where they fail. 
%% A fully worked out and implemented lexical analysis in the
%% framework of HPSG exists but is not included into this book due to space limitations. The
%% interested reader is referred to the full paper \citep{MuellerLFGphrasal}.

The traditional analysis of the resultative construction in the framework of LFG is a
lexical one \citep{Simpson83a}, but, more recently, several researchers
have suggested a different view on certain argument structure constructions in the
framework of LFG. For instance, \citet{Alsina96a} and \citet{Christie2010a} suggest analyzing resultative
constructions as phrasal constructions and \citet*{ADT2008a,ADT2013a}  argue for a phrasal
analysis of the (Swedish) caused motion construction. \citet{Toivonen2013a} discusses benefactive constructions of the type in (\mex{1}b):
\eal
\ex The performer sang a song.
\ex The performer sang the children a song.
%% \ex The children were sung a song. 
\zl
Toivonen notices that the benefactive NP cannot be fronted in questions (\mex{1}) and that
passivization is excluded for some speakers of English (\mex{2}).\footnote{%
  See \citet[\page 257]{Hudson92a-u} for references to several papers with varying judgments of
  question formation involving the fronting of the primary object. See \citet{LKD73a} for an experimental study.
} 
\eal
\label{ex-question-formation}
\ex[]{
I baked Linda cookies.
}
\ex[*]{
Who did I bake cookies?
}
\ex[]{
The kids drew their teacher a picture.
}
\ex[*]{
\label{ex-which-teacher-toivonen}
Which teacher did the kids draw a picture?
}
\zl
\ea[*]{
\label{ex-my-sister-was-carved}
My sister was carved a soap statue of Bugs Bunny (by a famous sculptor).
}
\z
%
While Toivonen provides a lexical rule"=based analysis of benefactives in her 2013 paper, she states in
the discussion section:
\begin{quote}
The manipulations that involve the word order consistently render the examples ungrammatical; see
section 2.3 for the relative ordering test, section 2.4 and examples (47--48) for wh-extraction,
section 2.5 for VP anaphora, and section 2.6 for pseudo-clefts. The distribution of benefactive NPs
is thus very limited: it can only occur in the frame given in (5). This does not directly follow
from the analysis given in section 3, and I will not attempt to offer an explanation for these
intriguing facts here. However, it is perhaps possible to adopt an analysis similar to the one Asudeh
et al. (2013) propose for the Swedish directed motion construction (Toivonen 2002). Asudeh et
al. (2013) posit a template that is directly associated with a construction-specific phrase
structure rule. \citep[\page 516]{Toivonen2013a}
\end{quote}
The configuration that she provides in her (5) is given in Figure~\vref{fig-benefactive-toivonen} here.
\begin{figure}
\centering
\begin{forest}
sm edges
[VP
  [V$'$
    [\vnull [bake]]
    [NP, calign with current
      [Mary,roof]]
    [NP [cookies,roof]]]]
\end{forest}
\caption{Phrasal configuration for benefactives according to \citet[\page 505]{Toivonen2013a}}\label{fig-benefactive-toivonen}
\end{figure}
\citet*{AGT2014a} develop the respective phrasal analysis of the benefactive construction.

  Note that Asudeh, Dalrymple, and Toivonen do not argue for a phrasal treatment of argument
  structure constructions in general. They do not assume that there is a phrasal transitive
  construction that licenses arguments for normal sentences like \emph{Kim likes Sandy.} or a
  phrasal ditransitive construction that licenses the objects of normal ditransitive verbs like
  \emph{give}. The authors continue to assume that the arguments of verbs like \emph{like} and
  \emph{give} are specified lexically. They
  just treat certain specific constructions phrasally, namely those that have a fixed
  conventionalized form or special idiosyncratic constraints on order that are difficult to capture lexically.

  %% I think the 2014 paper does not reflect the intuition behind the statement in
  %% \citew{Toivonen2013a} since \citet{AGT2014a} are dealing with the grammar of speakers that permit
  %% passivization (and as I show below also extraction of the secondary object) and hence the structure of the benefactive construction is not fixed. What
  %% I am criticizing here is an approach relying on configurations for phenomena that interact with
  %% valence change and extraction and other phenomena that distort phrasal configurations. 

  Nevertheless, the approach of \citet{AGT2014a} could be seen as a way to formalize phrasal constructional
  approaches like those by \citet{Goldberg95a,Goldberg2004a} and \citet{CJ2005a}. What I want to
  show in this book is that the phrasal LFG approach has
  too many drawbacks in comparison to the lexical approaches. Since the phrasal approach is rejected
  for two specific argument structure constructions (benefactives and resultatives), it follows that it cannot be a viable approach
  for all argument structure constructions. So even though Christie and Asudeh et.\ al.\ do not
  assume that all argument structure constructions should be handled as in phrasal Construction
  Grammar, these two proposals for two specific phrasal constructions can be used to show the problems that approaches have that
  treat all argument structure constructions as phrasal constructions.\footnote{%
    It is clear that other variants of the phrasal approach could exist in principle. It is
    difficult to prove that all imaginable variants of the phrasal approach run into problems. But
    the phenomena and their interaction discussed in this book can serve as a benchmark for
    alternative phrasal theories that may be developed in the future.
}

Another note of caution is necessary here. This book is not a book against Construction
Grammar. There are many versions of Construction Grammar. Most assume a phrasal treatment of
argument structure constructions
\citep{Tomasello2003a,Goldberg95a,Goldberg2006a,GJ2004a,BC2005a,vanTrijp2011a}, but there are variants
like Berkeley Construction Grammar \citep{Kay2005a} and Sign-Based Construction Grammar (SBCG, \citealp*{SBK2012a}) that are explicitly lexical. (See also \citew{Croft2003a} and \citew{Goldberg2013a}
for discussions of lexical and phrasal constructional approaches.) The proposal I work out in this
book in the framework of Constructional Head-Driven Phrase Structure Grammar (Constructional HPSG,
\citealp{Sag97a}) is a lexical constructional proposal. It is equivalent to what would be done in
SBCG, which comes with no big surprise since SBCG is a variant of HPSG \citep[486]{Sag2010b}.

I also do not argue against the attachment of templates to c-structure rules. In fact, it is good to
have this possibility. Such annotated c-structure rules can be used to describe phrasal constructions in which no plausible head
can be identified as, for instance, Jackendoff's N-P-N construction \citeyearpar{Jackendoff2008a}, which is exemplified in
(\mex{1}):
\ea
student after student
\z
Since -- as Jackendoff argued in detail -- no element of this phrase can plausibly be seen as the head
there is no element that could be seen as responsible for the internal structure of the
phrase. Therefore, there is no non-ad hoc lexical item to attach constraints to and attaching
templates to a c-structure seems to be the only option.

This book is structured as follows: Chapter~\ref{sec-template-approach} introduces the
template-based phrasal approach. I then discuss interactions of the resultative
and benefactive construction with extraction, passivization and coordination
(Chapter~\ref{sec-flexibility}). Chapter~\ref{sec-morphology} is devoted to requirements of
morphological processes. 
%Section~\ref{sec-lexical-integrity} deals with Lexical Integrity
%and what kind of generalizations have to be captured in morphology and syntax. 
I then go on to discuss possible
treatments of passivization and point out that generalizations are missed language internally
(Chapter~\ref{sec-missing-generalization-internal}).
Chapter~\ref{sec-generalizations} examines how the analyses could be adapted to German and I argue
that cross"=linguistic generalizations are not captured in phrasal
analyses. Chapter~\ref{sec-lexical-approach-hpsg} develops a lexical approach in the framework of
HPSG, explains how cross"=linguistic generalizations -- including generalizations regarding
constituent structure -- can be captured and shows how restrictions on
extraction and passivization can be captured in a lexical analysis. The book concludes in Chapter~\ref{sec-conclusions}.



\chapter{The template-based approach}
\label{sec-template-approach}

This section introduces two phrasal approaches in more detail. Both approaches are based
on templates \citep{DKK2004a}, glue semantics \citep{Dalrymple99a-ed}, and a version of the Lexical
Mapping Theory \citep{BresnanK89a-u,Kibort2008a}. Glue semantics is interesting since logical
formulae are resource sensitive;
%% \todostefan{MyP: Was ist resource sensitive? Ist nicht jede
%% kompositionelle Semantik resource sensitiv? Und wenn ja, warum ist dann glue semantics besonders interessant?}
that is, certain items have to
be consumed during a semantic combination. This sort of consumption can be used to model valence. I
start with the treatment of benefactives in \citet{AGT2014a} in the following subsection and then
turn to Christie's treatment of resultatives \citeyearpar{Christie2010a}.

\section{Benefactive constructions}

This subsection consists of two parts: I first explain the general assumptions made by
template-based approaches using glue semantics and then comment on why this is different from
earlier inheritance-based proposals and explain why certain problems do not arise and which problems are left.

\subsection{General assumptions and the \templaten{Benefactive} template}

Figure~\ref{fig-kim-ate-at-noon} shows the analysis of (\mex{1}) that is assumed by \citet[\page 75]{AGT2014a}:
\ea
Kim ate at noon.
\z
\begin{figure}%[!h]
%\vspace*{-.5cm}
\centering
\scalebox{1}{
\tree{IP}{
  \tree{\csn{(\up \feat{subj}) $=$ \down}{\rnode{kim}{\raisebox{1ex}{}}DP}}{
    \tri{Kim}
  }
  \tree{\csn{\updown}{\xprime{I}}}{
    \tree{\csn{\updown}{VP}}{
      \tree{\csn{\updown}{\rnode{dcs}{\raisebox{1ex}{}}VP}}{
        \tri{ate}
      }
      \tree{\csn{\down $\in$ (\up \feat{adj})}{PP}}{
        \tri{at noon}
      }
    }
  }
}
}
%
\begin{tabular}{ll}
\protect
 \begin{avm}
    \rnode{fs}{\raisebox{0ex}{}}[pred & \pred{eat}\\
            subj & \rnode{subjl}{}[pred & \pred{Kim}]\rnode{subjr}{}\\
            adj & \{ [``\textup{at noon}''] \}\\
            tense & past]
  \end{avm}\rnode{drink}{\raisebox{-3ex}{}}
&\hspace*{3em}
\protect
\begin{tabular}[t]{l}
\protect
\begin{avm}
\fst{\raisebox{.5ex}{e}}\rnode{d}{}[rel & \textrm{eat}\\
event & \fst{ev}[\phantom{.} & \phantom{.}]\\
arg$_1$ & \fst{\raisebox{.5ex}{k}}\rnode{k}{}[\phantom{.} &
\phantom{.}]\\
arg$_2$ & \fst{\raisebox{.5ex}{p}}\rnode{p}{}[\phantom{.} & \phantom{.}]
 ]
\end{avm}
\end{tabular}
\end{tabular}
%\vspace*{2ex}
\nccurve[ncurvB=1.5,angleA=-180,angleB=-200]{->}{dcs}{fs}
\ncput*[npos=.25]{$\phi$}
\nccurve[angleA=-180,angleB=-240]{->}{kim}{subjl}
\ncput*[npos=.55]{$\phi$}
\nccurve[angleA=0,angleB=-180]{->}{drink}{d}
\ncput*[npos=.55]{$\sigma$}
\nccurve[angleA=-40,angleB=-170]{->}{subjr}{k}
\ncput*[npos=.35]{$\sigma$}
%\vspace*{-1cm}
\caption{Analysis of \emph{Kim ate at noon.} according to \citet[\page 75]{AGT2014a}} \label{fig-kim-ate-at-noon}
%
\end{figure}
There is a constituent structure (c-structure) that is related via the function $\phi$ to a
functional structure (f-structure), which is in turn related to a semantic structure (s-structure) via a further function $\sigma$. The s-structure is a new semantic
level that is supposed to fulfill the function of the argument structure representation
(a-structure) that is usually assumed in versions of LFG that rely on Lexical Mapping Theory
(\citealt{BZ90a}; \citealt[Chapter~14]{BATW2015a}).

The authors follow a neo-Davidsonian approach, that is, verbs introduce a one-place predicate that
takes an event as its sole argument. Further argument roles can be added as predicating of the same
event. For instance, the meaning of \emph{Kim ate} in (\mex{0}) is represented as (\mex{1}), ignoring tense information.
\ea
$eat(e) \wedge agent(e) = kim$
\z
Agents and patients are introduced by \citegen[Section~6.2]{Findlay2016a-u} templates given in (\mex{1}):

\eal
\label{ex-templates-agent-patient-temp}
\ex \label{ex:agent-temp} \templaten{Agent} {\tempeq}\\
  \begin{tabular}[t]{l}
    \template{Arg1} \\[.5ex]
    \pformula{\lambda P\lambda x\lambda e.P(e) ~\wedge~ agent(e) =
      x}{[(\upsig \feat{event}) \linimp\ 
         \upsig] \linimp (\upsig 
         \feat{arg$_1$}) \linimp\ (\upsig \feat{event}) \linimp\
         \upsig}\\
     \end{tabular}

\ex\label{ex:patient-temp} \templaten{Patient} {\tempeq}\\
  \begin{tabular}[t]{l}
    \template{Arg2} \\[.5ex]
    \pformula{\lambda P\lambda x\lambda e.P(e) ~\wedge~ patient(e) =
      x}{[(\upsig \feat{event}) \linimp\ 
         \upsig] \linimp (\upsig 
         \feat{arg$_2$}) \linimp\ (\upsig \feat{event}) \linimp\
         \upsig}\\
     \end{tabular}
\zl
These templates call further templates called \templaten{Arg1} and \templaten{Arg2}, respectively,
and provide a meaning constructor that consists of a lambda expression (line two) and a
glue expression (line three). The lambda expression in both templates is looking for a $P$. This $P$ can only be
combined with the lambda expression if it simultaneously provides the resource {[(\upsig
    \feat{event}) \linimp\ \upsig]}.
%% \todostefan{MyP: Was bedeutet das? Das Prädikat soll eine
%%   Eventvariable enthalten? Bis hierher ist kein Unterschied zu einer Typentheorie erkennbar. Es
%%   sieht nur schwieriger aus.}
After the consumption of this resource the formula {(\upsig 
\feat{arg$_1$}) \linimp\ (\upsig \feat{event}) \linimp\ \upsig} results. This formula states that an
\feat{arg$_1$} has to be found. After combination with \feat{arg$_1$}, the resource (\upsig
\feat{event}) \linimp\ \upsig can be consumed by a tense predicate resulting in \upsigb. That is, we
arrive at a complete semantic proof that has used all resources. The actual proof involving the
templates in (\ref{ex-templates-agent-patient-temp}) is given in Figure~\vref{fig:benefactive-do-proof} and will be discussed in more detail  below.

The templates \templaten{Arg1} and \templaten{Arg2} are defined as shown in (\mex{1}a) and
(\mex{1}b), respectively. For completeness, I also give the definitions of \templaten{Arg3} and \templaten{Arg4}.
\eal
% ARG1, ARG2, ARG3 calling templates
\ex \templaten{Arg1} \tempeq\\
      \{ \template{Map}(\feat{minuso},\feat{arg}$_1$) $\mid$
      \template{NoMap}(\feat{arg}$_1$) \}
\ex \templaten{Arg2} \tempeq\\
      \{ \template{Map}(\feat{minusr},\feat{arg}$_2$) $\mid$
      \template{NoMap}(\feat{arg}$_2$) \}
\ex \templaten{Arg3} \tempeq\\
      \{ \template{Map}(\feat{pluso},\feat{arg}$_3$) $\mid$
      \template{NoMap}(\feat{arg}$_3$) \}
\ex \templaten{Arg4} \tempeq\\
      \{ \template{Map}(\feat{minuso},\feat{arg}$_4$) $\mid$
      \template{NoMap}(\feat{arg}$_4$) \}
\zl
The templates \templaten{Map} and \templaten{NoMap} are used in (\mex{0}) to either map the arguments to a
disjunction of grammatical functions or to declare that they are not mapped to f-structure items at
all. The disjunctions of grammatical functions correspond to disjunctions that are assumed in
Lexical Mapping Theory and are given in (\mex{1}):\footnote{%
  \feat{plusr} is not used anywhere in this book, but it plays a role in the analysis of the passive
  \parencites[\page 319]{Findlay2016a-u}[\page 78]{AGT2014a}.%
}
\eal
% minusr, minuso, plusr, pluso
\ex \feat{minusr} $\equiv$
      \{\feat{subj}$\mid$\feat{obj}\}
      \hfill $[-r]$
\ex \feat{minuso} $\equiv$
      \{\feat{subj}$\mid$\feat{obl}$_\theta$\}
      \hfill $[-o]$
\ex \feat{plusr} $\equiv$
      \{\feat{obl}$_\theta$$\mid$\feat{obj}$_\theta$\}
      \hfill $[+r]$
\ex \feat{pluso} $\equiv$ 
      \{\feat{obj}$\mid$\feat{obj}$_\theta$\}
      \hfill $[+o]$
\zl
The templates \templaten{Map} and \templaten{NoMap} are defined as follows:
\eal
\ex \templaten{Map}(F,A) \tempeq\\  (\up F)\sig $=$  (\upsig A) 
\ex \templaten{NoMap}(A) \tempeq\\  (\upsig A)$_{{\sigma}^{-1}}$ $=$ $\varnothing$  
\zl
The template \templaten{Map} takes its first argument F and states that the value of the $\sigma$
function of the value of F in the f-structure of the mother equals the A value in the $\sigma$ structure
of the mother.

The template \templaten{NoMap} says that the element A in a $\sigma$ structure is not mapped to a
grammatical function in the f-structure that belongs to the A feature (identified via an inverse
function from the semantic structure to the f-structure ($\sigma^{-1}$)). 

If we expand the templates for \templaten{Arg1}, \templaten{Arg2}, and \templaten{Arg3}, we
get:
\eal
% ARG1, ARG2, ARG3 expanded
\label{ex-arg1-arg2-arg3}
\ex \templaten{Arg1} \tempeq\\
      \{ (\up \{\feat{subj}$\mid$\feat{obl}$_\theta$\})\sig $=$  (\upsig \feat{arg$_1$})  $\mid$
      (\upsig \feat{arg$_1$})$_{{\sigma}^{-1}}$ $=$ $\varnothing$ \}
\ex \templaten{Arg2} \tempeq\\
      \{ (\up \{\feat{subj}$\mid$\feat{obj}\})\sig $=$  (\upsig \feat{arg$_2$})  $\mid$
      (\upsig \feat{arg$_2$})$_{{\sigma}^{-1}}$ $=$ $\varnothing$ \}

\ex \templaten{Arg3} \tempeq\\
      \{ (\up \{\feat{obj}$\mid$\feat{obj}$_\theta$\})\sig $=$  (\upsig \feat{arg$_3$})  $\mid$
      (\upsig \feat{arg$_3$})$_{{\sigma}^{-1}}$ $=$ $\varnothing$ \}

\zl
(\mex{0}a) says that either the $\sigma$ value of the \subjlfg is \feat{arg}$_1$ or the $\sigma$ value
  of the \obltheta is \feat{arg}$_1$ or \feat{arg}$_1$ is not realized in the f-structure at all.
(\mex{0}b) says that \feat{arg}$_2$ is mapped to \subjlfg or \obj or to nothing at all and (\mex{0}c)
  says that \feat{arg}$_3$ is mapped to \obj or \objtheta or to nothing at all.


%% \eal
%% \ex (\up \feat{minuso})\sig $=$  (\upsig \feat{arg}$_1$) 
%% \ex (\up \feat{minusr})\sig $=$  (\upsig \feat{arg}$_2$) 
%% \ex (\up \feat{pluso})\sig $=$  (\upsig \feat{arg}$_3$) 
%% \ex (\up \feat{minuso})\sig $=$  (\upsig \feat{arg}$_4$)
%% \zl


For verbs like \emph{draw}, which have both an agent and a patient, the templates for agent and patient can be combined into one template as in (\mex{1}):
\ea
% agent-patient
\templaten{Agent-Patient} {\tempeq}\\
     \begin{tabular}[t]{l}
       \template{Agent}\\
       \template{Patient} \\[.5ex]
     \end{tabular}
\z
Finally we need the template \templaten{Past} in (\mex{1}):
\ea
\templaten{Past} {\tempeq}\\
     \begin{tabular}[t]{l}
     (\up \feat{tense}) $=$ \feat{past}\\[.5ex]
     \pformula{\lambda P\exists e.[P(e) ~\wedge~ past(e)]}
     {[(\upsig \feat{event}) \linimp\ 
         \upsig] \linimp \upsig}\\
   \end{tabular}
\z
This template adds the \feat{tense} feature and the value \feat{past} to the f-structure, adds the
past semantics to an event and states a glue term that requires something that takes an event and
licenses a complete $\sigma$ structure[(\upsig \feat{event}) \linimp\ \upsig]. If this resource is
found, a complete $\sigma$ structure \upsig results.
%\todostefan{check}

With the template for \templaten{Past} in place, we can now have a look at the lexical entry for \emph{drew} in (\mex{1}):
\ea
% drew
\label{ex:draw-lex1}
  \begin{tabular}[t]{lll}
    \multicolumn{2}{l}{\word{drew}} & {V} \\
    \begin{tabular}[t]{l}
      (\up \feat{pred}) $=$ \pred{draw}\\
      \template{Past}\\
      \template{Agent-Patient}\\[.5ex]
      \formula{\lambda e.draw(e):(\upsig \feat{event}) \linimp\
         \upsig}
    \end{tabular}
  \end{tabular}
\z
The specification of the \predv in (\mex{0}) is unusual for LFG. Usually, \predvs come with a
specification of grammatical functions that have to be realized together with a predicate. The \predv is
the representation of valence information in LFG. This function is taken over by glue terms in
proposals that use glue semantics. Since glue semantics is resource sensitive, one can set things up
in a way to make sure that all the grammatical functions that are required to fill semantic roles
are realized in an utterance. 

If we expand the template calls, we get the f-structure constraints and semantic constructors in
(\mex{1}):
\ea
  \begin{tabular}[t]{lll}
    \multicolumn{2}{l}{\word{drew}} & {V} \\
    \begin{tabular}[t]{l}
      (\up \feat{pred}) $=$ \pred{draw}\\

% PAST
     (\up \feat{tense}) $=$ \feat{past}\\[.5ex]

% ARG1 from agent-template
      \{ (\up \{\feat{subj}$\mid$\feat{obl}$_\theta$\})\sig $=$  (\upsig \feat{arg}$_1$)  $\mid$
      (\upsig \feat{arg}$_1$)$_{{\sigma}^{-1}}$ $=$ $\varnothing$ \} \\[.5ex]

% ARG2 from patient-template
      \{ (\up \{\feat{subj}$\mid$\feat{obj}\})\sig $=$  (\upsig \feat{arg}$_2$)  $\mid$
      (\upsig \feat{arg}$_2$)$_{{\sigma}^{-1}}$ $=$ $\varnothing$ \}\\[.5ex]

% PAST meaning
     \pformula{\lambda P\exists e.[P(e) ~\wedge~ past(e)]}
     {[(\upsig \feat{event}) \linimp\ 
         \upsig] \linimp \upsig}\\

% ARG1 agent
    \pformula{\lambda P\lambda x\lambda e.P(e) ~\wedge~ agent(e) =
      x}{[(\upsig \feat{event}) \linimp\ 
         \upsig] \linimp (\upsig 
         \feat{arg$_1$}) \linimp\ (\upsig \feat{event}) \linimp\
         \upsig}\\

% ARG2 patient
    \pformula{\lambda P\lambda x\lambda e.P(e) ~\wedge~ patient(e) =
      x}{[(\upsig \feat{event}) \linimp\ 
         \upsig] \linimp (\upsig 
         \feat{arg$_2$}) \linimp\ (\upsig \feat{event}) \linimp\
         \upsig}\\

% draw meaning
      \formula{\lambda e.draw(e):(\upsig \feat{event}) \linimp\
         \upsig}
    \end{tabular}
  \end{tabular}
\z
The glue terms can be used in a proof as is shown in the box for \relation{draw} in
Figure~\vref{fig:benefactive-do-proof}. The proofs are basically lambda reductions with the
additional condition that resources that are paired with the lambda expression (the material to the
right of the colon) have to be used. So, for instance, when \template{Patient} is combined with
\textit{drew}, the resource (ev \linimp d) \linimp s \linimp ev \linimp d has to be used. Since
\textit{drew} provides ev \linimp d, the combination of the two items results in s \linimp ev
\linimp d. In the next step, an x:s is hypothesized, lambda reduction takes place and the resource
s is consumed yielding  ev \linimp d. This expression is combined with
\template{Agent}. \template{Agent} contains the glue term (ev \linimp d) \linimp k \linimp ev
\linimp d and since \template{Patient} + \textit{drew} was ev \linimp d, a combination is possible
and the result is k \linimp ev \linimp d. Now the x:s that was hypothesized earlier is reintroduced
into the formula resulting in s \linimp k \linimp ev \linimp d.

\citet*[\page 81]{AGT2014a} assume that information about benefactive arguments is introduced by the
c-structure rule in (\mex{1}):
\ea\label{c-struc-vp-benefactive}
\phraserule{V$'$}{
\rulenode{V\\* \up~=~\down\\*( @\textsc{Benefactive} )}
\rulenode{DP\\*(\up\ \obj) = \down}
\rulenode{DP\\*(\up\ \objtheta) = \down}\hspace{-2ex}
}
\z
The \templaten{Benefactive} template is specified in brackets, which -- in the context of template calls -- marks optionality. So the
c-structure rule can be used with normal ditransitive verbs or with transitive verbs and, in this
case, the \templaten{Benefactive} template would apply and license a further argument.

The \templaten{Benefactive} template is defined as follows:
\ea
   \templaten{Benefactive} {\tempeq}\\*
   \oneline{%
\begin{tabular}[t]{@{}l@{}}
     \template{Arg3}\\[.5ex]
   \pformula{\lambda x\lambda y\lambda P\lambda e.P(y)(e) ~\wedge~ \textit{beneficiary}\hspace{.1em}(e) = x}
     {(\upsig \feat{arg$_2$}) \linimp 
      (\upsig \feat{arg$_3$}) \linimp\newline {}[(\upsig \feat{arg$_2$}) \linimp (\upsig \feat{event}) \linimp
     \Upsig] \linimp
      (\upsig \feat{event}) \linimp \upsig}
  \end{tabular}}
\z
As \citet[\page 78]{AGT2014a} state, the template uses a trick. It first looks for \feat{arg$_2$} and
\feat{arg$_3$} and then combines with a verb looking for an \feat{arg$_2$}. In this way the resource
logic basically maps a two-place predicate to a three-place predicate.

If we expand the call to the \templaten{Arg3} template, we get (\mex{1}):
\ea
%\vspace{-\baselineskip}
\scalebox{.93}{%
%\oneline{%
\begin{tabular}[t]{@{}l@{}}
% ARG3
      \{ (\up \{\feat{obj}$\mid$\feat{obj}$_\theta$\})\sig $=$  (\upsig \feat{arg}$_3$)  $\mid$
      (\upsig \feat{arg}$_3$)$_{{\sigma}^{-1}}$ $=$ $\varnothing$ \}\\
% BENEFACTIVE
   \pformula{\lambda x\lambda y\lambda P\lambda e.P(y)(e) ~\wedge~ \textit{beneficiary}\hspace{.1em}(e) = x}
     {(\upsig \feat{arg$_2$}) \linimp 
      (\upsig \feat{arg$_3$}) \linimp\newline {}[(\upsig \feat{arg$_2$}) \linimp (\upsig \feat{event}) \linimp
     \Upsig] \linimp
      (\upsig \feat{event}) \linimp \upsig}
\end{tabular}}%\hspace{-3em}\mbox{}
\z
In an analysis of (\mex{1}), we would hence have the constraints on the lexical item for \emph{drew}
given in (\ref{ex:draw-lex1}) and the constraints in (\mex{0}).
\ea
Kim drew Sandy Godzilla.
\z
This means that the grammatical functions of the arguments are underspecified in the c-structure annotations of the lexical item and
the benefactive template. What we have so far is the set of constraints given in
(\ref{ex-arg1-arg2-arg3}). In order to get these disjunctions
resolved\label{page-disjunctions-gf-c-structure}, 
% Mapping auf varnothing ist durch die Templates ausgeschlossen
%and in order to exclude the option of \argone, \argtwo and \argthree being mapped to $\varnothing$
we need c-structure rules. In the case at hand we have the c-structure rule in (\ref{c-struc-vp-benefactive}) that
licenses the objects and we have an IP rule that combines the VP with an NP/DP. This c-structure
rule ensures that there is a \subjlfg. Without these additional constraints from c-structure
configurations, the approach would overgenerate. As I will show in Chapter~\ref{sec-active-passive}, this is
problematic since the assignment of grammatical functions in passives has to be taken care of by
c"=structure rules that are specific to the benefactive construction, which results in missing
generalizations.%\todostefan{ncline does not work properly in texlive 2017, only 2016 works}%\todostefan{kerning of \emph{beneficiary} is broken}

\begin{figure}
  \centering
%\begin{sideways}
\ncline{-}{d1}{d2}

\oneline{  
  \rnode{d1}{\formula{draw^\prime =} \fbox{
      \scalebox{0.8}{
  \begin{prooftree}
    \[
    \vmformula[\template{Agent}]{\lambda P\lambda y\lambda e.P(e)
      \wedge agent(e) = y}{(ev \linimp d) \linimp k \linimp\ ev
      \linimp d}
    \[
    \[
    \vmformula[\template{Patient}]{\lambda P\lambda x\lambda e.P(e)
      \wedge patient(e) = x}{(ev \linimp d) \linimp s  \linimp\ ev
      \linimp d}
    \hspace*{1em}
    \vmformula[drew]{\lambda e.draw(e)}{ev \linimp d}
    \justifies
    \formula{\lambda x\lambda e.draw(e) \wedge patient(e) = x:s \linimp
      ev \linimp d}
    \]
    \formula{[x:s]}^1
    \justifies
    \formula{\lambda e.draw(e) \wedge patient(e) = x:ev \linimp d}
    \]
    \justifies
    \formula{\lambda y\lambda e.draw(e) \wedge patient(e) = x \wedge
      agent(e) = y:k \linimp ev \linimp d}
    \]
    \justifies
    \formula{\lambda x\lambda y\lambda e.draw(e) \wedge patient(e) = x \wedge
      agent(e) = y:s \linimp k \linimp ev \linimp d}
    \using \linimpIi{1}
  \end{prooftree}
  }}}
}

\bigskip



\oneline{
\begin{prooftree}
\[
\vmformula[\template{Past}]{\lambda P\exists e.[P(e) \wedge
  past(e)]}{(ev \linimp d) \linimp d}
\[
\[
\[
\vmformula[\template{Benefactive}]{\lambda x\lambda y\lambda P\lambda
  e.P(y)(e) \wedge 
     \textit{beneficiary}\hspace{.1em}(e) = x}{s \linimp g \linimp (s \linimp ev
  \linimp d) \linimp ev \linimp d}
\hspace*{1em}
\vmformula[Sandy]{sandy}{s}
\justifies
\mformula{\lambda y\lambda P\lambda
  e.P(y)(e) \wedge 
     \textit{beneficiary}\hspace{.1em}(e) = sandy}{g \linimp (s \linimp ev
  \linimp d) \linimp ev \linimp d}
\]
\vmformula[Godzilla]{godzilla}{g}
\justifies
\mformula{\lambda P\lambda
  e.P(godzilla)(e) \wedge 
     \textit{beneficiary}\hspace{.1em}(e) = sandy}{(s \linimp ev
  \linimp d) \linimp ev \linimp d}
\]
\[
\[
\[
\rnode{d2}{\fbox{\formula{draw^\prime}}}
\hspace*{2em}
\formula{[z:s]}^2
\justifies
\formula{draw^\prime(z):k \linimp ev \linimp d}
\]
\vmformula[Kim]{kim}{k}
\justifies
\formula{draw^\prime(z)(kim):ev \linimp d}
\]
\justifies
\formula{\lambda z.draw^\prime(z)(kim):s \linimp ev \linimp d}
\using \linimpIi{2}
\]
\justifies
\formula{\lambda
  e.draw^\prime(godzilla)(kim)(e) \wedge 
     \textit{beneficiary}\hspace{.1em}(e) = sandy:ev \linimp d}
\]
\justifies
\formula{\exists e.[draw^\prime(godzilla)(kim)(e) \wedge 
     \textit{beneficiary}\hspace{.1em}(e) = sandy \wedge past(e)]:d}
\]
\justifies
\formula{\exists e.[draw(e) \wedge patient(e) = godzilla \wedge
  agent(e) = kim \wedge 
     \textit{beneficiary}\hspace{.1em}(e) = sandy \wedge past(e)]:d}
\using \betared
\end{prooftree}
}
%\end{sideways}
\caption{Proof for \textit{Kim drew Sandy Godzilla.}}
\label{fig:benefactive-do-proof}
\vfill
\ncline{-}{d1}{d2}
\end{figure}

\subsection{Inheritance-based analyses: Why do they work and where are the limits}

\citet{MuellerPersian} argued that semantics needs embedding and cannot be done in inheritance
networks. For instance, it was suggested to do morphology by inheritance. As \citet{KN93a} show this
fails for adjectives like \emph{undoable}. \emph{undoable} has two possible meanings that correspond
to two bracketings: \emph{undo-able} and \emph{un-doable}. In the first reading, the verb \emph{undo}
is combined with \suffix{able}; the verbal meaning is embedded under a modal operator. In the second
reading, the prefix \prefix{un} is attached to the adjective \emph{doable} and embeds the meaning
of \emph{doable} under the negation. If semantic information is the value of a feature and if
the properties of \emph{undoable} are inherited from \prefix{un}, \emph{do}, and \suffix{able}, we
get a conflict because rather than inheriting three incompatible semantic contributions from the
verb and the affixes, the contribution of the verb has to be embedded under the contribution of one
of the affixes and the result has to be embedded under the contribution of the other affix.  

\citet{MuellerLehrbuch1}, \citet{MuellerUnifying} and \citet{MWArgSt} argued that argument structure changing
phenomena cannot be treated via inheritance but need formal means that map representations to other
representations. An example for such problems are causative constructions in languages like Turkish.
Such causative constructions license additional arguments and they can be iterated. The analysis of
\citet{AGT2014a} seems to falsify my claims and seems to suggest that there is a way to analyze
argument structure constructions phrasally with inheritance of constraints playing an important role.

Traditional a-structure-based LFG approaches assume that sentences with different argument
realizations have different lexical items with different argument structure representations. The
argument structures are mapped to grammatical functions and these are realized according to the
syntax of the respective languages. For instance, \citet[Section~14.4.5]{BATW2015a} assume the following
a-structures for the transitive and the ditransitive use of \emph{cook}:
\eal
\ex \begin{tabular}[t]{@{}l@{}}
    Transitive:\\
    \begin{tabular}{@{}llc@{~~}c@{~~}c@{~~}c@{}}
    a-structure:  & \emph{cook} & $\langle$ & agent  & patient & $\rangle$\\
                  &             &           & [$-$o] & [$-$r]\\\\
    f-structure:  &             &           & \subjlfg  & \obj\\
    \end{tabular}
    \end{tabular}
\ex Ditransitive:\\
    \begin{tabular}{@{}llc@{~~}c@{~~}c@{~~}c@{~~}c@{}}
    a-structure:  & \emph{cook-for} & $\langle$ & agent  & beneficiary & patient & $\rangle$\\
                  &                 &           & [$-$o] & [$-$r]      & [$+$o]\\\\
    f-structure:  &                 &           & \subjlfg  & \obj        & \objtheta\\
    \end{tabular}
\zl
\largerpage
Lexical Mapping Theory makes sure that the arguments that are labeled with $-$o, $-$r and $+$o are
mapped to the respective grammatical functions. The important point about this analysis is that
there are two lexemes: one for transitive \emph{cook} with an a-structure that contains two elements and
one for the ditransitive version with an a-structure that contains three elements. The a-structures
are ordered lists with a fixed arity and it is impossible to add an element into the middle of such a list by a monotonic
gathering of constraints (\eg inheritance).\footnote{%
  It is possible to extend lists at the end if defaults and overriding are permitted. See
  \citew{MuellerDefaults} for discussion. It is also possible to leave the number of elements in a
  list underspecified and state constraints on membership and order in such lists. See \citew[Section~7.5.2]{MuellerLehrbuch1} for problems of
  such accounts.
}
The template-based approach circumvents this problem by not stipulating an order of elements in a
list. Rather than using an ordered representation like lists, it assumes an s-structure into which
features can be added by simple unification. These features are not ordered. The feature names have
numbers as part of the names but this is just mnemonic and if order effects are desired they have to
be modeled elsewhere. \citet{AGT2014a} impose the order"=specific constraints in the glue part of their semantic expressions. For
instance, the \templaten{Benefactive} template refers to \argtwo and \argthree and consumes
respective resources in a specified order.
%% \todostefan{MyP: Was sind die Beschränkungen von
%%   \templaten{Benefactive}? Was ist der Unterschied zwischen \emph{helfen} und \emph{unterstützen}?}

Turning to semantics, the claims regarding inheritance and embedding are true for frameworks in which
the semantic contribution is represented as a value of a feature (HPSG, \citealt{ps2,Sag97a}; BCG,
\citealt{KF99a}; SBCG, \citealt{Sag2012a}; FCG, \citealt{SteelsFluid-ed}). If two different semantic values are inherited from
supertypes, a conflict arises. To take an example, consider the \templaten{Agent} and the
\templaten{Patient} template. If we assumed that the meaning-constructor is the value of a feature,
say \textsc{sem}, we would have two conflicting values:
\eal
\ex \pformula{\lambda P\lambda x\lambda e.P(e) ~\wedge~ agent(e) =
      x}{[(\upsig \feat{event}) \linimp\ 
         \upsig] \linimp (\upsig 
         \feat{arg$_1$}) \linimp\ (\upsig \feat{event}) \linimp\
         \upsig}
\ex \pformula{\lambda P\lambda x\lambda e.P(e) ~\wedge~ patient(e) =
      x}{[(\upsig \feat{event}) \linimp\ 
         \upsig] \linimp (\upsig 
         \feat{arg$_2$}) \linimp\ (\upsig \feat{event}) \linimp\
         \upsig}
\zl


Note that representing these semantic contributions in lists would not help either, since this would
just shift the conflict to another place. Lists are ordered and if (\mex{0}a) is the first member of
a list and (\mex{0}b) is the first member of a second list, the two lists are incompatible. 
In order to avoid such conflicts auxiliary features and mappings
between auxiliary features may be used \citep{Koenig99a}. The problem is that one auxiliary feature
per interaction is needed \citep[Section~7.5.2.2]{MuellerLehrbuch1}.

Assuming sets rather than lists would not work either, if the general understanding of sets as is
common in HPSG \citep{PM90a} is assumed. What could be done is that one inherits constraints on list
or set membership. The \templaten{Agent} and \templaten{Patient} templates would then have the
following feature-value specification:
\eal
\ex \textsc{sem} \ibox{1} $\wedge$ \pformula{\lambda P\lambda x\lambda e.P(e) ~\wedge~ agent(e) =
      x}{[(\upsig \feat{event}) \linimp\ 
         \upsig] \linimp (\upsig 
         \feat{arg$_1$}) \linimp\ (\upsig \feat{event}) \linimp\
         \upsig} $\in$ \ibox{1}
\ex \textsc{sem} \ibox{1} $\wedge$ \pformula{\lambda P\lambda x\lambda e.P(e) ~\wedge~ patient(e) =
      x}{[(\upsig \feat{event}) \linimp\ 
         \upsig] \linimp (\upsig 
         \feat{arg$_2$}) \linimp\ (\upsig \feat{event}) \linimp\
         \upsig} $\in$ \ibox{1}
\zl
So one would say that the value of \textsc{sem} is a set \iboxb{1} and that the meaning constructor for \textsc{agent}
is an element of this set and that the meaning constructor for \textsc{patient} is an element of this set
too. Note that the set is not constrained otherwise, in principle any formula could be part of
this set. So one would need the additional assumption that we are looking for minimal models when we
interpret linguistic structures, an assumption that is usually made in LFG.


%% Interestingly this problem does not arise in the glue-based approach since \citet{AGT2014a} do not
%% assume that their semantic formulae are values of a feature. Rather they just collect the
%% semantic-constructors without attaching them to a feature.\footnote{%
%%   HPSG allows for the attachment of relational constraints to AVMs or types. The free-floating
%%   meaning constructors in Asudeh et. al.'s system can be understood similarly.
%% }

In general, such a system of semantics construction would not work since it would not be clear in
which order partial formulae that are inherited from supertypes are to
be combined. Authors have used
semantic types in order to make it clear what type of argument has to be combined with a certain
functor (\eg in GPSG, \citealt[Chapters~9--10]{GKPS85a}), but this does not help in all cases. The glue approach has
additional means to specify what is combined with what: specific resources are used when elements are
combined. So, while the lambda expressions for the agent and the patient template in (\mex{0}) are
identical, the glue resources are not. The \templaten{Agent} template involves an \argone and the \templaten{Patient}
template an \argtwo. Furthermore, the glue apparatus can be used for mapping predicates of a certain
arity to predicates of another arity. For instance, the \templaten{Benefactive} template requires an \feat{arg$_2$} and an \feat{arg$_3$}
and then a verb that selects for an \feat{arg$_2$} ([(\upsig \feat{arg$_2$}) \linimp (\upsig \feat{event}) \linimp \Upsig]). 
\ea
Part of the \templaten{Benefactive} template that remaps \argtwo to \argthree:\\
(\upsig \feat{arg$_2$}) \linimp (\upsig \feat{arg$_3$}) \linimp {}[(\upsig \feat{arg$_2$}) \linimp (\upsig \feat{event}) \linimp \Upsig]
\z
This basically turns a two-place verb selecting for an object (\argtwo) into a three-place verb that
has a new first object (\argtwo) and realizes the object of the two-place verb as its second object
(\argthree). The glue term basically does what a lexical rule does in
lexical rule-based systems, it maps a two-place predicate to a three-place predicate:
\ea
\sliste{ \feat{arg$_x$}, \feat{arg$_y$} } $\mapsto$ \sliste{ \feat{arg$_x$}, \feat{arg$_z$}, \feat{arg$_y$} }
\z
So a lexical item with several glue constraints attached to it corresponds to a lexical item with
several lexical rules attached to it (for later application). The resource sensitivity of the glue statements ensures that
the glue statements are used in a specific order in the proofs. Similarly, the input and output
conditions of lexical rules make sure that they are applied in a certain order.\footnote{%
See, for instance, \citew[\page 515]{Blevins2003a} for the application of the impersonal lexical rule to the output of the
passivization lexical rule. The output of the impersonal lexical rule cannot function as input to
passivization since passivization requires a subject to be suppressed and the subject was already
suppressed by the impersonalization.}

%% Anna Kibort 27.07.2016: Impersonal constructions assumes PRO/pro for the impersonal construction.
%% My thought: This automatically gets the case right. It just behaves as if there was a subject.

%% For an LFG grammar to work it is not sufficient to know how many arguments are realized, the
%% arguments have to be mapped onto grammatical functions to ensure correct positioning in c-structures
%% and correct case assignment. The grammatical functions for \argone, \argtwo and \argthree are specified disjunctively as is shown in
%% (\ref{ex-arg1-arg2-arg3}). This basically leaves room for all arguments to move up one step: the
%% subject can be dropped in the passive, the \obj can be realized as \subjlfg, and the \objtheta can
%% be realized as \obj. This works well for the normal passive in English, for which \citet{AGT2014a}
%% assume the following template:
%% \eas
%% \templaten{Passive} {\tempeq}\\
%%    \begin{tabular}[t]{l}
%%     (\up \feat{voice}) $=$ \fval{passive}\\
%%     \template{AddMap}(\feat{plusr},\feat{arg$_1$})\\[.5ex]
%%   ( \formula{\lambda P\exists
%%                 x.[P(x)]:[(\upsig \feat{arg$_1$})
%%                 \linimp\ \upsig] 
%%                 \linimp\ \upsig} )
%%   \end{tabular}
%% \zs
%% \feat{plusr} stands for \{\feat{obl}$_\theta$$\mid$\feat{obj}$_\theta$\} and since \argone was
%% specified to be \{\feat{subj}$\mid$\feat{obl}$_\theta$\}, the only option to realize the \argone is
%% as \feat{obl}$_\theta$.

%% However, this system fails in accounting for the interaction of the passive and impersonal constructions that
%% was described for languages like Lithuanian \citep[Section~5]{Timberlake82a},
%% Irish \citep{Noonan94a}, and Turkish \citep{Ozkaragoez86a} in the framework of Relational Grammar.
%% The cited authors discussed their data as instances of double passivization, but it was
%% argued by \citet{Blevins2003a} that these and similar examples from other languages are impersonal
%% constructions that can be combined with personal passives. We
%% will use Özkaragöz's Turkish examples in \pmex{1} for illustration \citeyearpar[\page
%% 77]{Ozkaragoez86a}.
%% \eal
%% \ex\label{ex-double-passivization-strangle}
%% \gll Bu şato-da boğ-ul-un-ur.\\
%%      this chateau-{\sc loc} strangle-{\sc pass}-{\sc pass}-{\sc aor}\\
%% \glt `One is strangled (by one) in this chateau.'
%% \ex\label{ex-double-passivization-hit}
%% \gll Bu oda-da döv-ül-ün-ür.\\
%%      this room-{\sc loc} hit-{\sc pass}-{\sc pass}-{\sc aor}\\
%% \glt `One is beaten (by one) in this room.'
%% \ex
%% \gll Harp-te vur-ul-un-ur.\\
%%      war-{\sc loc} shoot-{\sc pass}-{\sc pass}-{\sc aor}\\
%% \glt `One is shot (by one) in war.'
%% \zl
%% \suffix{In}, \suffix{n}, and \suffix{Il} are allomorphs of the passive morpheme. According to
%% Özkaragöz the data is best captured by an analysis that assumes that the passive applies to a
%% passivized transitive verb and hence results in an impersonal passive. The examples in (\mex{0}) use
%% his glossing. If Blevins analysis is correct, the last morpheme glossed as passive in (\mex{0}) has
%% to be glossed as impersonal.


%% The generalization about the impersonal construction is that the most prominent argument is
%% suppressed (the subject of a transitive, the subject of an unergative or unaccusative, or the subject
%% of a passive). In Asudeh et. al.'s system there is no way of finding out which element is the most
%% prominent one: it could be be \feat{arg$_1$} if we have a transitive in the impersonal construction but if the impersonal
%% construction applies to a passivized verb then it is \feat{arg$_2$} that is suppressed. Of course
%% one could set up an impersonal template that is parallel to the passive one and add a disjunction
%% that suppresses an \argtwo, but the point is that \argtwo can only be suppressed if \argone is
%% suppressed already. In order to model this with glue terms it seems to be necessary that the
%% impersonal construction has to ``know'' that there are certain effects that can be caused by other
%% constructions like the passive. In fact this would be equivalent to stating a separate impersonal +
%% passive construction, a missed generalization \citep[\page 51]{MWArgSt}.
%% The problem
%% with the glue approach to passive is that \argone, \argtwo and \argthree are not affected by glue
%% statements. They remain in the $\sigma$ structure without any order. What is needed instead is something that maps one
%% representation onto another one. In lexical rule-based approaches to passive \citep{ps,Mueller2003e,Blevins2003a}, the \argone is suppressed. This
%% results into a new configuration in which \argtwo is the first element:
%% \ea
%% \sliste{ \feat{arg$_1$}, \feat{arg$_2$}, \feat{arg$_3$} } $\mapsto$ \sliste{ \feat{arg$_2$}, \feat{arg$_3$} }
%% \z
%% In theories that map representations to other representations, the impersonal construction can be
%% described as suppressing the least oblique argument, that is, the first argument in a list. If the
%% impersonal applies to the unreduced valence list \argone is the first element and gets suppressed,
%% if it applies to a passivized verb, \argtwo is the first and gets suppressed.

%% This seems to be the most natural analysis to impersonals to me. In a standard approach to passive
%% in Lexical Mapping Theory the first argument is kept in the a-structure but it is mapped to
%% zero. Since the a-structure is an ordered list, one can say that the impersonal maps the first
%% non-zero grammatical function to zero.


%% Another phenomenon that is problematic for the template-based approach is causatives. In causatives
%% a causative-morpheme licenses an additional argument. \citet*{MSI99a} argued for a lexical treatment
%% of causatives in Japanese, pointing out for instance that verbs with a causative affix behave in
%% many respects like simplex verbs with the same valence. The interesting point about causatives is
%% that causative affixes can attach to monovalent or bi-valent verbs. In the glue-based approach one
%% would have to say that an \argone is mapped onto \argtwo and \argtwo is mapped to \argthree and so
%% on. So the causative template would have to distinguish between verbs that take one or two or more
%% arguments. (\mex{1}a) shows the case in which an mono-valent verb is combined a causative morpheme:
%% a verb requiring an \argone is combined with an \argtwo (the former \argone) and a new argument
%% \argone, which is the causer. (\mex{1}b) shows the case in which a mono-transitive verb requiring an
%% \argtwo and an \argone is combined with an \argthree (the former \argtwo), an \argtwo (the former
%% \argone) and a new \argone (the causer).
%% \eal
%% \ex part of the hypothetical causative template for intransitive verbs:\\
%% (\upsig \feat{arg$_2$}) \linimp (\upsig \feat{arg$_1$}) \linimp {}[(\upsig \feat{arg$_1$}) \linimp (\upsig \feat{event}) \linimp \Upsig]
%% \ex part of the hypothetical causative template for mono-transitive verbs:\\
%% (\upsig \feat{arg$_3$}) \linimp (\upsig \feat{arg$_2$}) \linimp (\upsig \feat{arg$_1$}) \linimp {}[(\upsig \feat{arg$_2$}) \linimp (\upsig \feat{arg$_1$}) \linimp (\upsig \feat{event}) \linimp \Upsig]
%% \zl
%% %
%% % Das geht wohl, weil dann zwei Sachen auf ARG2 gemappt würden und PRED-Werte immer verschieden
%% % sind.
%% %
%% %% The problem with this disjunctive specification of the semantics and glue terms for a causative
%% %% affix is that semantics does not distinguish between lexical items and their projections. For
%% %% semantics a transitive verb + object looks exactly the same as an intransitive verb. So if we have a
%% %% bivalent verb and combine it with its object it would be possible to combine the result 

%% A lexical rule-based approach could capture the relation between a certain lexeme and a lexeme with
%% added causer directly:
%% \ea
%% \ms{
%% arg-st & \ibox{1}
%% } $\mapsto$ 
%% \ms{
%% arg-st & \sliste{ NP } $\oplus$ \ibox{1}
%% }
%% \z
%% The lexical rule would map an item with a certain argument structure list \ibox{1} onto one with an
%% argument structure list that contains the causer (NP) in addition to the argument structure
%% list of the input of the rule \ibox{1}. The number of elements in \ibox{1} is irrelevant for the
%% rule as such. Of course there are constraints on possible argument lists that constrain the
%% categories and case of the elements in the output list.

%% Note that the lexical rule"=based approach does not have any problems with linking and
%% semantics. The following rule says that the semantic contribution of the input lexical item is
%% embedded as an argument of the \relation{cause} relation. The linking of the arguments contained in
%% the list \ibox{1} to arguments in the semantic contribution in \ibox{2} remains in tact in the
%% output of the lexical rule. What is added is the NP for the causer and the linking of this element
%% to the \textsc{causer} role.\footnote{%
%%   See \citew{MSI99a} for an explicit proposal for causatives in Japanese along similar lines.
%% }
%% \ea
%% \ms{
%% arg-st & \ibox{1}\\
%% cont   & \ibox{2}
%% } $\mapsto$ 
%% \ms{
%% arg-st & \sliste{ NP\ind{3} } $\oplus$ \ibox{1}\\[2mm]
%% cont   & \ms[cause]{
%%          causer & \ibox{3}\\
%%          effect & \ibox{2}
%%          }
%% }
%% \z
%% It is unclear how the template-based approach deals with the semantic embedding that is required on
%% the $\sigma$ structure. A lot of remapping seems to be required to make the template"=based approach
%% work. 

%% It should also be noted here that the the question of whether causatives should be licensed
%% lexically or phrasally is not at issue here. The question is what kind of mappings can be done with
%% the template/glue machinery and it seems to be the case that causatives cannot be handled adequately
%% in the approach under discussion. The basic problem is that $\sigma$ structures do not impose an order on
%% the elements contained in them. The feature names seem to suggest an order but they are just
%% mnemonic, features in AVMs are not ordered. So the order of arguments is only reflected by the order
%% of application of glue resources. If arguments are added the glue terms have to be manipulated.

\section{Resultative constructions}

\citet{Christie2010a} assumes the c-structure rule in (\mex{2}) for transitive resultative
constructions like the one in (\mex{1}):
\ea
He hammered the metal flat.
\z
\ea
\label{ex-christie-result-c-structure}
\phraserule{V$'$}{
\rulenode{V\\* \up~=~\down}
\rulenode{DP\\*(\up\ \obj) = \down}
\rulenode{\{ DP|AP|PP \}\\*(\up\ \xcomp) = \down\\*(\down\ \subjlfg) = (\up\ \obj)\\*\hspace{-3em}@\textsc{result-t}((\up\ \textsc{pred} \textsc{fn})) }\hspace{-2ex}
}
\z
The resultative template licenses the result predicate and provides a glue semantics term that
licenses subject and object. \citet{Christie2010a} assumes the following lexical entry for
the transitive verb \emph{hammer}:
\ea
\oneline{%
\begin{tabular}[t]{@{}lcl@{}}
hammer & V & $\lambda e . hammer(e) : (\uparrow _{\sigma} $\textsc{rel}$ ) $\\
       &   & $\left( \begin{tabular}{@{~}l@{~}}
                     @\textsc{transitive}(hammer) \\
                     \small $\lambda P \lambda x \lambda y \lambda e . P(e) \wedge agent(e)=x \wedge patient(e)=y $: \normalsize \\
                     $(\uparrow _{\sigma} $\textsc{rel}$) \linimp\ (\uparrow \subj) _{\sigma}
                     \linimp\ (\uparrow \obj) _{\sigma} \linimp\ \ \uparrow _{\sigma}$
                     \end{tabular} \right) $
\end{tabular}}
\z
The resource sensitive semantics and the specification of a \predv is declared to be optional. When
these verbs are used in the c-structure rule in (\ref{ex-christie-result-c-structure}), the lexical
information is replaced by the information contributed by the resultative template. Christie assumes
that all sentences must have a specified \predv and therefore the optional \predv must be realized
in simple sentences without a result predicate.  

Christie does not explain how resultatives with intransitive verbs as in
(\ref{ex-runshoes-threadbare}), repeated here as (\mex{1}) for convenience, are analyzed but by analogy there
would be lexical items for intransitive verbs with an optional meaning contribution and a
resultative template which integrates the meaning of the result predicate with the meaning of the
intransitive verb, and which licenses an additional object argument.
\ea
\label{ex-runshoes-threadbare-two}
He runs his sneakers threadbare.
\z


The previous subsections introduced the phrasal template-based analyses of benefactive constructions
and resultative constructions by \citet{AGT2014a} and \citet{Christie2010a}. In what follows, I will
explain the problematic aspects. I start with a section that shows that neither the resultative
construction nor the benefactive construction is fixed in its form. The data challenges Toivonen's
motivations for a phrasal construction \citep{Toivonen2013a}.

\chapter{The flexibility of the constructions}
\label{sec-flexibility}

\citet{Christie2010a}, \citet{Toivonen2013a} and \citet{AGT2014a} suggest phrasal constructions
for resultative and benefactive constructions with a fixed number of daughters on the right-hand
side of the c"=structure rule. \citet{Christie2010a} proposes the following c"=structure rule for the introduction of the result
predicate and its subject:
\ea
\label{c-structure-resultative-christie}
\phraserule{V$'$}{
\rulenode{V\\* \up~=~\down}
\rulenode{DP\\*(\up\ \obj) = \down}
\rulenode{\{ DP|AP|PP \}\\*(\up\ \xcomp) = \down\\*(\down\ \subjlfg) = (\up\ \obj)\\*
             \hspace{-3em}@\textsc{result-t}((\up \textsc{pred} \textsc{fn})) }\hspace{-2ex}
}
\z
In Christie's analysis, verbs are assumed to only optionally provide semantic and f-structure
constraints. If they enter the resultative construction in (\mex{0}), the construction takes over
and provides a \textsc{pred} value and specifications for grammatical functions.

\largerpage
The rule for the benefactive construction in (\mex{1}) was provided in
(\ref{c-struc-vp-benefactive}) and is repeated here as (\mex{2}) for convenience:
\ea
The performer sang the children a song.
\z
\ea\label{c-struc-vp-benefactive-two}
\oneline{\phraserule{V$'$}{
\rulenode{V\\* \up~=~\down\\*( @\textsc{Benefactive} )}
\rulenode{DP\\*(\up\ \obj) = \down}
\rulenode{DP\\*(\up\ \objtheta) = \down}
}}
\z
According to the \citet[\page 81]{AGT2014a}, the noun phrase \emph{the children} is not an argument
of \emph{sing} but is contributed by the c"=structure rule that optionally licenses a benefactive.

As will be shown in the following, neither the resultative construction nor the benefactive
construction is fixed in this form. Let us look at resultatives first. \citet[\page 185]{CR92a} discuss
extraction data like those in (\mex{1}):
\eal\label{ex-resultative-just-verb-remains}
\ex[?]{
How shiny do you wonder which gems to polish?
}
\ex[?]{
Which colors do you wonder which shirts to dye?
}
%% \ex[?]{
%% Which sizes do you wonder which logs to cut?
%% }
\zl
These examples show that it is possible to extract both the result phrase and the object. As we see
in the examples in (\mex{1}), the objects can be extracted with the result predicate remaining in the V$'$:
\eal
\ex[]{
I wonder which gems to polish shiny.
}
\ex[]{
I wonder which shirts to dye that color.
}
\zl
It is also possible to extract the result predicate and leave the object in place:
\eal
\ex[]{
I wonder how shiny to polish the gems.
}
\ex[]{
I wonder which color to dye the shirts.
}
\zl
Apart from extraction, passivization is possible as well:
\eal
\ex[]{
The shoes were polished shiny.
}
\ex[]{
The shirts were dyed a different color.
}
\zl
This means that the object, the result predicate, or both the object and the result predicate may be missing from the
resultative construction in (\ref{c-structure-resultative-christie}). The same is true for the
benefactive construction. \citet{AGT2014a} deal with grammars of speakers of English that allow for
passivization of benefactive constructions. For those speakers all examples in (\mex{1}) are fine:
\eal
\label{ex-prepare-benefactive}
\ex Her husband prepared her divine and elaborate meals.
\ex
\label{ex-she-had-been-prepared-a-meal} 
She had been prepared divine and elaborate meals.
\ex
\label{ex-benefactive-just-verb} 
Such divine and elaborate meals, she had never been prepared before, not even
by her ex-husband who was a professional chef.
\zl
%\largerpage[3]
The examples show that some speakers permit the promotion of the benefactive to subject as in
(\mex{0}b,c) and the remaining object can be extracted as in (\mex{0}c). 
% Doug 30.11.2016:
% the eggs she was boiled never pleased her
% the stories (that) they were read

%% Note also that the benefactive is extractable contrary to Toivonen's claims:
%% \ea
%% I wonder which teacher to prepare a meal. (check)
%% \z
%% The secondary object is extractable as well:
%% \ea
%% I wonder what kind of meal to prepare the teacher. (check)
%% \z
%% This is mere speculation but maybe the unacceptability of Toivonen's example in (\ref{ex-which-teacher-toivonen}) is due to
%% higher integration costs of the fronted material. The example in (\mex{0}) was constructed so that
%% there is no intervening material between the benefactive and the VP from which it is extracted. See
%% \citew{Gibson98a} on integration costs. 

%\largerpage[2]
While the extraction of the benefactive is out as (\ref{ex-which-teacher-toivonen}), repeated here
as (\mex{1}a), shows, the examples in (\mex{1}b,c) show that the secondary object in a benefactive construction can be extracted.
\eal
\label{ex-extraction-secondary-object}
\ex[*]{
\label{ex-which-teacher-toivonen-two}
Which teacher did the kids draw a picture?
}
\ex[]{
What kind of picture did the kids draw the teacher? 
}
\ex[]{
the picture that the kids drew the teacher 
}
%What type of cake did Peter prepare the children? 
% I wonder what kind of meal to prepare the teacher. Hudson: OK
\zl
The benefactives seem to pattern with normal ditransitives here. For an overview, citing several
other sources, see \citew[\page 258]{Hudson92a-u}. Hudson reports that the extraction of the primary object of normal ditransitives is also
judged as marked or even ungrammatical by many authors and informants:
% Larson88, Ziv & Sheintuch, 1979, 
% Fodor: viele ungrammatisch, mache markiert
% Brass & Lasnik und Jackendoff finden die OK.
\eal
\judgewidth{\%}
\ex[]{
We give children sweets.
}
\ex[]{
Which sweets do you give children \_?
}
\ex[\%]{
Which children do you give \_ sweets?
}
\zl

Some variants of LFG account for extraction by assuming that the extracted element is not realized locally. The respective daughter
in a rule is optional and the place in the f"=structure is filled via functional uncertainty
(\citealp{KZ89a}; \citealt[\page 415]{Dalrymple2001a-u}; \citealp*{DKK2001a-u}; \citealt{ZK2002a}). This means that in
(\ref{ex-resultative-just-verb-remains}) and (\ref{ex-benefactive-just-verb}), we have a situation
in which it is just the verb that remains in the VP. All other elements are either promoted to
grammatical functions that are realized outside of the VP or they 
are extracted. Thus nothing is left of the original configuration, it is just the
verb. \citegen{Christie2010a} analysis of the resultative would be in deep trouble since she assumed that the
resultative template is optionally introduced at the result predicate and overwrites optional information coming
from the verb. As is clear from looking at the examples in (\ref{ex-resultative-just-verb-remains}),
attaching the constraint to the extracted result predicate would be inappropriate since the result
predicate can be fronted and would appear in another local domain (the one of \emph{wonder} rather
than \emph{dye}, compare also the discussion of (\ref{ex-dieser-frau-hat-er-behauptet})). The constraints would apply to
the wrong f"=structure. The phrasal approach could be saved by assuming traces (as
\citet[Chapter~6]{Berman2003a} does for extraction crossing clause boundaries). This would be compatible with
Christie's proposal since the structure would remain the same with some arguments being realized by empty elements.\footnote{%
    Mary Dalrymple and Miriam Butt (p.\,c.\ 2016) pointed out another solution to me: one can annotate the c-structure
    rule for the CP that combines an extracted phrase and a C$'$. Extracted phrases find the place in
    the f-structure that belongs to the place from which they are extracted by functional
    uncertainty. The resultative template could be associated with the respective place in the
    f-structure by functional uncertainty as well. However, we would then have a grammar that introduces
    resultative constructions in at least two places: SpecCP and in a special resultative V$'$. A
    generalization about English (and German) is that constituents can be extracted out of their local
    contexts and be fronted. Although technically possible, I consider it inappropriate to state at the
    SpecCP node any information about the internal structure of subconstituents from which the
    extraction took place. For certain types of resultative constructions, a resultative template in fronted position would license an additional
    object and result predicate in an embedded V$'$. Note also that authors who assume a phrasal
    resultative construction would probably also want to assume other phrasal constructions as well. If these allow extraction of crucial
    parts, the respective annotations at SpecCP would be necessary. The generalization about extraction
    would be missed. (See also the discussion of Figure~\ref{fig-case-gf-benefactive} below.)

    In addition, the lexical approach assumes one place where the resultative predicate is licensed: the
    lexical rule. The phrasal approach would assume at least two (unrelated) places. On Occamian grounds,
    the lexical analysis is to be preferred. 
}

\largerpage[2]
The situation with the benefactive construction is similar: in (\ref{ex-benefactive-just-verb}) we
have a bare verb and all other items are promoted or extracted. The template is associated with the
verb. One could either insist on the phrasal pattern in (\ref{c-struc-vp-benefactive}) and posit an
additional rule for the passive (see Chapter~\ref{sec-missing-generalization-internal}) and a trace
for extraction or assume that constituents are optional and that rules like
(\ref{c-struc-vp-benefactive}) can be used to account for all examples in
(\ref{ex-prepare-benefactive}). Under the latter proposal, the c"=structure is not really
restrictive. In the analysis of (\ref{ex-benefactive-just-verb}), only the verb is present and
one therefore could assume a lexical approach in which the benefactive template is associated with
the verb right away. (See the discussion of (\ref{ex-coordination-benefactive}), which suggests that there is an advantage for
the lexical proposal.)

\citet[\page 81]{AGT2014a} state that ``The call to \templaten{Benefactive} is optional, such
that the double-object rule is general and can also apply to non-benefactive cases.'' If
passivization and extraction are treated by declaring arguments to be optional, this also has to be
reflected in the phrase structure rule in (\ref{c-struc-vp-benefactive}). The rule has to account
for both verbs with a benefactive argument and normal ditransitive verbs.
If the rule in (\ref{c-struc-vp-benefactive}) is supposed to rule out passives like
(\ref{ex-my-sister-was-carved}), repeated here as (\mex{1}) for convenience, the benefactive NP has to be obligatory. 
\ea[*]{
\label{ex-my-sister-was-carved-two}
My sister was carved a soap statue of Bugs Bunny (by a famous sculptor).
}
\z
However, this would also rule out passives of normal ditransitives like (\mex{1}).
\ea[]{
\label{ex-my-sister-was-given}
My sister was given a soap statue of Bugs Bunny (by a famous sculptor).
}
\z
\largerpage[2]
So, if the rule were responsible for normal ditransitives as well as for benefactives, all
constraints regarding the obligatory presence of daughters would have to reside in the template
since this is the only part that is different between benefactives and normal ditransitives. The
templates defined by \citet{AGT2014a} contain semantic constraints and constraints relevant for
argument structure mappings. Nothing syntactic is encoded there. So, either the authors assume that
benefactives pattern like normal ditransitives syntactically in the speaker group that they
examine and then there would be no need to introduce the benefactive argument phrasally or there is
a difference and then a special benefactive c"=structure rule should be assumed that is incompatible
with normal ditransitive verbs.\footnote{%
 An alternative may be to say that the V$'$ rule with two objects is for benefactives and for
 ditransitive verbs with all objects realized in the VP. One would then assume that the passive of
 ditransitives is taken care of by the phrase structure rule in (i).
\ea\label{c-struc-vp-ditransitive-passive}
\oneline{\phraserule{V$'$}{
\rulenode{V\\* \up~=~\down}
\rulenode{DP\\*(\up\ \objtheta) = \down}
}}
\z
Since the benefactive template is not mentioned in this rule, no benefactive argument would be
licensed in the respective configuration. \citet*{BATW2015a} state that \objtheta is the grammatical
function for secondary objects. Hence, a rule like (i) is a special rule with a missing primary
object. The only purpose (i) would serve in a grammar of English would be to account for the passive
and primary object extraction of ditransitive verbs. The rule would be a stipulation and a
generalization about the passivizability of ditransitive verbs would be missed.
}



%% \citet[\page 81]{AGT2014a} state that \emph{The call to \templaten{Benefactive} is optional, such
%%  that the double-object rule is general and can also apply to non-benefactive cases.} If
%%  passivization and extraction are treated by declaring arguments to be optional the phrase structure
%%  rule in (\ref{c-struc-vp-benefactive}) has to be formulated to account for normal ditransitive
%%  verbs. If the rule in (\ref{c-struc-vp-benefactive}) is supposed to rule out passives like
%%  (\ref{ex-my-sister-was-carved}) the benefactive NP has to be obligatory.
%
% Für Passiv der ditransitiven Verben kann man sage, dass eine V' -> V, NP Regel angewednet wird 
%% \eal
%% \ex If a cookie was stored on a web site, then you would have to identify yourself somehow so the
%% site would know which cookie to give you.
%% \ex 

%% Constraints on extraction and passivization would have to be part of the template since this
%% is the only part in the rule that is specific to benefactives. In order to capture Toivonen's
%% constraints on fixedness of the structure, a special c"=structure for benefactive constructions
%% would be necessary.

Before I turn to the analysis of active/passive alternations, I want to take a look at the
interaction of morphology and the constructions under consideration.

\chapter{Morphology and valence}%, and Lexical Integrity}
\label{sec-morphology}
\label{sec-resultative-adjectival-participles}


%\section{Morphology and Valence}

Morphological processes have to be able to see the valence of the element they apply to (this point
was also made by \citew[Section~4.2]{MWArgSt} in connection with the GPSG representation of
valence). For instance, the generalization about productive \suffix{bar} `able' derivation in German is that it only applies to
verbs that govern a subject and an accusative object. While \emph{lösbar} `solvable' and
\emph{ver\-gleich\-bar} `comparable' can be formed, \noword{schlafbar} `sleepable' and \noword{helfbar} `helpable' are ruled out:
\eal
\settowidth\jamwidth{(NP[nom], NP[acc], PP[mit])}
\ex[]{
\gll lösbar\\
     solveable\\ \jambox{(NP[nom], NP[acc])}
}
\ex[]{
\gll vergleichbar\\
     comparable\\ \jambox{(NP[nom], NP[acc], PP[mit])}
}
\ex[*]{
\gll schlafbar\\
     sleepable\\ \jambox{(NP[nom])}
}
\ex[*]{
\gll helfbar\\
     helpable\\\jambox{(NP[nom], NP[dat])}
}
\zl
The resultative construction also interacts with \bard: the adjectives \emph{leerfischbar}
`empty.fish.able' = `It is possible to fish X empty.' and \emph{platt\-fahr\-bar}\footnote{%
  \url{http://www.forum-3dcenter.org/vbulletin/archive/index.php/t-236032.htmls}, 2016-06-02.
} `flat.drive.able' can be formed. If arguments are introduced by phrasal configurations which refer to
fully derived and inflected words or phrases consisting of words, the
accessibility of the valence information to the morphology component is not given and it remains an
open question how phrasal analyses can explain the contrasts in (\mex{0}) and the fact that
\bard does apply to verbs in the resultative construction. In \citegen{Christie2010a} analysis, the intransitive
verbs would probably be represented as intransitive in the lexicon with an optional semantic
representation. As was argued in \citew{Mueller2003b}, the derivational affix attaches to the verbal
stem and hence the information about an accusative object would not be available in Christie's
approach. Even if one assumes that \emph{leer} and \emph{fisch} are combined before the attachment
of \suffix{bar}, it is unclear what licenses this combination. The fact that \emph{leerfisch} takes
an accusative object would have to be available at the point when \suffix{bar} attaches and could
not be contributed by phrase structure rules in the syntax. Reviewers suggested that special rules
could combine adjectives and verbs in the morphology component and license the object there. While
this is possible in principle, this would be an instance of a missed generalization since one would
have to assume two unrelated rules that mention the resultative template.
%\todostefan{Timm: Warum "`unrelated"'? Sie haben doch wahrscheinlich einen ähnlichen Zweck und eine ähnliche Form?}

Furthermore, there are resultative constructions with phrasal result phrases like (\mex{1}) and here
it could not be argued that PP and verbal stem form a new verbal stem to which \suffix{bar}
attaches.
%% \eal
%% %Dachschiefer muss besonders gleichmäßig sein und ist nur in bergfrischem Zustand in dünne Scheiben
%% %spaltbar.\footnotemark\\
%% \ex
%% \gll Dachschiefer [\ldots] ist nur  in bergfrischem Zustand in dünne Scheiben spaltbar.\footnotemark\\
%%      roof.slate   {}       is  only in mountain.fresh state in thin slices    splitable\\
%% \glt `Roof slate can only be split into thin slices when it comes fresh from the mountain.'
%% \footnotetext{
%% \url{http://www.schnitzius-weine.de/schieferpfad/Naturbaustoff.htm}, 26.06.2016.
%% } 
%% %% \ex Schneideinrichtung, mit welcher der Rohmaterialstrang in einzelne Rohlinge zerschneidbar ist\\
%% %% \url{https://www.google.com/patents/DE102009017632A1?cl=de}. 26.06.2016.
%% %% \ex eine Brennlanze (20), mittels der die großvolumigen Bären in transportable und handhabbare Teile
%% %% zerschneidbar sind
%% %% http://www.google.de/patents/EP0504664A1?cl=de
%% \ex 
%% \gll Stickerei der höchsten Qualität, in verschiedene Muster zerschneidbar für unendliche Verwendungsmöglichkeiten.\footnotemark\\
%%      embroidery of.the highest quality in different patterns cutable       for infinite  uses\\
%% \footnotetext{
%% \url{https://www.ernessa.com/catalog/product_info.php?products_id=143}, 26.06.2016.
%% }
%% \glt `Embroidery of the highest quality that can be cut into different patterns for infinitely many uses.'
%% \zl
\ea
\gll Die Mauer ist in kleine Stücke fahrbar.\\
     the wall  is into small pieces driveable\\
\glt `The wall can be driven into small pieces.'
\z
If one is ready to follow \citet{Bruening2018a} and get rid of the assumption of Lexical Integrity,
then (\mex{0}) can be analyzed as a combination of \emph{in kleine Stücke fahr-} and \suffix{bar},
but if one wants to maintain the view that words are the atoms of syntax \citep[\page 10]{ADT2013a}, the \bard constitutes
evidence against Alsina's and Chris\-tie's phrasal approach (as a general approach that also holds
for German). 

Another example of derivational morphology showing that information about valence is needed at the
word level is the derivation of adjectival participles: this derivation is only possible if the verb
requires an accusative object. So, when the adjectival participle is derived, this information has
to be accessible.
%% \footnote{%
%%   This only follows if one believes that affixes do not combine with phrases but with stems, an
%%   assumption commonly made in LFG (\citealp{BM95a}, see also footnote~\ref{fn-lexical-integrity}).
%% }
\citet{Alsina96a} showed that the passive of resultatives can be accounted for lexically since the
subject that has to be suppressed is available in the stem. It can be suppressed when the participle
is formed and when further arguments are added in the syntax, these have to be realized as
subjects. However, this fails in the case of adjectival participles. The adjective derivation
applies to a passivized verbal stem that has at least one argument slot open: the accusative object
in the active. (\mex{1}) shows an example:
\eal
\ex[]{
\gll Er        tanzt  die        Schuhe blutig / in Stücke.\\
     he.\nom{} dances the.\acc{} shoes  bloody {} into pieces\\
}
\ex[]{
\gll die in Stücke / blutig getanzten Schuhe\\
     the into pieces {} bloody danced shoes\\
}
\ex[*]{
\gll die getanzten Schuhe\\
     the danced    shoes\\
}
\zl
The shoes are not a semantic argument of \emph{tanzt}. Nevertheless, the referent of the NP that is realized as
accusative NP in (\mex{0}a) is the element the adjectival participle in (\mex{0}b) predicates
over. Adjectival participles like the one in (\mex{0}b) are derived from a passive participle of a
verb that governs an accusative object. The example in (\mex{0}c) shows that the passive participle cannot
be formed with unergative intransitive verbs. This should be contrasted with a transitive verb like
\emph{lieben} `to love':
\ea
\gll der geliebte Mann\\
     the loved    man\\
\glt `the beloved man'
\z
The transitive verb allows the formation of the adjectival participle and the participle with
resultative predicate in (\mex{-1}b) behaves completely parallel.

If the accusative object in resultative constructions is licensed phrasally by
configurations like the one in (\ref{ex-christie-result-c-structure}), it cannot be explained why the participle \emph{getanzte}
can be formed despite the absence of an accusative object in the valence specification of the
verb. See the next section for further interactions of resultatives and morphology.
%\citew[Section~5]{Mueller2006d} for
%further examples of the interaction of resultatives and morphology.
% Other valence-dependent derivations are the \bard (\suffix{able}). Resultatives appear in
% German \bards: \emph{leerfischbar} `empty.fishable' and \emph{Leerfischbarkeit}
% `empty.fishability'. The object of \emph{leer fischen} `to fish empty' is not the object of
% \emph{fischen} and hence it cannot be explained why \emph{fischbar}
The conclusion, which was drawn in the late 70s and early 80s by \citet[\page 412]{Dowty78a}
and \citet[\page 21]{Bresnan82a}, is that phenomena that feed morphology should be treated
lexically. The natural analysis in frameworks like HPSG, CG, CxG, and LFG is therefore a lexical
one, for example one that assumes a lexical rule for the licensing of resultative constructions. See
\citew{Verspoor97a}, \citew{Wechsler97a}, \citew{WN2001a}, Wunderlich (\citeyear[\page
  45]{Wunderlich92a-u-kopiert}; \citeyear[\page 120--126]{Wunderlich97c}), \citew{KW98a},
 \citew[Chapter~5]{Mueller2002b}, 
%\citew{Kay2005a}, das ist Dativ
\citew{Simpson83a} and \citew{Christie2015a-u} for lexical proposals in some of  these frameworks. 


%\section{Lexical Integrity in recent LFG publications}

%% \chapter{Lexical Integrity in recent LFG publications}
%% \label{sec-lexical-integrity}

%% Asudeh, Dalrymple \& Toivonen's papers are about the concept of lexical integrity and about
%% constructions.\footnote{\label{fn-lexical-integrity}
%%   \citet[\page 92]{BATW2015a} define lexical integrity as follows: ``Morphologically complete words are leaves of the c-structure tree, and each leaf corresponds to one and only one c"=structure node.''
%% } \citet{AT2014a} replied to the target article by \citet{MWArgSt} and pointed out that their
%% template approach makes it possible to specify the functional structure of words and phrases
%% alike. In the original paper they discussed the Swedish word \emph{vägen}, which is the definite
%% form of \emph{väg} `way'. They showed that the f"=structure is parallel to the f"=structure for the
%% English phrase \emph{the way}. I think the reply by Asudeh and Toivonen missed the point of the
%% criticism. Müller \& Wechsler did not criticize the template-based approach as such, they just
%% pointed out that a complete theory of natural language(s) has to deal with morphology and that it
%% has to explain how morphological phenomena that refer to valence information can be handled. It is
%% not sufficient to be able to provide the f"=structure of words, the question is how this f-structure
%% can be systematically related to the parts of the words in a morphological analysis. More generally
%% speaking, one wants to derive all properties of the involved words, that is, their valence, their
%% meaning, and the linking of this meaning to their dependents. What \submitOrNormal{Müller \&
%%   Wechsler used in their}{we used in our}  argument was
%% parallel to what Bresnan used in her classical argument for a lexical treatment of passive. So
%% either Bresnan's argument (and Müller \& Wechsler's) is invalid or both arguments are valid and there is a problem
%% for phrasal template-based approaches to argument structure constructions and for phrasal approaches
%% to argument structure constructions that assume lexical integrity in general. I want to
%% give another example that was already discussed in \citew[\page 869]{Mueller2006d}\reallyomitted{ but was omitted in
%% \citew{MWArgSt} due to space limitations}. I will first point out why this example is problematic for
%% phrasal approaches and then explain why it is not sufficient to be able to assign certain
%% f"=structures to words: In (\mex{1}a), we are dealing with a resultative construction.
%% According to the common phrasal approach, which \submitOrNormal{Müller
%%   \& Wechsler}{we} termed the \emph{plugging approach}, the resultative
%% meaning is contributed by a phrasal construction into which the verb \emph{fischt} is
%% inserted (\citealt{Goldberg95a,GJ2004a}; \citealt{Christie2010a}). There is no lexical item that requires a resultative predicate as 
%% its argument. If no such lexical item exists, then it is unclear how the relation between (\mex{1}a)
%% and (\mex{1}b) can be established: 

%% \eal
%% \ex 
%% \gll {}[dass] jemand die Nordsee leer fischt\\
%% 	 {}\spacebr{}that somebody.\nom{} the.\acc{} North.Sea empty fishes\\
%% \ex\label{bsp-leerfischung}
%% \gll wegen   der \emph{Leerfischung}  der    Nordsee\footnotemark\\
%%      because of.the empty.fishing of.the North.Sea\\
%% \footnotetext{
%%         Example from the national newspaper taz, 20.06.1996, p.\,6.%
%% }
%% \glt `because of the fishing that resulted in the North Sea being empty'
%% \zl
%% As Figure~\vref{fig-rc-nom} shows, both the arguments selected by the heads and the structures are completely different.
%% In (\mex{0}b), the element that is the subject of the related construction in (\mex{0}a) is not realized. As is normally the case in nominalizations,
%% it is possible to realize it in a PP with the preposition \emph{durch} `by':
%% \ea
%% \gll wegen der Leerfischung der Nordsee durch die Anrainerstaaten\\
%%      because of.the.\gen{} empty.fishing of.the.\gen{} North.Sea by the neighboring.countries\\
%% \glt `because of the fishing by the neighboring states that resulted in the North Sea being empty'
%% \z
%% %
%% \begin{figure}
%% %\hfill
%% \resizebox{.49\textwidth}{!}{
%% \begin{forest}
%% sm edges
%% [S
%% 	[NP{[\textit{nom}]}
%% 		[jemand;somebody]]
%% 	[NP{[\textit{acc}]}
%% 		[die Nordsee;the North.Sea, roof]]
%% 	[Adj
%% 		[leer;empty]]
%% 	[V
%% 		[fischt;fishes]]]
%%   \end{forest}
%%   }
%% \hfill
%% \resizebox{.49\textwidth}{!}{
%% \begin{forest}
%% sm edges
%% [NP
%% 	[Det
%% 		[die;the]]
%% 	[N$'$
%% 		[N
%% 			[Leerfischung;empty.fishing]]
%% 		[NP{[\textit{gen}]}
%% 			[der Nordsee;of.the North.Sea, roof]]]]
%% \end{forest}
%% }
%% %\hfill\mbox{}
%% \caption{Resultative construction and nominalization}\label{fig-rc-nom}
%% \end{figure}%
%% %
%% If one assumes that the resultative meaning comes from a particular configuration in which a verb
%% is realized, there would be no explanation for (\mex{-1}b) since no verb is involved in the analysis
%% of this example. One could of course assume that a verb stem is inserted into a construction both in
%% (\mex{-1}a) and (\mex{-1}b). The inflectional morpheme \suffix{t} and the derivational
%% morpheme \suffix{ung} as well as an empty nominal inflectional morpheme would then be independent syntactic
%% components of the analysis. However, since \citet[\page 119]{Goldberg2003a} and \citet{ADT2013a} and
%% \citet{AT2014a} assume lexical integrity, only entire words can be inserted into syntactic constructions and hence
%% the analysis of the nominalization of resultative constructions sketched here is not an option for them.

%% It would also be possible to assume that both constructions  in (\mex{1}), for which structures such as those in
%% Figure~\ref{fig-rc-nom} would have to be assumed, are connected via metarules.\footnote{%
%%   LFG does not assume transformations or metarules, but such an extension would be necessary if one
%%   insisted on the phrasal analysis.
%%   For instance, Goldberg (p.\,c.\ 2007, 2009) suggests connecting certain constructions using GPSG"=like metarules.
%% Note also that GPSG metarules relate phrase
%% structure rules, that is, local trees. The structure in
%% Figure~\ref{fig2-rc-nom-construction}, however, is highly complex. The question of relating certain
%% configurations will be discussed again in Chapter~\ref{sec-missing-generalization-internal}, which
%% deals with missing language-internal generalizations and transformations in the sense of
%% transformational grammar.%
%% }$^,$\footnote{%
%%   The structure in (\mex{1}b) violates a strict interpretation of lexical integrity as is commonly assumed in
%%   LFG. Under the LFG view it is not allowed to access the internal structure of words. \citet{Booij2005a,Booij2009a}, working in Construction Grammar, subscribes to a somewhat
%%   weaker version, however. This weaker view could also be adopted in LFG.%
%% }
%% \eal
%% \ex {}[ Sbj Obj Obl V ]
%% \ex {}[ Det [ [ [ Adj V ] -ung ] ] NP[\type{gen}] ]
%% \zl
%% The construction in (\mex{0}b) corresponds to
%% Figure~\vref{fig2-rc-nom-construction}.\footnote{%
%%   I do not assume zero affixes for inflection. The respective affix in
%%   Figure~\ref{fig2-rc-nom-construction} is there to show that
%%   there is structure. Alternatively one could assume a unary branching rule/construction as is
%%   common in HPSG/Construction Morphology (Riehemann \citeyear{Riehemann93a,Riehemann98a}; \citealt{Booij2010a}).
%% }
%% \begin{figure}
%% \centering
%% \begin{forest}
%% %sm edges
%% for tree={fit=rectangle}
%% [NP
%% 	[Det]
%% 	[N$'$
%% 		[N 
%%                    [N-Stem
%% 			[V-Stem 
%%                           [Adj]
%% 			  [V-Stem] ]
%% 			[N-Aff [-ung] ]]
%%                    [N-Inflection [\trace] ]]
%% 		[{NP[\textit{gen}]}] ] ]
%% \end{forest}
%% \caption{Resultative construction and nominalization}\label{fig2-rc-nom-construction}
%% \end{figure}%
%% The genitive NP is an argument of the adjective. It has to be linked semantically to the subject slot of the adjective.
%% Alternatively, one could assume that the construction only has the form [[Adj V] \suffix{ung}], that
%% is, that it does not include the genitive NP. But then, one could also assume that the verbal variant
%% of the resultative construction has the form [OBL V] and that Sbj and Obj are only represented in
%% valence lists. This would almost be a lexical analysis, however.\todostefan{fails for PP V ung}

%% Turning to lexical integrity again, I want to point out that there seem to be just two options for template-based
%% approaches to resultatives to deal with nominalizations like the one in
%% Figure~\ref{fig2-rc-nom-construction}. Either they go for a lexical approach and assign the
%% template to the combination of adjective and verb and hence admit that the genitive argument is
%% licensed lexically and not by a phrasal construction or they assume that the template is attached to
%% the N in the rule that combines N and NP[\textit{gen}]. The template would say that the N may contribute a
%% resultative meaning and an appropriately linked genitive argument. What is needed, however,
%% is a principled account of how the f"=structure of the N comes about and how it is related to the
%% resultative construction at the sentence level. One could of course claim that the
%% \templaten{resultative} template applies to N and since the \suffix{ung} affix is resource sensitive
%% it consumes certain semantic resources and adds others (the trick that was used in the
%% \templaten{Benefactive} template). Such templates that remap predicates to take different resources
%% are equivalent to postponed lexical rules. Note though that this analysis would require the
%% stipulation of an optional \templaten{resultative} template at the N node, basically saying: there
%% may be a resultative construction deeply embedded somewhere inside of the N. Remember that
%% \emph{Leerfischbarkeit} `empty.fish.able.ity' with embedding of the resultative construction under a
%% modal affix is possible. The point is that nothing is known about
%% the internal structure of the N if we assume lexical integrity and speculating about possible
%% resultative constructions inside the N is mere guesswork. Since resultative constructions are not
%% the only argument structure sensitive constructions that interact with nominalization, one would
%% have to specify all possible interactions as a big disjunction at the N node. A rather unattractive
%% consequence. Note further that the \templaten{Benefactive} template refers to the $\sigma$ structure
%% of its mother node. For \emph{Leerfischbarkeit} this would be a nominal structure containing the
%% modal operator and not the \argtwo and \argthree that the template refers to. The general problem is
%% that embedding is needed to account for morphological derivations and that inheritance fails to
%% capture this \citep{KN93a}. The inherited information may be relevant at different levels of embedding (\eg no
%% embedding in the verb phrase where the template refers to the $\sigma$ structure of the mother
%% directly and some embedding in the derived noun). The general problem was also discussed by
%% Müller (\citeyear[Section~5.3]{Mueller2006d}; \citeyear{MuellerPersian}) in more detail. See also
%% \citew{MuellerDefaults}.

%% In lexical approaches a verbal stem for \stem{fisch}
%% selecting for a subject argument would be related to a verbal stem \stem{fisch} that selects for a
%% subject, an object and a result predicate. This stem can be nominalized by normal nominalization
%% rules. The nominalization rules take care of the realization of the verbal arguments as genitive
%% NPs. This is the normal realization of arguments in nominalizations and it is completely independent
%% of the resultative construction.

I now turn to active/passive alternations and point out that the phrasal approach is missing generalizations.

\chapter{Missing generalizations: Active/passive alternations}
\label{sec-missing-generalization-internal}
\label{sec-active-passive}


%% They could assume allo-constructions and make both c"=structures inherit from the same super construction.
%%
In this section, I want to show that \citeauthor*{AGT2014a}'s \citeyearpar{AGT2014a} approach to the phrasal introduction
of benefactives either does not need to be stated at the phrasal level since the phrasal
construction does not contribute relevant information or that the approach misses generalizations
regarding the configurations for active and passive.

(\mex{1}), taken from \citet[\page 72]{AGT2014a}, provides examples of the benefactive construction
in an active and a passive variant:
\eal
\ex The performer sang the children a song.
\ex The children were sung a song. 
\zl
According to the authors, the noun phrase \emph{the children} is not an argument of \emph{sing} but
contributed by the c"=structure rule in (\ref{c-struc-vp-benefactive}), which optionally licenses a
benefactive. The rule is repeated here as (\mex{1}) for convenience:
\ea\label{c-struc-vp-benefactive-three}
\phraserule{V$'$}{
\rulenode{V\\* \up~=~\down\\*( @\textsc{Benefactive} )}
\rulenode{DP\\*(\up\ \obj) = \down}
\rulenode{DP\\*(\up\ \objtheta) = \down}\hspace{-2ex}
}
\z
Whenever this rule is called, the template \textsc{Benefactive} can add a benefactive role and the
respective semantics, provided this is compatible with the verb that is inserted into the structure. The
authors show how the mappings for the passive example in (\mex{-1}b) work, but they do not provide
the c"=structure rule that licenses such examples. Unless one assumes that arguments in
(\ref{c-struc-vp-benefactive-three}) can be optional (see below), one would need a c"=structure rule for
passive VPs and this rule has to license a benefactive as well.\footnote{%
  See, for instance, \citew{BC2005a} and \citew{vanTrijp2011a} for Construction Grammar analyses that assume active
  and passive variants of phrasal constructions. See \citew{Cappelle2006a} on allostructions in general.
  } So it would be:
\ea\label{c-struc-vp-benefactive-passive}
\phraserule{V$'$}{
\rulenode{V[pass]\\* \up~=~\down\\*( @\textsc{Benefactive} )}
\rulenode{DP\\*(\up\ \objtheta) = \down}
}
\z
Note that a benefactive cannot be added to just any verb: adding a benefactive to an intransitive verb as
in (\mex{1}a) is out and the passive that would correspond to (\mex{1}a) is ungrammatical as well,
as (\mex{1}b) shows:
\eal
\ex[*]{
He laughed the children.
}
\ex[*]{
The children were laughed.
}
\zl
The benefactive template would account for the ungrammaticality of (\mex{0}) since it requires an
\argtwo to be present, but it would admit the sentences in (\mex{1}b,c) since \emph{give} with
prepositional object has an \argtwo \citep[\page 317]{Kibort2008a}.
\eal
\ex[]{
He gave it to Mary.
}
\ex[*]{
He gave Peter it to Mary.
}
\ex[*]{
Peter was given it to Mary.
}
\zl
\emph{give} could combine with the \emph{to} PP semantically and would then be equivalent to a
transitive verb as far as resources are concerned (looking for an \argone and an \argtwo). The
benefactive template would map the \argtwo to \argthree and hence (\mex{0}b) would be
licensed. Similar examples can be constructed with other verbs that take prepositional objects, for
instance \emph{accuse sb.\ of something}.
Since there are verbs that take a benefactive and a PP object as shown by (\mex{1}), (\mex{0}b) cannot be ruled out
with reference to non-existing c"=structure rules.
\ea
I buy him a coat for hundred dollar.
\z

So, if the c-structure is to play a role in argument structure constructions at all, one could not
just claim that all c"=structure rules optionally introduce a benefactive argument. Therefore there is something special about the two rules in (\ref{c-struc-vp-benefactive-two})
and (\ref{c-struc-vp-benefactive-passive}). The problem is that there is no relation between these
rules. They are independent statements saying that there can be a benefactive in the active and that
there can be one in the passive. This is what \citet[\page 43]{Chomsky57a} criticized in 1957 with
respect to simple phrase structure grammar and
this was the reason for the introduction of transformations. Bresnan"=style LFG captured the
generalizations by lexical rules \citep{Bresnan78a,Bresnan82a} and later by lexical rules in combination with Lexical Mapping
Theory \citep{Toivonen2013a}. But if elements are added
outside the lexical representations, the representations where these elements are added 
have to be related too. One could say that our knowledge about formal tools has changed since
1957. We now can use inheritance hierarchies to capture generalizations. So one can assume a type
(or a template) that is the supertype of all those c"=structure rules that introduce a
benefactive. But since not all rules allow for the introduction of a benefactive element, this
basically amounts to saying: c"=structure rule A, B, and C allow for the introduction of a
benefactive. In comparison, lexical rule"=based approaches have one statement introducing the
benefactive. The lexical rule states what verbs are appropriate for adding a benefactive and
syntactic rules are not affected.

Asudeh (p.\,c.\ May 2016) and an anonymous reviewer of Head\-Lex16 pointed out to me that the rules in
(\ref{c-struc-vp-benefactive-three}) and (\ref{c-struc-vp-benefactive-passive}) can be generalized over if
the arguments in (\ref{c-struc-vp-benefactive-three}) are made optional. (\mex{1}) shows the rule in
(\ref{c-struc-vp-benefactive-three}) with the DPs marked as optional by the brackets enclosing them.
\ea\label{c-struc-vp-benefactive-optional-args}
\phraserule{V$'$}{
\rulenode{V\\* \up~=~\down\\*( @\textsc{Benefactive} )}
\rulenode{(DP)\\*(\up\ \obj) = \down}
\rulenode{(DP)\\*(\up\ \objtheta) = \down}\hspace{-2ex}
}
\z
Since both of the DPs are optional, (\mex{0}) is equivalent to a specification of four rules, namely
(\ref{c-struc-vp-benefactive-three}) and the three versions of the rule in (\mex{1}):
\eal
\ex\label{c-struc-vp-benefactive-optional-args-otheta}
\phraserule{V$'$}{
\rulenode{V\\* \up~=~\down\\*( @\textsc{Benefactive} )}
%\rulenode{(DP)\\*(\up\ \obj) = \down}
\rulenode{DP\\*(\up\ \objtheta) = \down}\hspace{-2ex}
}
\ex\label{c-struc-vp-benefactive-optional-args-obj}
\phraserule{V$'$}{
\rulenode{V\\* \up~=~\down\\*( @\textsc{Benefactive} )}
\rulenode{DP\\*(\up\ \obj) = \down}
%\rulenode{(DP)\\*(\up\ \objtheta) = \down}\hspace{-2ex}
}
\ex\label{c-struc-vp-benefactive-optional-args-none}
\phraserule{V$'$}{
\rulenode{V\\* \up~=~\down\\*( @\textsc{Benefactive} )}
%\rulenode{(DP)\\*(\up\ \obj) = \down}
%\rulenode{(DP)\\*(\up\ \objtheta) = \down}\hspace{-2ex}
}
\zl
(\mex{0}a) is the variant of (\ref{c-struc-vp-benefactive-optional-args}) in which the \obj is
omitted (needed for (\ref{ex-she-had-been-prepared-a-meal})), (\mex{0}b) is the variant in which the
\objtheta is omitted (needed for (\ref{ex-extraction-secondary-object})) and in (\mex{0}c) both DPs are
omitted (needed for (\ref{ex-benefactive-just-verb})). Hence,
(\ref{c-struc-vp-benefactive-optional-args}) can be used for V$'$s containing two objects, for V$'$s
in the passive containing just one object, for V$'$ with the secondary object extracted and for V$'$
in the passive with the secondary object extracted. The template-based approach does not
overgenerate since the benefactive template is specified such that it requires the verb it applies
to to select for an \argtwo. Since intransitives like \emph{laugh} do not select an \argtwo, a
benefactive cannot be added. So, in fact, the actual configuration in the c-structure rule does only
play a minor role: the account mainly relies on semantics and resource sensitivity. There is one
piece of information that is contributed by the c-structure rule: it constrains the grammatical
functions of \argtwo and \argthree, which are underspecified in the template definitions for \argtwo
and \argthree (see the discussion on page~\pageref{page-disjunctions-gf-c-structure}). \argtwo can
be realized as \subjlfg or as \obj. In the active case, \argone will be the \subjlfg and, because of function argument bi-uniqueness \citep[\page 334]{BATW2015a}, no other element can be the \subjlfg and hence \argtwo has to
be an \obj. \argthree can be either an \obj or an \objtheta. Since \argtwo is an \obj in the active,
\argthree has to be an \objtheta in the active. In the passive case, \argone is suppressed or
realized as \obltheta (\emph{by} PP). \argtwo will be realized as \subjlfg (since English requires a \subjlfg to be realized) and \argthree could be realized as either
\obj or \objtheta. This is not constrained by the template specifications so far. Because of the
optionality in (\ref{c-struc-vp-benefactive-optional-args}), either the \obj or the \objtheta
function could be chosen for \argthree. This means that either Lexical Mapping Theory has to be
revised or one has to make sure that the c-structure rule used in the passive of benefactives states the grammatical
function of the object correctly. Hence one would need the c-structure rule in
(\ref{c-struc-vp-benefactive-passive}) and then there would be the missing generalization I pointed
out above.

If one finds a way to set up the mappings to grammatical functions without reference to c-structures
in lexical templates, this means that it is not the case that an argument is added by
a certain configuration the verb enters in. Since any verb may enter (\mex{0}) and since the only
important thing is the interaction between the lexical specification of the verb and the benefactive
template, the same structures would be licensed if the benefactive template were added to the
lexical items of verbs directly. The actual configuration would not constrain anything. All (alleged) arguments
from language acquisition and psycholinguistics (for an overview of such arguments see \citew{MWArgSt,MWArgStReply}) for phrasal analyses would not apply to such a phrasal account.

If the actual c-structure configuration does not contribute any restrictions as to what arguments
may be realized and what grammatical functions they get, the difference between the lexical use of the
benefactive template and the phrasal introduction as executed in
(\ref{c-struc-vp-benefactive-optional-args}) is really minimal. However, there is one area in
grammar where there is a difference: coordination. As \citet[Section~6.1]{MWArgSt} pointed out, it is
possible to coordinate ditransitive verbs with verbs that appear together with a
benefactive. (\mex{1}) is one of their examples:
\ea%l
\label{ex-coordination-benefactive}
%\ex 
\label{ex-offered-and-made}
She then offered and made me a wonderful espresso -- nice.\footnote{%
  \url{http://www.thespinroom.com.au/?p=102}, 2012-07-07.}
%% \ex My sisters just baked and gave me a nutella cupcake with mint chocolate chip ice-cream in the middle and milk chocolate frosting on
%% top.\footnote{%
%% \url{http://bambambambii.tumblr.com/post/809470379}. 05.06.2012.
%% }
\z
If the benefactive information is introduced at the lexical level, the coordinated verbs basically
have the same selectional requirements. If the benefactive information is introduced at the phrasal
level, \emph{baked} and \emph{gave} are coordinated and then the benefactive constraints are imposed
on the result of the coordination by the c-structure rule. While it is clear that the lexical items
that would be assumed in a lexical approach can be coordinated in a symmetric coordination, problems
seem to arise for the phrasal approach. It is unclear how the asymmetric coordination of the mono-
and ditransitive verbs can be accounted for and how the constraints of the benefactive template are
distributed over the two conjuncts. The fact that the benefactive template is optional does not help
here since the optionality means that the template is either called or it is not. The situation is
depicted in Figure~\vref{fig-optionality-coordination}.
\begin{figure}
\scalebox{.8}{%
\begin{forest}
sm edges
%[IP
%  [NP [she]]
%  [I$'$
    [VP
      [\begin{tabular}{@{}c@{}}
       V\\(\template{Benefactive})
       \end{tabular}
        [V [offered]]
        [Conj [and]]
        [V [made]]]
      [NP [me]]
      [NP [an espresso,roof]]]
    %]]
\end{forest}} \raisebox{2cm}{\scalebox{.8}{$\equiv$}} \\

\hfill\scalebox{.8}{%
\begin{forest}
sm edges
%[IP
%  [NP [she]]
%  [I$'$
    [VP
      [\begin{tabular}{@{}c@{}}
       V\\(\template{Benefactive})
       \end{tabular}
        [V [offered]]
        [Conj [and]]
        [V [made]]]
      [NP [me]]
      [NP [an espresso,roof]]]
    %]]
\end{forest}}\hfill \raisebox{2cm}{\scalebox{.8}{$\vee$}} \hfill
\scalebox{.8}{%
\begin{forest}
sm edges
%[IP
%  [NP [she]]
%  [I$'$
    [VP
      [\begin{tabular}{@{}c@{}}
       V%\\\template{Benefactive}
       \end{tabular}
        [V [offered]]
        [Conj [and]]
        [V [made]]]
      [NP [me]]
      [NP [an espresso,roof]]]
    %]]
\end{forest}}\hfill\mbox{}
\caption{The optionality of a call of a template corresponds to a disjunction.}\label{fig-optionality-coordination}
\end{figure}
The optionality of the template call in the top figure basically corresponds to the disjunction of
the two trees in the lower part of the figure. The optionality does not allow for a distribution to one of the daughters in a coordination.

Mary Dalrymple (p.\,c.\ 2016) pointed out that the coordination rule that coordinates two verbs can
be annotated with two optional calls of the benefactive template.
\ea
\phraserule{V}{
\rulenode{V\\*( @\textsc{Benefactive} )}
\rulenode{Conj}
\rulenode{V\\*( @\textsc{Benefactive} )}\hspace{-2ex}
}
\z
In an analysis of the examples in (\ref{ex-offered-and-made}), the template in rule
(\ref{c-struc-vp-benefactive-three}) would not be called but the respective templates in (\mex{0}) would
be called instead. While this does work technically, similar coordination rules would be needed for all other
constructions that introduce arguments in c"=structures. Furthermore, the benefactive would have to
be introduced in several unrelated places in the grammar and finally the benefactive is introduced
at nodes consisting of a single verb without any additional arguments being licensed, which means
that one could have gone for the lexical approach right away. Timm Lichte (p.\,c.\ 2016) pointed out
an important consequence of a treatment of coordination via (\mex{0}): since the result of the
coordination behaves like a normal ditransitive verb it would enter the normal ditransitive
construction and hence it would be predicted that none of the constraints on passive and extraction
that are formulated at the phrasal level would hold if an item is coordinated with either another
benefactive verb or a normal ditransitive verb like \emph{give}. This is contrary to the facts: by
coordinating items with strong restrictions with items with weaker restrictions, one gets a
coordination structure that is at least as restrictive as the items that are coordinated. One does
not get less restrictive by coordinating items.

The next section deals with German and explains in detail why cross"=linguistic generalizations are not
captured in the phrasal approach, but I want to mention two phenomena here since they are relevant
to the point of missing language internal generalizations. As was shown in \citew[Section~5]{Mueller2006d}, there is interaction between the resultative construction and nominalizations, which
cannot be captured by inheritance. Similarly there are prenominal adjectival phrases in German that
include resultatives and/or benefactives (Section~\ref{sec-resultative-adjectival-participles},
Section~\ref{sec-adjectival-participles-benefactive}). For these phenomena, the interaction of the
respective constructions follows immediately from a lexical approach while the interaction has to be stated on a case by case basis
on the template-based phrasal approach. So, while the passive example above may be dealt with by
underspecification, \eg optionality of arguments, this is not possible for the nominalization
structures since the syntax of NPs is really different from the syntax of verb phrases. 
\eal
\ex 
\gll {}[dass] jemand die Nordsee leer fischt\\
	 {}\spacebr{}that somebody.\nom{} the.\acc{} North.Sea empty fishes\\
\ex\label{bsp-leerfischung}
\gll wegen   der \emph{Leerfischung}  der    Nordsee\footnotemark\\
     because of.the empty.fishing of.the North.Sea\\
\footnotetext{
        Example from the German newspaper taz, 20.06.1996, p.\,6.%
}
\glt `because of the fishing that resulted in the North Sea being empty'
\zl
In a phrasal world, transformations or meta-rules would be needed to capture the relation between the verbal and
the nominal structures. Note that GPSG-style metarules relate local trees, that is, trees of depth
one. The structure for the noun phrase in (\mex{0}b) is something like (\mex{1}) and more elaborate
than a local tree. 
\ea
{}[ Det [ [ [ Adj V ] -ung ] ] NP[\type{gen}] ]
\z
This means that
transformations with their full power would be needed to relate this structure to verbal
structures. Such powerful transformations were abandoned in all branches of linguistics a long
time ago \citep{Chomsky81a}.


%\section{Conclusions}

In summing up this section, it can be said that either the c-structure configurations do not
contribute any constraints that are relevant for the analysis of argument structure constructions
apart from the \templaten{Benefactive} template itself or they do and then there is a missing
generalization since active and passive c-structures are unrelated.

To relate the c-structure rules or complete trees, one would need meta-rules or transformations, respectively. No such devices are
needed in lexical approaches, in which complex structures are licensed by valence information of
lexical items and abstract rules or schemata.
% (general c-structure rules in LFG, the Head-Complement Schema in HPSG, Merge in Minimalism). 
Rather than relating rules that license certain structures or relating certain structures directly,
lexical items are related by lexical rules.

\chapter{Crosslinguistic generalizations}
\label{sec-generalizations}


\submitOrNormal{\citet{MWArgSt} argued}{In \citet{MWArgSt} we argued} that the approach to Swedish caused motion constructions by
\citet*{ADT2008a,ADT2013a} would not carry over to German since the German construction interacts with derivational
morphology. \citet{AT2014a} argued that Swedish is different from German and hence there would not
be a problem. However, the situation is different with the benefactive construction and with
resultative constructions. Although English and German do differ in many respects, both languages
have similar benefactive and resultative constructions.

In the following subsections I discuss the properties of these constructions in detail and show that
a lexical account works for both German and English while a phrasal account does not extend to less
configurational languages like German.

\section{The benefactive construction}

German has a benefactive construction that is rather similar to the English construction. 
\eal
\ex He baked her a cake.
\ex
\label{ex-er-buk-ihr-einen-kuchen} 
\gll Er buk   ihr        einen Kuchen.\\
     he.\nom{} baked her.\dat{} a.\acc{} cake\\
\zl
%\largerpage[2]
German differs from English in having a dative case and this affects phenomena like passivization, but in
general the constructions are similar enough to make it worthwhile to look for crosslinguistic
generalizations.
In what follows, I look at ways to account for constituent structure in German and show that all
imaginable ways are incompatible with approaches that assume that arguments are introduced in
certain configurations.

\subsection{Binary branching structures}

The analysis of the free constituent order in German was explained by assuming binary branching
structures in which a VP node is combined with one of its arguments or adjuncts (see Berman
\citeyear[Section~2.1.3.1]{Berman96a-u}; \citeyear{Berman2003a} and also \citealp{Choi99a-u}).
For instance, \citet[\page 37]{Berman2003a} assumes the analysis depicted in
Figure~\vref{fig-berman-clause}.
\begin{figure}[t]
\centerfit{
\begin{forest}
sm edges
[CP
  [\cnull [weil;because]]
  [VP
    [DP [der Vater; the father,roof]]
    [VP 
       [DP [den Jungen;the boy,roof]]
       [VP
         [\vnull [lobt;praises]]]]]]
\end{forest}}
\caption{Analysis of German embedded clauses according to \citet[\page 37]{Berman2003a}}\label{fig-berman-clause}
\end{figure}
The c"=structure rule for VP-argument combinations is provided in (\mex{1}):
\ea
\label{lfg-vp-regel-two}
\phraserule{VP}{
\rulenode{DP\\* (\upsp \subjlfg|\obj|\objtheta) = \down}
\rulenode{VP\\* \up~=~\down}}
\z
The dependent elements contribute to the f"=structure of the verb and coherence/""completeness ensure that all
arguments of the verb are present. One could add the introduction of the benefactive argument to
the VP node of the right-hand side of the rule as in (\mex{1}):
\ea
\label{lfg-vp-regel-three}
\phraserule{VP}{
\rulenode{DP\\* (\upsp \subjlfg|\obj|\objtheta) = \down}
\rulenode{VP\\* \up~=~\down\\*( @\textsc{Benefactive} )}}
\z
However, since the verb-final variant of
(\ref{ex-er-buk-ihr-einen-kuchen}) would have the structure in (\mex{1}), one would get spurious
ambiguities: since the benefactive could be introduced at any of the three VP nodes in (\mex{1}),
one would get three analyses with exactly the same semantics.
\ea
\label{ex-er-ihr-einen-kuchen-buk}
\gll weil    [\sub{VP} er [\sub{VP} ihr [\sub{VP} einen Kuchen [\sub{VP} [\sub{V} buk]]]]]\\
     because {}        he.\nom{} {}        her.\dat{} {}        a.\acc{} cake       {}        {}       baked\\
\z
So the only way to avoid this seems to be to introduce the benefactive at the rule that got the recursion
going, namely the rule in (\mex{1}), which projects the lexical verb to the VP level.
\ea
\label{LFG-v-vp-two}
\phraserule{VP}{
\rulenode{(V)\\* \up~=~\down}}
\z
But this unary branching rule is almost a lexical rule. 
%% stimmt nicht, weil VP koordiniert werden kann.
%% The only difference is again the interaction
%% of the rule with coordination that was discussed with respect to the examples in (\ref{ex-coordination-benefactive}).

Note that there is a further problem for the template-based approach. The traceless approach to the
verb position in German developed by \citet{Berman2003a} assumes that the verb is optional in
(\mex{0}). The optionality is marked by enclosing the V in brackets. Because of the optionality, there is nothing to attach the benefactive template to. Even if one would change
the notational schema of LFG and allow for the attachment of f-structure constraints to mother nodes,
this would not solve the problem since a principle that is called Economy of Expression
(\citealt[\page 81]{Bresnan2001a}; \citealt*[\page 90]{BATW2015a}) removes/avoids nodes without
daughters.\footnote{%
  ``All syntactic phrase structure nodes are optional and are not used unless required by independent
  principles (completeness, coherence, semantic expressivity).'' \citep[\page 90]{BATW2015a}}
The verb-initial variant of (\ref{ex-er-ihr-einen-kuchen-buk})
is given in Figure~\ref{fig-berman-clause-v1}.
\begin{figure}
\begin{forest}
sm edges
 [CP
  [\cnull [buk;baked]]
  [VP
    [DP [er; he]]
    [VP 
       [DP [ihr;her]]
       [VP
         [DP [einen Kuchen;a cake, roof]]]]]]
\end{forest}
\hfill
\begin{forest}
sm edges
 [CP
  [\cnull [buk;baked]]
  [VP
    [DP [er; he]]
    [VP 
       [DP [ihr;her]]
       [VP
         [DP [einen Kuchen;a cake, roof]]
         [VP [\trace]]]]]]
\end{forest}
\mbox{}
\caption{Left: Analysis of German verb-initial clauses in a co-head approach with empty nodes removed
  because of Economy of Expression according to \citet[\page
    41]{Berman2003a}, Right: Analysis with VP introducing the
benefactive template}\label{fig-berman-clause-v1}
\end{figure}  
There is no verbal node to which one can attach the benefactive template and introducing it at the C
node seems counter-intuitive. The natural place for it to be introduced is the verb since it
has to be realized somewhere in the sentence. This is of course the lexical approach. Of course one
could insist on introducing constraints regarding a benefactive argument at the projection in
(\ref{LFG-v-vp-two}). For instance, one could assume that the V is optional and that the annotation
is made at the VP. The result would be the structure at the right in
Figure~\ref{fig-berman-clause-v1}. The V is omitted, but the VP node has to be there since it
contributes the benefactive constraints. So whether there are verb traces or not would depend on the
presence of argument structure changing elements in the clause, a highly counter-intuitive
outcome. Again, if the information about the benefactive argument is introduced lexically, the
left structure in Figure~\ref{fig-berman-clause-v1} can be assumed and no additional assumptions are necessary.

\label{sec-gf-introduced-by-case}%
As an alternative to introducing the benefactive template at a V or VP node, one could assume that the
dative DP introduces the benefactive. \citet{Berman2003a} develops an analysis in which grammatical functions are assigned via
implicational constraints that infer the grammatical function from the case of an NP/DP. 
Figure~\ref{fig-case-gf-berman}, which is a simplified version of the figure she discusses on
p.\,37, shows the implicational constraints and that they are attached to certain phrase structure positions.
\begin{figure}
%\centering
\oneline{%
\begin{forest}
sm edges
[CP,fit=tight
  [\cnull [weil;because]]
  [VP
    [{\begin{tabular}[t]{@{}c@{}}(\down \textsc{case} = nom) $\Rightarrow$ (\up \subjlfg = \down)\\DP\end{tabular}} [der Vater;the father]]
    [VP
       [{\begin{tabular}[t]{@{}c@{}}(\down \textsc{case} = acc) $\Rightarrow$ (\up \obj = \down)\\DP\end{tabular}} [den Jungen;the boy]]
       [VP [V [lobt;praises]]]]]]
\end{forest}}
\caption{Correspondence between case and grammatical function according to \citet[\page 37]{Berman2003a}}\label{fig-case-gf-berman}
\end{figure}%
This proposal was criticized in \citet[Section~7.4]{MuellerGT-Eng1} since case in German cannot be unambiguously related to
grammatical functions. In the case at hand the presence of a dative could be used to infer the
grammatical function of a benefactive argument and hence find a natural place for the attachment of
the benefactive template. However, the situation is not as simple as it first
may appear. In examples like (\mex{1}a) we have a so-called dative passive. The dative object is
promoted to subject and hence gets nominative case. When verbal projections are embedded under AcI verbs,
their subject is realized as accusative. (\mex{1}b) shows an example of the embedding of the
benefactive construction under an AcI verb in which the benefactive argument is realized as
accusative. Finally, the nominalization in (\mex{1}c) shows that the benefactive argument can be
realized in the genitive as well.
\eal
\label{ex-benefactive-in-differentcases}
\ex[]{
\gll Der Mann bekam einen Kuchen gebacken.\\
     the.\nom{} man got a.\acc{} cake baked\\
}
\ex[?]{ 
\gll Peter lie\ss{} den Mann einen Kuchen gebacken bekommen und kümmerte sich nicht darum.\\
     Peter let the.\acc{} man a.\acc{} cake baked get   and cared    \textsc{refl} not there.around\\
\glt `Peter permitted that the man got a cake baked and did not care about this.'
}
\ex[]{
\gll Das Kuchen-gebacken-Bekommen der Männer nervt mich.\\
     the cake-backed-get the.\gen{} men nerve me\\
\glt `The getting cake baked of the men annoys me.'
}
\zl
This can be accounted for straightforwardly in a lexical approach in which the dative is a dependent
of \emph{backen}. Either a lexical rule or the auxiliary verb takes care of the fact that the dative
argument has to be realized as nominative in dative-passive constructions like (\mex{0}a) (see
\citew[Section~3.2.3]{Mueller2002b} for details of an auxiliary-based approach in HPSG). When
dative"=passives are embedded under AcI verbs, the subject becomes the object of the AcI verb and
hence receives accusative. And finally, arguments with structural case that are realized in nominal
environments get genitive, as in (\mex{0}c). Nothing special has to be stipulated in the lexical
approach.
A phrasal approach that wants to assign semantic roles based on dative case is lost though.

Note also that the dative can be fronted across clause boundaries:
\ea
\label{ex-dieser-frau-hat-er-behauptet}
\gll Dieser Frau hat er behauptet, nie einen Kuchen zu backen.\\
     this.\dat{} woman has he.\nom{} claimed     never a.\acc{} cake to bake\\
\glt `He claimed that he never bakes this woman a cake.'
\z
A simple model that adds an \objtheta to the f-structure in which a dative appears would fail here, since
the \obj belongs into the f-structure of \emph{backen} rather than into the f-structure of
\emph{behauptet}. This is due to the fact that the benefactive is extracted and not realized within
the VP with the appropriate f-structure (\emph{nie einen Kuchen zu backen} `never a cake to
bake'). The situation is depicted in Figure~\vref{fig-case-gf-benefactive}.
\begin{figure}
%\centering
\oneline{%
\begin{forest}
sm edges
[CP
  [{\begin{tabular}[t]{@{}c@{}}(\down \textsc{case} = dat) $\Rightarrow$ (\up \objtheta = \down)\\DP\end{tabular}} [dieser Frau;this woman, roof]]
  [\cbar
     [C [hat;has]]
     [VP
       [DP [er;he]]
       [VP
          [VP
            [V [behauptet;claimed]]]
          [VP [nie einen Kuchen zu backen;never a cake to bake,roof]]]]]]
\end{forest}}
\caption{Benefactive construction with fronted dative. Assignment of grammatical functions based on
  case would exclude such structures}\label{fig-case-gf-benefactive}
\end{figure}
So one would either have to assume a dative trace in the \emph{backen} VP, something that is usually
not done, or functional uncertainty \citep{KZ89a} would be needed to find the right f-structure or $\sigma$ structure. This 
means that benefactive arguments have to ``know'' where they could come from. This is an unwanted
consequence since the treatment of nonlocal dependencies should be independent of the benefactive construction.

%\largerpage[2]
The examples in (\ref{ex-benefactive-in-differentcases}) show that the benefactive argument, which is realized as \objtheta in
active sentences can be realized as \subjlfg (\mex{0}a) or as \obj (\mex{0}b). This
means that one cannot assume a c-structure rule that combines an \objtheta DP with a verb and
(optionally) attach the benefactive template to this DP. Rather one has to say that subjects, objects and secondary
objects may be benefactive arguments. This means that one could assume that the benefactive template
is optionally associated with the DP in the c-structure rule in (\ref{lfg-vp-regel-three}), but this
would result in the same spurious ambiguities that result from an attachment to the VP node. 


\subsection{Flat structures}

A reviewer of Joint 2016 Conference on Head-driven Phrase Structure Grammar and Lexical Functional
Grammar (HeadLex2016) suggested that flat structures could be assumed for German as well. The
first problem with this is that most authors working in HPSG and LFG follow \citet{Haider93a}
in assuming that there is no IP/VP separation in German. For finite verbs it is assumed that
subjects are realized in the verbal domain just like other arguments are (\citealp[Section~3.2.2, Section~3.2.3]{Berman2003a}; \citealp[\page 412]{ZK2002a}). So for German one would
have to assume a c-structure rule that includes the subject (as \citealp[\page 412]{ZK2002a} do) and
hence would have a rule that differs from the c-structure rule for English. A missed generalization.

% Stimmt nicht, weil die Stelligkeit verschieden wäre. Gilt nur für den lexikalischen Ansatz.
%% \footnote{%
%%   In a HPSG setting this would not be a problem since subjects of finite verbs are treated as
%%   complements and one could use a flat schema that combines the verb with all complements.
%%   The same schema would work for English VPs and German finite clauses. See \citew{MuellerGermanic}.%
%% }


\subsubsection{Adjuncts}

\largerpage[2]
Furthermore, German differs from English in allowing adjuncts to appear everywhere between the
arguments of a verb. So, all of the following sentences are possible:
\eal
\ex 
\gll dass der Mann seiner Frau den Kuchen morgen bäckt\\
     that the man  his    wife the cake   tomorrow bakes\\
\glt `that the man will bake his wife the cake tomorrow'
\ex 
\gll dass der Mann seiner Frau morgen den Kuchen bäckt\\
     that the man  his    wife tomorrow the cake   bakes\\
\glt `that the man will bake his wife the cake tomorrow'
\ex 
\gll dass der Mann morgen seiner Frau den Kuchen bäckt\\
     that the man  tomorrow  his    wife the cake   bakes\\
\glt `that the man will bake his wife the cake tomorrow'
\ex 
\gll dass morgen der Mann seiner Frau den Kuchen bäckt\\
     that tomorrow the man  his    wife the cake    bakes\\
\glt `that the man will bake his wife the cake tomorrow'
\zl
As \citet{Uszkoreit87a} has shown, all adjunct positions can be filled simultaneously and it is also
possible to have more than one adjunct per adjunct position. The modified flat c-structure would
look as in (\mex{1}):\footnote{%
  See \citet[\page 146]{Uszkoreit87a} and \citet{Kasper94a} for similar flat rules in GPSG and HPSG,
  respectively.
}

\ea\label{c-struc-vp-benefactive-resultative-optional-args-adjuncts}
\resizebox{.99\linewidth}{!}{%
\begin{tabular}[t]{@{}l@{~}l@{}}
{V$'$} $\to$ & \rulenode{XP*\\*\down\kern .2em $\in$ (\up\ \adj)} \rulenode{(DP)\\*(\up\ \subjlfg) =
  \down} \rulenode{XP*\\*\down\kern .2em $\in$ (\up\ \adj)} \rulenode{(DP)\\*(\up\ \obj) = \down}\\[10mm]
             &  \rulenode{XP*\\*\down\kern .2em $\in$  (\up\ \adj)} \rulenode{(DP)\\*(\up\ \objtheta) = \down} \rulenode{XP*\\*\down\kern .2em $\in$ (\up\ \adj)}
\rulenode{(V)\\* \up~=~\down\\*( @\textsc{Benefactive} )}
\end{tabular}}
\z
The `*' stands for arbitrarily many repetitions. While this rule works for German, it is
inappropriate for English. One could say that English has a more specific version of this rule,
namely a rule in which the number of possible adjuncts is specified to be zero. However, this would beg the
question how the more general rule could play a role in the grammar of English. One would have to
stipulate that the language acquisition process somehow involves rules like (\mex{0}) but the
English speaking children have to find out that they cannot use adjuncts in the respective
slots. This is implausible if one does not want to assume that rules like (\mex{0}) are innate and
language learners derive more specific instances from them. So, again there are differences in the
grammars of German and English that prevent phrasal approaches from capturing the commonalities of
argument structure constructions.


%% Appart from this first difference between German and English there are several problems for analyses
%% that assume a flat structure. The following paragraphs deal with the traceless analysis of verb
%% position in German, with interactions of the benefactive construction and passive and control, with
%% verbal complexes, and with coordination and partial verb phrase fronting. I show that these areas in
%% German grammar are problematic for a phrasal approach that relies on syntactic configurations for
%% the introduction of arguments.

%% \subsubsection{Traceless accounts of German clause structures}

%% German is a V2 language and it is usually assumed across frameworks (GB, \citealt{Haider93a}; LFG,
%% \citealt{Berman2003a}; HPSG, \citealt{Kiss95a}, \citealt{Meurers99c}, \citealt{Mueller2005d,MuellerGS}) that
%% the finite verb is realized in the C position in V1 and V2 sentences.\todostefan{add sources} 
%% \citet{Berman2003a} developed a co-head account of the German clause structure in which both the C
%% position and the verbal projections can contribute to the same f-structure. If the finite verb is realized in C it contributes its
%% f-structure information from there. The analysis of (\mex{1}) is depicted in Figure~\ref{fig-baekt-er-ihr-einen-Kuchen}.
%% \ea
%% \gll Bäckt er ihr einen Kuchen.\\
%%      bakes he.\nom{} her.\dat{} a.\acc{} cake\\
%% \glt `Does he bake her a cake.'
%% \z
%% \begin{figure}
%% \begin{forest}
%% sm edges
%% [CP
%%   [C [bäckt; bakes]]
%%   [VP 
%%     [DP [er;he]]
%%     [DP [ihr;her]]
%%     [DP [einen Kuchen;a cake,roof]]]]
%% \end{forest}
%% \caption{Analysis of \emph{Bäckt er ihr einen Kuchen?} with VP external
%%   verb}\label{fig-baekt-er-ihr-einen-Kuchen}
%% \end{figure}
%% If the verb is an optional constituent of the VP one does not need verb traces. Of course this
%% traceless account can be extended to XP extraction as well. All NPs/DPs that are involved in
%% benefactive constructions can be extracted as (\mex{1}) shows:
%% \eal
%% \ex 
%% \gll Er bäckt ihr einen Kuchen.\\
%%      he.\nom{} bakes her.\dat{} a.\acc{} cake\\
%% \glt `He bakes her a cake.'
%% \ex 
%% \gll Ihr bäckt er einen Kuchen.\\
%%      her.\dat{} bakes he.\nom{} a.\acc{} cake\\
%% \ex 
%% \gll Einen Kuchen bäckt er ihr.\\
%%      a.\acc{} cake bakes he.\nom{} her.\dat\\
%% \zl
%% This means that all the parts of the assumed benefactive c-structure must be optional if traces are
%% to be avoided. But if all parts of the c-structure are optional, there is nothing left to attach the
%% benefactive template to. Again the only option seems to be to assume empty elements for the
%% introduction of the benefactive or to extend the LFG notation and introduce the benefactive
%% constraints at the mother node.

%% \subsubsection{Benefactive in interaction: passive and control}

%% But even with traces the parts of the benefactive c-structure would have to be optional since not
%% all parts are realized in all environments:
%% \eal
%% \ex
%% \gll Er bäckt ihr einen Kuchen.\\
%%      he.\nom{} bakes her.\dat{} a.\acc{} cake\\
%% \glt `He bakes her a cake.'
%% \ex
%% \gll Ihr wurde ein Kuchen gebacken.\\
%%      her.\dat{} was   a.\nom{} cake     baked\\
%% \ex 
%% \gll Sie bekam einen Kuchen gebacken.\\
%%      she.\nom{} got   a.\acc{} cake baked\\
%% \ex 
%% \gll Sie versucht, [einen Kuchen gebacken zu bekommen].\\
%%      she.\nom{} tried     \spacebr{}a.\acc{} cake baked to get\\
%% \zl
%% (\mex{0}a) is the normal benefactive construction in the active with all parts realized. (\mex{0}b)
%% is the agentive passive with \emph{werden} `be', in which the subject is suppressed. (\mex{0}c) is
%% the dative passive in which the dative argument of a participle is realized as subject, and
%% (\mex{0}d) is the most interesting case, namely a control construction in which the subject of the
%% controlled verbal projection is not expressed. Since the subject of \emph{einen Kuchen gebacken zu
%%   bekommen} `a cake baked get' is the benefactive argument and since these subjects are not empty pronominal elements
%% in LFG, there is no benefactive element present in the respective structures for such
%% constructions. So it is unclear where the information about grammatical functions could be attached
%% to in cases like (\mex{0}d). In comparison, the examples in (\mex{0}) are entirely unproblematic for
%% lexical accounts. The benefactive behaves like a normal dative argument and the dative passive and
%% control constructions are perfectly well understood. Nothing special has to be said about
%% benefactive constructions apart from the lexical rule that licenses them.

\subsubsection{Scrambling}

German differs from English in allowing for almost free orderings of arguments. This also affects
benefactives as is shown by (\mex{1}):
\eal
\label{ex-scrambling-benefactive}
\ex 
\gll dass der Mann seiner Frau einen Kuchen bäckt\\
     that the man  his    wife a cake bakes\\
\glt `that the man bakes his wife a cake'
\ex 
\gll dass der Mann einen Kuchen seiner Frau bäckt\\
     that the man  a cake       his wife    bakes\\
\glt `that the man bakes his wife a cake'
\ex 
\gll dass dieser Frau jeder Mann einen Kuchen bäckt\\
     that this   woman every man a cake bakes\\
\glt `that every man bakes this woman a cake'
\ex 
\gll dass dieser Frau solchen Kuchen niemand bäckt\\
     that this woman  such.a  cake   nobody  bakes\\
\glt `that nobody bakes this woman such a cake'
\ex 
\gll dass einen Kuchen dieser Frau niemand bäckt\\
     that a cake this woman nobody bakes\\
\glt `that nobody bakes this woman a cake'
\ex 
\gll dass einen Kuchen niemand dieser Frau bäckt\\
     that a cake nobody this woman bakes\\
\glt `that nobody bakes this woman a cake'
\zl
This can be captured by either stating six c-structure rules that all involve the benefactive
template or by using just one c-structure rule that does not specify the grammatical functions of
the involved DPs. See \citew[\page 413]{ZK2002a} for the latter approach.\footnote{%
  A simplified version of Zaenen \& Kaplan's rule is given in Footnote~\ref{fn-zk-rule}. Their rule
  only deals with NPs/DPs. German also allows for the scrambling of PPs, APs, VPs and even CPs. So the
  category and the grammatical functions that are assigned in such a general rule for the German
  clause have to be more inclusive.
} In any case the c-structure rule or
rules would differ from what was assumed for English and there would be no way to capture the generalization.
 

\subsubsection{Verbal complexes}
\label{sec-verbal-complex}

Apart from these differences between English and German, phrasal accounts are challenged by the fact that the verb may be separated from the benefactive DP/NP by an auxiliary:
\ea
\label{ex-wird-backen-muessen}
\gll dass er ihr einen Kuchen wird backen müssen\\
     that he.\nom{} her.\dat{} a.\acc{} cake       will bake   must\\
\glt `that it will be necessary for him to bake her a cake'
\z
Sentences like (\mex{0}) are usually analyzed by assuming that \emph{backen} and \emph{müssen} form
a verbal complex, which is in turn embedded under the future auxiliary \emph{wird}
(\citealt{Bech55a}; \citealt{HN89a,HN94a}; \citealp{Haider90a}; \citealt{Kiss95a};
\citealt{Meurers99a}; \citealt{Kathol2001a}; \citealt{Mueller99a,Mueller2002b};
\citealt[Section~3.2.4]{Berman2003a}; \citealt{FR2009a-u}). The complete
verbal complex is combined with \emph{einen Kuchen}, \emph{ihr} and \emph{er}. There have been
proposals for a flat analysis of sentences containing a verbal complex \citep{BvN98} but these
relied on argument attraction and suggested a very general dominance schema. Of course one could (optionally) add
the benefactive template to a very general schema but this solution would not be an implementation
of the pattern-based constructional approaches in which it is assumed that certain specific
configurations license the introduction of specific arguments.

Note that the benefactive template cannot be attached to the VP node or to the dative DP. The benefactive template
would add arguments to the $\sigma$-structure of \emph{wird} since \emph{backen} is embedded under
\emph{wird} and \emph{müssen}. \emph{wird} introduces a tense relation and \emph{müssen} a modal
operator. Depending on the analysis of the semantic structure, the \argtwo and \argthree referred to
in the \templaten{Benefactive} template would end up in the $\sigma$ structure of \emph{wird} or
\emph{müssen} rather than in the $\sigma$ structure of \emph{backen}.\footnote{%
  This is not a problem for grammatical functions in f-structures  since raising predicates are assumed to have
  additional slots for raised arguments, but this should not be the case for semantic
  representations.
}


To make things even more complicated for the phrasal account, the verbal complexes can be coordinated. (\mex{1}) illustrates:
\ea
\gll dass er ihr einen Kuchen wird backen müssen oder hat backen sollen\\
     that he.\nom{} her.\dat{} a.\acc{} cake       will bake   must   or   has bake   shall\\
\glt `that it will be necessary for him to bake her a cake or that he should bake her a cake'
\z
Such sentences can be accounted for easily if one assumes that \emph{wird backen müssen} and
\emph{hat backen sollen} form verbal complexes which are then coordinated and finally combined with
the other NPs/DPs in the sentence. The structure for (\ref{ex-wird-backen-muessen}) with a flat VP
including the subject is given in Figure~\vref{fig-wird-backen-muessen}.\footnote{%
  This structure is equivalent, modulo node names, to what Zaenen \& Kaplan's rules \citeyearpar[\page 413]{ZK2002a} would license. Zaenen \& Kaplan's rule incorporate a linearization constraint that prevents orders in
  the verbal complex in which the governing verb is not final. So they exclude so-called Auxiliary
  Flip. But this can be fixed easily.
}

\begin{figure}
\centering
\begin{forest}
sm edges
[VP
  [DP [er;he]]
  [DP [ihr;her]]
  [DP [einen Kuchen;a cake,roof]]
  [VC
      [V [wird;will]]
      [VC 
        [V [backen;bake]]
        [V [müssen;must]]]]]
\end{forest}
\caption{Analysis of German clause involving a verbal complex and a flat clausal structure}\label{fig-wird-backen-muessen}
\end{figure}

\citet{BKPZ82a-u} developed an account of cross"=serial dependencies that explains sentences like
(\mex{1}) by assuming that objects and prepositional objects are part of a verbless VP with the verb
being realized in the verbal complex. 
\ea
\gll dat Jan Piet Marie de kinderen zag helpen laten zwemmen\\
     that Jan Piet Marie the children saw help let swim\\
\glt `that Jan saw Piet help Marie make the children swim'
\z
Since both the content of the verbless VP and the respective verb in the verbal complex are mapped
to the same f-structure (the \vcomp value of the respective mother node), the objects are mapped to
the correct f-structure. Since there is a VP in this account one could be tempted to believe that
this account could be extended to German and a resultative or benefactive VP could be assumed for
German, which would make the argument above irrelevant.

I do not want to discuss the details of Bresnan et al.'s proposal here but simply want
to point out that it would not extend to German since, first, subjects can be scrambled with other
arguments so a VP without the subject would not be an appropriate unit to begin with and, second,
subjects of higher verbs may be scrambled with objects of embedded verbs, as is demonstrated by the
examples in (\mex{1}):
\largerpage
\eal
\ex 
\gll dass den Mann       seiner Frau     solchen Kuchen niemand backen sah\\
     that the.\acc{} man his.\dat{} wife such.\acc{} cake nobody.\nom{} bake saw\\
\glt `that nobody saw the man bake his wife such a cake'
\ex 
\gll dass den        Mann niemand seiner Frau solchen Kuchen backen sah\\
     that the.\acc{} man  nobody.\nom{} his.\dat{} wife such cake bake saw\\
\glt `that nobody saw the man bake his wife such a cake'
\ex
\gll dass solchen Kuchen niemand den Mann seiner Frau backen sah\\
     that such.\acc{} cake nobody.\nom{} the.\acc{} man his.\dat{} wife bake saw\\
\zl
These are three examples exemplifying the phenomenon but in principle all permutations of
arguments belonging to verbs of a verbal complex are possible. Of course there are problems with
arguments of the same case when there is not sufficient context information to resolve which
argument fills which role, but this is also the case with simplex verbs. See \citew[Section~11.4]{Mueller99a} for discussion.
  
The point about these examples is that unless one is willing to include the subject of AcI verbs
among the daughters of a very flat phrase structure rule, there is no way to model sentences like
(\mex{0}) with a flat rule for benefactives like (\ref{c-struc-vp-benefactive-optional-args}) or
(\ref{c-struc-vp-benefactive-resultative-optional-args-adjuncts}) and continuous
constituents.\footnote{\label{fn-zk-rule}%
The rule in (i) is an adapted version of the rule that \citet[\page 413]{ZK2002a} use to
describe sentences with verbal complexes:
\ea
\label{lfg-vp-regel-zk}
\phraserule{VP}{
\rulenode{DP*\\* (\upsp \comp* \textsc{NGF}) = \down}
\rulenode{(V$'$)\\* \up~=~\down}}
\z
The `*' after the DP symbol means that there can be arbitrarily many DPs. The grammatical function
that is filled by the DPs is not specified. It is just specified that the DP has to fill an NGF,
where NGF stands for \emph{nominal grammatical function}, that is \subjlfg, \obj, OBJ2, \ldots. The
\comp* ensures that the specification of the nominal grammatical function can reach any feature
structure at the end of a path of several \comp{}s. This solves the problem that \emph{einen Kuchen}
in (\ref{ex-wird-backen-muessen}) is the object of \emph{backen}. By using the functional
uncertainty it is possible to assign the \obj function to the f"=structure of \emph{backen}. But
note that this schema is very general. Mentioning benefactives on either the DP or the V$'$ would
not capture the constraints that are supposed to be associated with a constructional pattern for
benefactives. Independent of where the benefactive template is introduced one would need functional
uncertainties to find an appropriate verb in the verbal complex. 

Note furthermore that Zaenen \& Kaplan's rule is too simple. Their rule and the version provided
here only admits NPs/DPs before the verbal complex but German allows for NPs, PPs, APs and even VPs
and CPs and there is no fixed order of these elements. So rather than specifying DP* and appropriate
grammatical function assignments, one would have to specify (DP|PP|AP|VP|CP)* and appropriate
grammatical functions.

In any case the idea of isolating a constructional pattern for benefactives would
not be captured by such a proposal and again the rule in (i) and possible adaptions are very different from the c"=structure
rule for English.
}


Having a rule that combines \emph{backen} with four NPs is 
highly implausible though, since the nominative depends on the AcI verb and not on \emph{backen}. In
any case such a rule would be inappropriate for English. See also \citew[Section~2.5]{Mueller2006d} for examples of
the interaction of AcI verbs with resultative constructions.

\subsubsection{Coordination, fronting and partial structures}

\citet{FR2009a-u} assume a flat VP for German to account for certain coordination
structures. However, their theory of coordination assumes partial VPs. In the analysis of (\mex{1}), the VP \emph{seiner Frau buk} `his wife baked' would
be coordinated with \emph{seiner Tochter zeigte} `his daughter showed'.
\ea
\gll dass er den Kuchen [seiner Frau buk] und [seiner Tochter zeigte]\\
     that he.\nom{} the.\acc{} cake \spacebr{}his.\dat{} wife baked but \spacebr{}his.\dat{} daughter showed\\
\glt `that he baked his wife a cake and showed it to his daughter'
\z
These partial VPs are parallel to the VPs in approaches with binary
branching. Any LFG analysis of German has to admit such partial VPs since German allows for partial VP fronting:\footnote{%
  Again see \citew{ZK2002a} for an account of partial VP fronting in German in the framework of LFG.
}
\eal
\ex
\gll {}[Seiner Frau backen] würde er solche Kuchen niemals.\\
     \spacebr{}his.\dat{} wife bake would he.\nom{} such.\acc{} cakes never\\
\glt `He would never bake such cakes for his wife.'
\ex
\gll {}[Solche Kuchen backen] würde er seiner Frau niemals.\\
     \spacebr{}such.\acc{} cakes bake would he.\nom{} his.\dat{} wife never\\
\ex
\gll {}[Backen] würde er seiner Frau solche Kuchen niemals, kaufen schon.\\
     \spacebr{}bake would he.\nom{} his.\dat{} wife such.\acc{} cakes never      buy    \textsc{part}\\
\glt `He would never bake his wife such a cake, but he would buy one.'
\zl
Hence the idea that the benefactive is introduced in a special phrase structural configuration
together with a verb and all other objects would
not work for German. See \citet{Nerbonne86a} and \citet{Johnson86a}, who introduced lexical valence
representations in a Categorial Grammar style into GPSG since there was no way to make the phrasal
GPSG approach compatible with German partial VP fronting data. 

Note also that such frontings can occur with a modal as main verb:
\ea
\gll {}[Einen solchen Kuchen backen] musste er seiner Frau noch nie.\\
     \spacebr{}a.\acc{} such    cake   bake   must   he.\nom{} his.\dat{}    wife yet never\\
\glt `He never had to bake his wife such a cake.'
\z
Such examples also pose problems for adding f-structure information since the dative appears in the
domain of the modal rather than the domain of the main verb (see previous paragraph).

None of these data poses a problem for standard LFG: the lexical analysis of benefactives that was suggested by \citet{Toivonen2013a}
interacts with the analysis of partial verb phrase frontings suggested by \citet{ZK2002a} without
further ado.

\subsection{Other environments}
\label{sec-adjectival-participles-benefactive}

Note also that benefactive datives appear in adjectival environments as in (\mex{1}):
\eal
\ex
\gll der seiner Frau einen Kuchen backende Mann\\
     the his.\dat{} wife a.\acc{} cake baking man\\
\glt `the man who is baking a cake for his wife'
\ex
\gll der einen Kuchen seiner Frau backende Mann\\
     the a.\acc{} cake  his.\dat{} wife baking man\\
\glt `the man who is baking a cake for his wife'
\zl
The examples in (\mex{0}) show that the arguments of \emph{backende} may be scrambled, as is common
in verbal environments.

In order to account for these datives one would have to assume that the adjective-to-AP rule that
would be parallel to (\ref{LFG-v-vp-two}) introduces the dative. The semantics of the benefactive
template ensures that the benefactive argument is not added to intransitive
verbs like \emph{lachen} `to laugh' or participles like \emph{lachende} `laughing'. 
While this is a possible analysis, I find the overall approach unattractive. First, it does not have anything to do with
the original constructional proposal but just states that the benefactive may be introduced at
several places in syntax. Second, the unary branching syntactic rule applies to a lexical
item and hence is very similar to a lexical rule. Third, the analysis does not capture cross"=linguistic commonalities of the
construction. In a lexical rule"=based approach such as the one that was suggested by
\citet[Section~5]{BC99a}, \citet{Cook2006a-u}, \citet{Kibort2008a}, and \citet{Toivonen2013a}, a
benefactive argument is added to certain verbs and the lexical rule is parallel in all languages
that have this phenomenon. The respective languages just differ in the way the arguments are
realized in respect to their heads. In languages that have adjectival participles, these are derived
from the respective verbal stems. The morphological rule is general and does not refer to benefactive
arguments and the syntactic rules for adjectival phrases do not have to mention benefactive
arguments either. 

\subsection{Summary}

I showed in this section that it is not viable to introduce the benefactive argument in binary
branching structures since there is no canonical place for doing so. Introducing it at the VP
recursion results in spurious ambiguities. Introducing it at the rule that gets the recursion going
is almost equivalent to the lexical approach and in any case it would not have anything to do with a
specific configuration that licenses the construction. Making the benefactive template dependent on
the presence of a DP/NP with a certain case fails for several reasons: first, the benefactive
argument may be realized in various cases and, second, it may be realized far away from its canonical
place and hence all constraints referring to the f-structure or the $\sigma$ structure would
potentially address wrong structures because of nonlocal dependencies. I furthermore showed that
flat structures are not an option either since partial structures are needed for partial verb phrase
fronting and coordination, and in any case flat structures may be interrupted by verbal complexes
that embed the main verb under modal operators, again leading to the inaccessibility of the relevant
f-structures and $\sigma$ structures.

The lexical approach adds information right at the place where the necessary information is
accessible. None of the discussed problems affects the lexical approach.


\section{Resultative constructions}


Having discussed the benefactive construction, I now turn to Christie's analysis of resultative constructions. \citet{Christie2010a}
suggests the following c-structure rule for resultatives in English:
\ea
\label{c-struc-result-two}
\phraserule{V$'$}{
\rulenode{V\\* \up~=~\down}
\rulenode{DP\\*(\up\ \obj) = \down}
\rulenode{\{ DP|AP|PP \}\\*(\up\ \xcomp) = \down\\*(\down \subjlfg) = (\up \obj)\\*\hspace{-3em}@\textsc{result-t}((\up \textsc{pred} \textsc{fn})) }\hspace{-2ex}
}
\z
Christie claims that the result predicate cannot be extracted. According to her examples like
(\mex{1}) are ungrammatical (p.\,157):
\ea[?]{
Pink, Kim dyed her hair.
}
\z
She rates the example with a `?' rather than a `*', but examples of this kind have been frequently
cited in the literature as well-formed (see also the discussion of
(\ref{ex-resultative-just-verb-remains}) on p.\,~\pageref{ex-resultative-just-verb-remains}) and
corpus examples like (\mex{1}a) can be found:
\eal
\ex  What Color Should You Dye Your Hair?\footnote{%
  \url{http://www.gurl.com/2011/06/28/quiz-what-color-should-you-dye-your-hair-beauty/}, 2016-05-28.
}
\ex How flat did John hammer the metal?\footnote{%
  \citew[\page 115]{Roberts88a-u}.
}
\zl
Examples like (\mex{-1}) and (\mex{0}) are usually treated by
functional uncertainty. The element that is extracted is declared to be optional in the
c-structure. The f-structure slot is filled via a functional uncertainty equation. The problem with
an optional result phrase in (\mex{-1}) is that (\mex{-1}) could be used to analyze simple
verb phrases of strictly transitive verbs. This would result in two analyses of sentences with
transitive verbs with exactly the same f-structure. A clearly unwanted result. Of course one could
argue that the rule in (\ref{c-struc-result-two}) is supposed to cover both strictly transitive
verbs and transitive verbs with a result predicate, but this would not capture the original
constructional idea that the phrasal configuration somehow is connected to the meaning contributed
by the pattern \citep{Goldberg95a,GJ2004a}.
The problem with result predicate extraction could be fixed by shifting the call to the \textsc{result-t}
template to the verb since the verb cannot be extracted. However, this shifting the template to the verb would not help in the case of
German. German V2 and V1 clauses are usually analyzed as involving head-movement of the verb. C and
V are treated as co-heads and the functional information is contributed by the finite verb in C
rather than by an empty element in the VP. The analysis is shown in Figure~\vref{fig-flat-vp-verb-movement-resultatives}.
\begin{figure}
\centering
\begin{forest}
sm edges
[CP
  [C [hämmert; hammers]]
  [VP 
    [DP [Peter;Peter]]
    [DP [das Metall;the metal]]
    [AP [flach;flat]]]]
\end{forest}
\caption{Analysis of a V1 sentence with verb-position via C/V co-heads and an English-style flat VP}\label{fig-flat-vp-verb-movement-resultatives}
\end{figure}
The consequence of this is that all elements that would be part of a
resultative VP can in fact be realized outside of this VP: the subject, the object and the result can be
extracted and the verb can be realized in C. To illustrate this, the elements that are missing from
the VP are indicated by \_$_i$ and \_$_j$ in (\mex{1}):
\eal
\ex 
\gll Peter$_i$ hämmert$_j$ [\sub{VP} \_$_i$ das Metal flach \_$_j$].\\
     Peter     hammers     {}        {}    the metal  flat\\
\ex 
\gll Das Metal$_i$ hämmert$_j$ [\sub{VP} Peter \_$_i$ flach \_$_j$].\\
     the metal     hammers     {}        Peter {}     flat\\
\ex 
\gll Flach$_i$ hämmert$_j$ Peter [\sub{VP} das Metal \_$_i$ \_$_j$].\\
     flat      hammers     Peter {}        the metal\\
\zl
Hence, there is no reliable element to attach the resultative template to. The only sensible option 
%% that
%% seems to be left if one wanted to insist on such a phrasal analysis seems to be to stipulate a special
%% resultative VP construction for (\mex{0}c) with the resultative template attached to
%% the object. This is basically the solution that was criticized in \citew[Section~2.2]{Mueller2006d} since it
%% requires the stipulation of special constructions for cases in which parts of a construction are
%% dislocated. 
seems to be the extension of LFG's c-structure annotation conventions: the
resultative template would be added to the mother node of the VP.  

Furthermore, German differs from English in forming verbal complexes, as was already discussed in
Section~\ref{sec-verbal-complex}. \citet[Section~5.1]{Mueller2002b} argued that result adjectives should also be treated as part of the predicate complex. Hence, the structure of
(\mex{1}a) differs from the one of the corresponding English example.
\eal
\ex dass er das Metal flach hämmert
\ex that he hammers the metal flat
\zl
The respective structures are shown in Figure~\vref{fig-resultative-English-German}.
\begin{figure}
\centering
\begin{forest}
[,phantom,s=4em
[VP
  [V [hammers]]
  [DP [the metal,roof]]
  [AP [flat]]]
[VP
  [DP [das Metall,roof]]
  [VC 
    [A [flach]]
    [V [hämmert]]]]]
\end{forest}
\caption{Resultative structures in English and German with a verbal complex for German}\label{fig-resultative-English-German}
\end{figure}
As explained above, both the adjective and the verb and even adjective and verb simultaneously may
be realized outside of the verbal complex, so there is no reliable element to attach the resultative
template too. One could attach it to the mother node of the verbal complex but this would not
include the object as in English or one could attach it to the VP dominating the object and the
verbal complex but the latter proposal could not even enforce the presence of a result predicate of
a certain category and a verb since adjective, verb and object are not in the same local tree.

In the lexical approach the template is combined with the verb directly both in German
and in English (and other languages). The verb can be realized in C or in V. It contributes valence
information to the f-structure that belongs to the V projection either as head or co-head and
this ensures that the result predicate and the object has to be present in the
f-structure. Extracted elements are contributed to the f-structure via functional uncertainty.

Concluding this subsection, it can be said that the fact that result adjectives form a verbal complex in German while they are part of the
VP in English could not be covered if the use of templates required similar structures
cross"=linguistically. The lexical approach, on the other hand, does not have problems since the
lexicon just states which arguments are needed leaving the actual realization to the syntax, which
may be different from language to language.

\section{Interaction between the benefactive and the resultative construction}

As was already pointed out in \citew[\page 861]{Mueller2006d}, the benefactive construction and the
resultative construction interact. The example in (\mex{1}c) shows that both a dative argument and an
accusative argument may be added to the valence representation of a verb.
\eal
\ex[]{
\gll {}[dass] er fischt\\
     \spacebr{}that he fishes\\
}
\ex[*]{
\gll {}[dass] er ihm fischt\\
     \spacebr{}that he him fishes\\
}
\ex[]{\label{ex-dass-er-ihm-den-Teich-leer-fischt}
\gll {}[dass] er ihm den Teich leer fischt\\
     \spacebr{}that he.\nom{} him.\dat{} the.\acc{} pond empty fishes\\
}
\zl

In order to cover this in a phrasal analysis, one would need a resultative/benefactive c-structure rule like
(\mex{1}):\footnote{%
  This ignores the possibility of inserting adjuncts and the option to scramble the DP arguments.
}
\ea\label{c-struc-vp-benefactive-resultative-optional-args}
%\resizebox{.99\linewidth}{!}{%
\begin{tabular}[t]{@{}l@{~}l@{}}
{V$'$} $\to$ & 
  \rulenode{(DP)\\*(\up\ \subjlfg) = \down}
  \rulenode{(DP)\\*(\up\ \obj) = \down} 
  \rulenode{(DP)\\*(\up\ \objtheta) = \down}\\\\[-2mm]
%
& \rulenode{(\{AP|PP\})\\*(\up\ \xcomp) = \down\\*( @\textsc{result-t}((\up PRED FN)) )}%\hspace{-2ex}
  \rulenode{(V)\\* \up~=~\down\\*( @\textsc{Benefactive} )}
\end{tabular}
%}
\z

The interesting thing about this rule is that all items on the right-hand side are optional. The
rule licenses any combination of these items. In order to avoid overgeneration, it has to be made
sure that exactly the right combination of items is present. This is ensured by the templates that
regulate which grammatical functions have to be realized. The c-structure does not play any role in
this business. Hence we could also assume a lexical approach and even return to binary branching
structures: with binary branching structures each subtree licenses a head with an adjunct
or an argument and it is either the f-structure + coherence and completeness or the glue semantics
that has to make sure that all needed and only those arguments are present.

Note also that the combination of the benefactive and the resultative is hardly acceptable in
English (\citealt[\page 53]{BZ90a}; \citealt*[\page 339]{BATW2015a}) and Norwegian \citep{Tungseth2007a-u}.
\ea[*]{
He fished him the pond empty.
}
\z
So this means that the constituents in the right-hand side of the rule in
(\ref{c-struc-vp-benefactive-resultative-optional-args}) may never be realized simultaneously, if we
want to assume this rule for both German and English. This is a
very strange situation for a phrase structure rule indeed, even more so for a constructional
theory. Note also that the rule in (\ref{c-struc-vp-benefactive-resultative-optional-args}) could
not be learned by speakers of English since they never hear all components simultaneously.
The generalization that has to be captured is that benefactives may be added to verbs with
an accusative object and that accusative objects and result predicates may be added to intransitive
verbs. Lexical rule"=based approaches cover this. The two phenomena are independently covered by two
lexical rules: the benefactive lexical rule requires a verb that governs the accusative and adds an
argument as second argument, which is realized as dative in German. The resultative lexical rule
that is needed for the examples discussed above takes an intransitive verb as input and licenses one
that governs an accusative and a result predicate. This is the same across several languages. What is
different is the interaction between the rules. While German does allow benefactives with
resultative constructions, English does not permit this. So the English rule is more constrained, but
the general form of the rules is similar and generalizations can be captured.

%\chapter{Lexical rules in LFG}

\section{Summary}

I showed in this section that it is difficult to find places for the attachment of the benefactive
and resultative templates in phrasal LFG analyses. Even if it was straightforward to find
attachment sites, the respective c"=structures would be different in English and German. While
having different c"=structures for different languages is common in LFG and not a problem per se,
associating the benefactive and resultative construction with totally different configurations in
the descriptions of the two languages misses a generalization. What has to be accounted for is
that we have the same type of dependency in both languages and this can be expressed in valence
representations in lexical items that interact with syntactic rules of the respective languages \citep{MuellerLexikon}.

\chapter{A lexical approach that can capture the cross"=linguistic generalizations}
\label{sec-lexical-approach-hpsg}

This book has argued for a return to lexical analyses. Analyses of the respective phenomena have been
worked out in LFG by \citet{Simpson83a}, \citet{BZ90a} and \citet{Christie2015a-u} for resultative
constructions in English and \citet{Cook2006a-u}, \citet{Kibort2008a} and \citet{Toivonen2013a} for
English benefactives. These analyses assume lexical items with a certain a-structure and related items
with a different a-structure. I could point to these well-established analyses and leave it at that,
but I want to use the remainder of the book to work out a detailed analysis of the phenomena that
have been mentioned throughout the book and explain how their interactions are captured. The
underlying framework that is assumed is Constructional HPSG \citep{Sag97a} with the basic assumptions regarding
German and more generally Germanic made in \citew{MuellerLehrbuch3,MuellerHPSGHandbook,MuellerGermanic}.
The analysis is able to capture generalizations about the benefactive construction across the
Germanic languages, something that is not possible in LFG since the labels for the arguments
(\argtwo and \argthree) and the grammatical functions of the benefactive argument are different in
German and English.
\eal
\ex He baked her a cake.
\ex 
\gll Er buk ihr einen Kuchen.\\
     he.\nom{} baked her.\dat{} a.\acc{} cake\\
\zl
While \emph{her} is \argtwo and \obj, the corresponding object in German is \argthree and \objtheta
in German. This distinction is important since \emph{her} can be promoted to subject (for speakers
who allow for the passive), but \emph{ihr}, being dative, cannot be promoted to subject in the
normal agentive passive. In what follows, I show how the commonalities between (\mex{0}a) and
(\mex{0}b) can be accounted for.

\largerpage
What is needed is basically two lexical rules: one for the introduction of the
benefactive argument and one for the introduction of resultative predicates and the respective
object. In addition, we of course need syntactic schemata that license the structures of German and
English. These schemata are maximally simple. Four schemata are relevant in the context of this
book: 1) the Specifier-Head Schema, 2) the Head-Complement Schema, 3) the Filler-Head Schema and 4)
the Predicate Complex Schema.

\section{Phrase structure, argument structure mappings and scrambling}

\subsection{Argument structure mappings}

I assume that all lexical items come with a list that contains their arguments, the so-called
argument structure list (\argst). The elements of this list are mapped to valence features. In
English and other SVO languages the first element of the \argstl is mapped to the \textsc{specifier}
feature (\spr) and all other arguments are mapped to the \textsc{complements} list (\comps). In
German and other SOV languages all arguments of finite verbs are mapped to \comps and the value of
the \sprf is the empty list. The lexical items in (\mex{1}) illustrate:
\eal
\label{lex-giv-geben}
\ex lexical entry for the stem \emph{give}:\\
\ms{
spr    & \ibox{1} \sliste{ NP[\type{nom}]$_i$ }\\[2mm]
comps  & \ibox{2} \sliste{ NP[\type{acc}]$_j$, NP[\type{acc}]$_k$ }\\
arg-st & \ibox{1} $\oplus$ \ibox{2}
} 
\ex lexical entry for the stem of \emph{geben} `give':\\
\ms{
spr    & \sliste{ }\\
comps  & \ibox{1} \\
arg-st & \ibox{1} \sliste{ NP[\type{nom}]$_i$, NP[\type{dat}]$_j$, NP[\type{acc}]$_k$ }\\
} 
\zl
Both argument structure lists have the same order, corresponding to agent, recipient and
theme. Because of this, the linking constraints for both English and German are parallel and
generalizations are captured. The languages differ in how the arguments are realized: In English the
first argument is mapped to the \sprl \iboxb{1} and the others to \comps \iboxb{2}, while in German,
the complete \argstl \iboxb{1} is mapped to \comps. Several authors have argued that there is no
structural difference between subjects and objects of finite verbs in languages like German
(\citealp[Section~6.3]{Haider93a}; \citealp[\page 376]{Eisenberg94b}; \citew[\page
  36--37]{Berman2003a}) and this claim is reflected by treating subjects as complements. 


\subsection{Phrase structure rules}

\largerpage
The Figures~\ref{fig-nobody-gives-him-the-book}
and~\ref{fig-niemand-ihm-das-buch-gibt} show how the lexical items can be used in actual analyses.

\begin{figure}
\centerfit{%
\begin{forest}
sm edges
[{V[\spr \eliste, \comps \eliste]}
   [{NP[\type{nom}]} [nobody] ]
   [V\feattab{
      \spr \sliste{ NP[\type{nom}]$_i$ }, \comps \sliste{} }
     [V\feattab{
         \spr \sliste{ NP[\type{nom}]$_i$ },\\
         \comps \sliste{ NP[\type{acc}]$_k$}} 
        [V\feattab{
           \spr \sliste{ NP[\type{nom}]$_i$ },\\
           \comps \sliste{ NP[\type{acc}]$_j$, NP[\type{acc}]$_k$}} [gives] ]
        [{NP[\type{acc}]$_j$} [him] ] ]
     [{NP[\type{acc}]$_k$} [the book,roof ] ] ] ]
\end{forest}}
\caption{Analysis of an English example with a ditransitive verb}\label{fig-nobody-gives-him-the-book}
\end{figure}



\begin{figure}
\centerfit{%
\begin{forest}
sm edges
[{V[\spr \eliste, \comps \eliste]}
   [{NP[\type{nom}]$_i$} [niemand;nobody] ]
   [V\feattab{
      \spr \sliste{ }, \comps \sliste{ NP[\type{nom}]$_i$ } }
      [{NP[\type{dat}]$_j$} [ihm;him] ] 
      [V\feattab{
         \spr \sliste{ },\\
         \comps \sliste{ NP[\type{nom}]$_i$, NP[\type{dat}]$_j$}} 
         [{NP[\type{acc}]$_k$} [das Buch;the book,roof ] ]
         [V\feattab{
           \spr \sliste{  },\\
           \comps \sliste{ NP[\type{nom}]$_i$, NP[\type{dat}]$_j$, NP[\type{acc}]$_k$}} [gibt;gives] ] ] ] ]
\end{forest}}
\caption{Analysis of a German example with a ditransitive verb}\label{fig-niemand-ihm-das-buch-gibt}
\end{figure}
I assume binary branching structures for both German and English. English, being a VO language, is assumed to combine the
head with the first element on the \compsl first, while in the analysis of the German example the last element of the
\compsl is combined first with the head.

The schemata that license these structures are shown in Figure~\ref{fig-spr-head}
and~\ref{fig-head-comp}. Figure~\ref{fig-spr-head} shows a sketch of the Specifier-Head Schema,
which licenses structures with a specifier. These are subject-VP combinations in English and
determiner-\nbar combinations in both English and German.
\begin{figure}
\centering
\begin{forest}
[{H[\spr \ibox{1}]}
  [\ibox{2}]
  [{H[\spr \ibox{1} $\oplus$ \sliste{ \ibox{2} }]}]]
\end{forest}
\caption{Sketch of the Specifier-Head Schema}\label{fig-spr-head}
\end{figure}
The figure shows that the \sprl of the head (marked by H) is split into two parts: a list with
exactly one element \iboxb{2} and a rest \iboxb{1} ($\oplus$ stands for list concatenation). \ibox{2} has to match the element that is
combined with the head. The remaining list \iboxb{1} is the value of the \sprl of the mother
node. Usually this list is the empty list, but see \citew{MOe2013b} for an analysis of object shift
in Danish where multiple specifiers are assumed.

Figure~\ref{fig-head-comp} shows the Head-Complement Schema as it would be needed for English.
\begin{figure}
\centering
\begin{forest}
[{H[\comps \ibox{1}]}
  [{H[\comps  \sliste{ \ibox{2} } $\oplus$ \ibox{1}  ]}]
  [\ibox{2}]]
\end{forest}
\caption{Sketch of the Head-Complement Schema}\label{fig-head-comp}
\end{figure}
The \compsl of the head is split into two lists. One contains exactly one element, the element that
is combined with the head \iboxb{2}. The remainder of the list is passed up to the \compsl of the mother.


For German, I assume that adjuncts may attach to any verbal projection (to be precise, to any
verb-final projection) and in English adjuncts may attach to VPs. Because of the binary branching
structures, the fact that adjuncts can appear anywhere between arguments in German is accounted
for. Adjuncts do not have to be mentioned in argument structure constructions, as would be the case
in phrasal models of German syntax assuming flat structures (see (\ref{c-struc-vp-benefactive-resultative-optional-args-adjuncts}) on page~\pageref{c-struc-vp-benefactive-resultative-optional-args-adjuncts}).


\subsection{Scrambling}

\largerpage
Now, German differs from English in allowing scrambling. Rather than having a fixed constituent
order as in English, German allows for six order variants of sentences with verbs with three
arguments (see (\ref{ex-scrambling-benefactive}) on page~\pageref{ex-scrambling-benefactive}). This can be allowed for by relaxing the order in which heads are combined with their
arguments. The more general schema is provided in Figure~\ref{fig-head-comp-free}.

\begin{figure}
\centering
\begin{forest}
[{H[\comps \ibox{1} $\oplus$ \ibox{2}]}
  [\ibox{3}]
  [{H[\comps  \ibox{1} $\oplus$ \sliste{ \ibox{3} } $\oplus$ \ibox{2}  ]}]]
\end{forest}
\caption{Sketch of the Head-Complement Schema for languages with free
  constituent order}\label{fig-head-comp-free}
\end{figure}
In this version of the Head-Complement Schema the \compsl of the head is split into three lists: the
two lists \ibox{1} and \ibox{2} and a list in the middle that contains exactly one element
\iboxb{3}. \ibox{3} is combined with the head and the \compsl of the mother contains all remaining
complements, namely \ibox{1} $\oplus$ \ibox{2}. This general schema allows for various
instantiations: \ibox{1} and \ibox{2} may contain elements or be empty. If \ibox{1} is empty, we get
VO languages with strict order and if \ibox{2} is empty, we get OV languages with strict order. This
gives the nice result that the grammar of English is more restrictive than the one of German, since
the schema for English is basically the same as in Figure~\ref{fig-head-comp-free} but with the
additional constraint that \ibox{1} is the empty list.



\section{Lexical items and lexical rules}

In what follows, I briefly discuss case assignment and linking. For a more detailed discussion of
case assignment see \citew[Chapter~14]{MuellerLehrbuch1}.

\subsection{Structural and lexical case}

\largerpage
The following lexical items and lexical rules assume a distinction between structural and lexical
case. Roughly speaking, structural case is case that is assigned in certain structures, that is,
case that may change. In contrast, lexical case does not depend on the environment a
lexical item is used in. I assume that verbal arguments that are realized as nominative and
accusative in active sentences bear structural case. Following \citet{Haider86}, the dative in German is treated as a lexical
case. (\mex{1}) shows examples of structural cases:
\eal
\ex 
\gll Der Teich ist leer.\\
     the pond is empty\\
\ex 
\gll Er fischt den Teich leer.\\
     he fishes the pond empty\\
\ex 
\gll Der Teich wird leer gefischt.\\
     the pond is empty fished\\
\glt `the pond is fished empty'
\ex 
\gll das Leerfischen des Teiches\\
     the empty.fishing the pond\\
\glt `the fishing of the pond empty'
\zl

Case is assigned according to the following case principle \citep{Prze99,Meurers99b}:\footnote{%
  This Case Principle is a declarative version of the case assignment theory that was developed by
  \citet*{YMJ87}.
}

\begin{principle-break}[Case Principle]
\label{case-p}
\begin{itemize}
\item In a list that contains both subjects and complements of a verbal head,
the first element with structural case is assigned nominative case unless it is raised
to a dominating head.
\item All other elements of this list with structural case are assigned accusative case.
\item In nominal environments all elements with structural case are assigned genitive case.
\end{itemize}
\end{principle-break}

This principle is not specific to German and English. It accounts for the case assignment of many languages, for
instance Icelandic \citep{MuellerGermanic} and also Hindi \citep{MuellerCoreGram}.

\subsection{Linking}

(\ref{lex-giv-geben}) showed the argument structure of \emph{give} and how the elements of the
\argstl are distributed to the valence features. Assuming the distinction between lexical and
structural cases, we have the \argstv in (\mex{1}). The referential indices of the NP arguments are
linked to semantic roles of the predicate \emph{geben}. Instead of traditional role names like
agent, recipient and theme, I use the features \textsc{arg1}, \textsc{arg2} and
\textsc{arg3}.\footnote{%
Since semantic relations correspond to types and types are specified for the features that are
appropriate for objects of the type, it follows that \textsc{arg1}, \textsc{arg2} and \textsc{arg3} are always
present in feature structures of type \type{geben}. The AVM in (\mex{1}) corresponds to the more
canonical notation $geben(x,y,z)$, where $x$ is \textsc{arg1}, $y$ is \textsc{arg2} and $z$ is \textsc{arg3}. 
%% For verbs, I use these features like proto-roles in the
%% sense of \citet{Dowty91a}. For example, the subject of \emph{sehen} `see' is usually called an
%% experiencer. This is problematic if one wants to define passive as the suppression of an agent since
%% \emph{sehen} `see' allows for passivization. If both the subject and the object are classified as
%% \argone, it is possible to generalize over these argument roles. There are also predicates that do
%% take an \argtwo but no \argone. 

There are alternative ways to label
the arguments. Authors like \citew{DK2000b-u} use the terms \textsc{actor} and \textsc{undergoer}. For worked out
linking theories in HPSG see \citew{Wechsler91a-u}, \citew{DK2000b-u} and \citew{HMW2016a-u}.
}
\textsc{arg0} is the event variable, also represented as the \textsc{index}
(\textsc{ind}) under \textsc{content} (\textsc{cont}). Due to space limitations it is impossible to
explain the complete semantic setup, but the interested reader is referred to
\citew*{CFPS2005a}.\footnote{%
  The representations below are simplified. I do not use handles and labels as is common in Minimal
  Recursion Semantics.
}
\eas 
Lexical entry for the stem \stem{geb} `give':\\
\ms{
arg-st & \sliste{ NP[\str]\ind{1}, NP[\ldat]\ind{2}, NP[\str]\ind{3} }\\[2mm]
cont   & \ms{ ind & \ibox{4} event
            }\\
rels   & \relliste{ \ms[geben]{
                    arg0 & \ibox{4}\\
                    arg1 & \ibox{1}\\
                    arg2 & \ibox{2}\\
                    arg3 & \ibox{3}
                    } }
} 
\zs
The linking pattern for the English lexical item is completely parallel: the first argument is
linked to \textsc{arg1}, the second to \textsc{arg2}, and the third to \textsc{arg3}.

This seems to be similar to what \citet{AGT2014a} do with their
\templaten{Agent} and \templaten{Patient} templates. The \templaten{Agent} template introduces an
\argone and the \templaten{Patient} template introduces an \argtwo. See page~\pageref{ex:agent-temp} for the
definition of their templates. For the predicate \emph{draw} they assume that it has an \argone and
an \argtwo on semantic structure if it is used without the benefactive argument. The benefactive
template adds an \argthree and remaps \argtwo to \argthree. So rather than \emph{godzilla}, abbreviated as
\textit{g} in (\mex{1}a), \emph{Sandy} is the \argtwo in (\mex{1}b). \emph{Godzilla} is \argthree in
(\mex{1}b).
\eal
\ex 
\begin{tabular}[t]{@{}l@{}}
s-structure for \emph{Kim drew godzilla}:\\ 
\ms{
rel & draw\\
event & ev\\
arg1 & k\\
arg2 & g\\
}
\end{tabular}
\ex
\begin{tabular}[t]{@{}l@{}}
s-structure for \emph{Kim drew Sandy godzilla}:\\
\ms{
rel & draw\\
event & ev\\
arg1 & k\\
arg2 & s\\
arg3 & g\\
}
\end{tabular}
\zl
This is different in the HPSG proposal. The type definition states that \type{draw} has an event
variable (\textsc{arg0}) and two arguments (\textsc{arg1}, \textsc{arg2}). The benefactive cannot be added to the
structure of \type{drew} since the definition of the transitive verb \type{draw} does not contain a slot for a benefactive argument. As will be shown below
the information about the benefactive relation is introduced in a Neo-Davidsonian way instead.   

\subsection{Lexical rules}

This section discusses the lexical rules for benefactives and for resultative constructions and the
interaction of these lexical rules with nominalization.

\subsubsection{Benefactives}

\largerpage
I assume the lexical rule in (\mex{1}) for adding an additional benefactive argument:
\eas
Lexical rule for benefactives (German and English):\\
\ms[stem]{
arg-st & \sliste{ \ibox{1} NP[\str] } $\oplus$ \ibox{2} \sliste{ NP[\str] | \ldots }  \\[1mm]
cont   & \ms{ ind & \ibox{3} }\\
rels   & \ibox{4}
}~$\mapsto$\\
%
\ms{
arg-st & \sliste{ \ibox{1} NP[\str], NP\ind{5} } $\oplus$ \ibox{2}\\[2mm]
cont   & \ms{ ind & \ibox{3} }\\
rels   & \ibox{4} $\oplus$ \liste{ \ms[benefactive]{
                                   arg0 & \ibox{3}\\
                                   arg1 & \ibox{5}\\
                                   } }
} 
\zs
The \argstl of the input has to include two NPs with structural case (a nominative and an accusative
argument in the active). The \argstl of the input is split into two lists: one that contains a
single NP[\str] and another one that contains an NP[\str] and some possibly non-empty rest. The
\argst in the output of the lexical rule contains the initial NP of the input \iboxb{1}, an
additional NP and the list \ibox{2}, that is, at least the second NP with structural case.
The input description mentions the index of the input verb, which is the event variable
\iboxb{3}. The list of semantic relations that is contributed by the input sign is \ibox{4}. The
output specification of the lexical rule contains the list of relations of the input plus a
benefactive relation that states that the benefactive of the event \ibox{3} is \ibox{5}. \ibox{5} is
identified with the referential index of the added NP.

The output of the lexical rule is a verb stem with at least three arguments. Language-specific
constraints for verbs with three nominal arguments apply and ensure that the middle NP has
structural case in English and lexical dative in German. This is not shown in the general version of
the lexical rule above. The English version of the rule is more restrictive than the German one in
requiring that the input be strictly transitive. This excludes the application of the benefactive
lexical rule to the output of the resultative lexical rule for English but allows for this in German.

The result of applying the lexical rule to (\mex{1}a) is (\mex{1}b):

\eal
\ex 
\begin{tabular}[t]{@{}l@{}}
monotransitive version of \emph{backen}:\\
\ms{
    phon   & \phonliste{ back }\\[2mm]
    arg-st & \sliste{ NP[\str]\ind{1}, NP[\str]\ind{2} }\\
    ind    & \ibox{3} event\\
    rels   & \relliste{ \ms[backen]{
                        arg0 & \ibox{3}\\
                        arg1 & \ibox{1}\\
                        arg2 & \ibox{2}\\
                        } }\\
    }
\end{tabular}
\ex
\begin{tabular}[t]{@{}l@{}}
ditransitive version of \emph{backen}:\\
\ms{
    phon   & \phonliste{ back }\\[2mm]
    arg-st & \sliste{ NP[\str]\ind{1}, NP[\ldat]\ind{4}, NP[\str]\ind{2} }\\
    ind    & \ibox{3} event\\
    rels   & \relliste{ \ms[backen]{
                        arg0 & \ibox{3}\\
                        arg1 & \ibox{1}\\
                        arg2 & \ibox{2}\\
                        }, \ms[benefactive]{
                            arg0 & \ibox{3}\\
                            arg1 & \ibox{4}\\ } }\\
    }
\end{tabular}
\zl
\largerpage[2]
A dative argument is added between the two NPs that bear structural case and this dative argument is
linked to a role in the benefactive relation.

The lexical item for the three-place \emph{bake} in English would be parallel to (\mex{0}b). Since
this \emph{bake} has the same valency as the three-place verb \emph{give}, the syntactic structure
it can appear in is parallel. See Figure~\ref{fig-nobody-gives-him-the-book} for an example.

%% The lexical rule for English is very similar. Of course, English does not have a dative case. This
%% can be captured by assuming a type for ditransitive verbs that is language specific. In German,
%% ditransitives have a lexical dative as the second argument, while they have an argument with
%% structural case in second position in English. The lexical rule refers to this language specific
%% type and hence does not have to mention case at all.

Of course, the analysis presented here is incomplete in the sense that further constraints are
needed to prevent the application of the rule to semantically inappropriate verbs. Since
the focus of the book is to discuss phrasal and lexical approaches and since the incorporation of
the respective semantic constraints into a phrasal approach would be exactly parallel, I do not go
into semantic details here.

\subsubsection{Resultative constructions}

The lexical rule for resultative constructions with intransitive, mono-valent verbs or mono-valent variants of
transitive verbs is provided in (\mex{1}):\footnote{%
  This rule is not complete. Further constraints regarding the semantics of the input verb have to
  be stated.%
}
\ea
Lexical rule for resultatives:\\
\ms{
arg-st & \sliste{ \ibox{1} NP[\str] } \\[2mm]
cont   & \ms{ ind & \ibox{2} }\\
rels   & \ibox{3}\\
} 
$\mapsto$
\ms{
arg-st & \sliste{ \ibox{1}, \ibox{4} NP[\str], X(P)[\textsc{prd}+, \subj \sliste{ \ibox{4} }]:\ibox{5} }\\[2mm]
cont   & \ms{ ind & \ibox{6} event }\\
rels   & \ibox{3} $\oplus$ \liste{ \ms[cause]{
                                   arg0 & \ibox{6}\\
                                   arg1 & \ibox{2}\\
                                   arg2 & \ibox{7}\\
                                   }, \ms[become]{
                                       arg0 & \ibox{7}\\
                                       arg1 & \ibox{5}\\
                                      }}\\
} 
\z
The input is a verbal stem that selects for an NP with structural case and the output is a verbal
stem selecting for two NPs with structural case and a result predicate. The second NP is both an
argument of the verb and the subject of the result predicate. I assume that there is not a person that causes
the change of state but rather that the event of the input verb \iboxb{2} causes the change of
state. This makes it possible to capture cases in which there is no participant in the causing
event:\footnote{%
   Parallel structures in English are ungrammatical. They can be ruled out by requiring that the
   first NP argument is referential.
}
%%   linking 
% ähnlich:
% https://de.pinterest.com/pin/28499410113182985/
% Pride parade was rained on but it ended with a rainbow

\ea
\gll Es regnet die Stühle / Wäsche nass.\\
     it rains  the chairs {} clothes wet\\
\z
The highest event of the semantic representation in the output of the rule is the \type{cause} event
\iboxb{6}. Since \type{cause} is the highest event, \iboxt{6} is also the \textsc{index} value of
the output, which is represented under \textsc{cont|ind}. The \type{cause} event has as its first
argument the event expressed by the input verb \iboxb{2} and as its second argument the
\type{become} predicate \iboxb{7}. The \type{become} predicate takes the contribution of the
predicative phrase \iboxb{5} as its argument. 
\pagebreak

The result of the rule application to (\mex{1}a) is shown in (\mex{1}b):
\eal
\ex
\begin{tabular}[t]{@{}l@{}}
intransitive version of \emph{fischen}:\\
\ms{
    phon   & \phonliste{ fisch }\\[2mm]
    arg-st & \sliste{ NP[\str]\ind{1} }\\
    ind    & \ibox{6}\\
    rels   & \relliste{ \ms[fischen]{
                        arg0 & \ibox{6}\\
                        arg1 & \ibox{1}\\
                        } }\\
    }
\end{tabular}

\ex 
%\resizebox{.99\linewidth}{!}{%
\begin{tabular}[t]{@{}l@{}}
resultative version of \emph{fischen}:\\
\ms{
    phon   & \phonliste{ fisch }\\[2mm]
    arg-st & \sliste{ NP[\str]\ind{1}, \ibox{2} NP[\str], X(P)[\textsc{prd}+, \subj \sliste{ \ibox{2} }]:\ibox{4} }\\
    ind    & \ibox{5}\\
    rels   & \relliste{ \ms[fischen]{
                        arg0 & \ibox{6}\\
                        arg1 & \ibox{1}
                        },
                        \ms[cause]{
                        arg0 & \ibox{5}\\
                        arg1 & \ibox{6}\\
                        arg2 & \ibox{7}
                        },
                        \ms[become]{
                          arg0 & \ibox{7}\\
                          arg1 & \ibox{4}
                        } }\\
    }
\end{tabular}
%}
\zl
The event variable of the \type{cause} relation is \iboxt{5}. It is also the \indv of the lexical item.
\iboxt{6} is the event variable of \emph{fischen}. It is the \textsc{arg1} of the \type{cause}
relation. The second argument of the \type{cause} relation is the \type{become} relation. The
\type{become} relation takes the semantic contribution of the result predicate \iboxb{4} as argument.

A more readable semantic representation corresponding to the one in (\mex{0}b) is given in (\mex{1}):
\ea
cause(e1, fischen(e2,x), become(e3,P))
\z

The lexical item for the resultative construction may be input to the benefactive lexical rule. The
output is shown in (\mex{1}):
\ea 
ditransitive version of resultative \emph{fischen}:
%\oneline{%
\ms{
    phon   & \phonliste{ fisch }\\[2mm]
    arg-st & \sliste{ NP[\str]\ind{1}, NP[\ldat]\ind{3}, \ibox{2} NP[\str], X(P)[\textsc{prd}+, \subj \sliste{ \ibox{2} }]:\ibox{4} }\\
    ind    & \ibox{5}\\
    rels   & \relliste{ \ms[fischen]{
                        arg0 & \ibox{6}\\
                        arg1 & \ibox{1}
                        },
                        \ms[cause]{
                        arg0 & \ibox{5}\\
                        arg1 & \ibox{6}\\
                        arg2 & \ibox{7}
                        },
                        \ms[become]{
                          arg0 & \ibox{7}\\
                          arg1 & \ibox{4}
                        },
                        \ms[benefactive]{
                            arg0 & \ibox{5}\\
                            arg1 & \ibox{3}
                        } }\\
    }
%}
\z
In (\mex{0}) we have NP[\str], NP[\str] and the predicative X(P) as arguments and in addition we
also have the benefactive NP[\ldat]. The benefactive NP is linked to the \type{benefactive} relation
\iboxb{3}.

A more readable semantic representation corresponding to (\mex{0}) is the formula in (\mex{1}):
\ea
cause(e1, fischen(e2,x), become(e3,P)) $\wedge$ benefactive(e1,y)
\z


Figure~\vref{fig-analyis-benefactive+resultative} shows the analysis of (\ref{ex-dass-er-ihm-den-Teich-leer-fischt}).
\begin{figure}[b]
\centerfit{%
\begin{forest}
sm edges
[{VP[\eliste]}
   [{NP[\type{nom}]} [er;he] ]
   [\vbar\feattab{
      \sliste{ NP[\type{nom}] } }
      [{NP[\type{dat}]} [ihm;him] ] 
      [\vbar\feattab{
         \sliste{ NP[\type{nom}], NP[\type{dat}]}} 
         [{NP[\type{acc}]} [den Teich;the pond,roof ] ]
         [V$^0$\feattab{
           \sliste{ NP[\type{nom}], NP[\type{dat}], NP[\type{acc}]}} 
           [A [leer;empty]]
           [V$^0$\feattab{
             \sliste{ NP[\type{nom}], NP[\type{dat}], NP[\type{acc}], A}},l sep+=2ex
             [V$^{-1}$\feattab{
               \sliste{ NP[\type{nom}], NP[\type{dat}], NP[\type{acc}], A}}, edge label={node[midway,right]{Inflection-LR}},l sep+=2ex
               [V$^{-1}$\feattab{
               \sliste{ NP[\type{nom}], NP[\type{acc}], A}}, edge label={node[midway,right]{Benefactive-LR}},l sep+=2ex 
                 [V$^{-1}$\feattab{
                   \sliste{ NP[\type{nom}]}}, edge label={node[midway,right]{Result-LR}} 
                   [fisch-;fish] ] ] ] ] ] ] ] ]
\end{forest}
}
\caption{Analysis of [\emph{dass}] \emph{er ihm den Teich leer fischt} `that he fishes the pond empty for him', an example in which the benefactive
  and the resultative construction interact}\label{fig-analyis-benefactive+resultative}
\end{figure}
The resultative lexical rule applies to the mono-valent version of the lexical item for \stem{fish}
`to fish'. The lexical rule licenses another stem that selects for two NPs with structural case, which are
resolved to nominative and accusative in the example at hand. The benefactive lexical rule applies
to this lexical item and licenses another lexical item that selects for nominative, dative,
accusative and a result predicate. An inflectional lexical rule licenses the \vnull. The \vnull is
combined with the adjective to form a verbal complex, indicated by the label \vnull at the mother
node. \emph{leer fischt} is combined with its arguments by the German version of the Head-Complement
Schema in Figure~\ref{fig-head-comp-free}, and hence it is explained why six orders of the nominative, dative and accusative argument
are possible. In the analysis suggested here, the fact that scrambling is possible is a fact of German syntax that is independent of
how the arguments are licensed.

Note that all the stems in Figure~\ref{fig-analyis-benefactive+resultative} could be input to
derivational lexical rules that derive prenominal participles:
%%  (\mex{1}a--e) or \suffix{bar}
%% adjectives (\mex{1}f--:
\eal
\ex 
\gll der fischende Mann\\
     the fishing man\\
\ex 
\gll der den Teich leer fischende Mann\\
     the the pond empty fishing man\\
\glt `the man who fishes the pond empty'
%\todostefan{who fishes out the pond} Mary, Ash, ... read this
\ex 
\gll der den Teich seinem Freund leer fischende Mann\\
     the the pond  his    friend empty fishing man\\
\glt `the man who fishes the pond empty for his friend'
\ex 
\gll der leer gefischte Teich\\
     the empty fished pond\\
\ex 
\gll der dem Besitzer leer gefischte Teich\\
     the the owner    empty fished pond\\
\glt `the pond that was fished empty for the owner'
\zl
The derivational rules are independent of the benefactive and the resultative construction and apply
to verbs that have a subject in the case of the first participle formed with \suffix{end} and to
verbs that have an underlying object (transitive verbs and unaccusative ones) in the case of the
second participle formed with \prefix{ge} \suffix{t}. See \citet[\page 160]{Mueller2002b} for a formulation of
the latter rule. 

\subsubsection{Nominalizations}

There are several variants of nominalizations. The noun can be used with an agent as specifier as in
(\mex{1}a) or with a normal determiner as in (\mex{1}b,c):
\eal
\ex 
\gll Peters Leerfischung des Teiches\\
     Peter's empty.fishing of.the pond\\
\ex 
\gll die Leerfischung des Teiches\\
     the empty.fishing of.the pond\\
\ex 
\gll die Leerfischung des Teiches durch Peter\\
     the empty.fishing of.the pond by Peter\\
\zl
The important point is that this is independent of the resultative construction. (\mex{1}) shows an
example with a transitive verb:
\eal
\ex 
\gll Peters Zerstörung des Buches\\
     Peter's destruction of.the book\\
\ex 
\gll die Zerstörung des Buches\\
     the destruction of.the book\\
\ex 
\gll die Zerstörung des Buches durch Peter\\
     the destruction of.the book by Peter\\
\zl
I assume that the nominalization attaches to the verb stem. In the case of the resultative
construction in (\mex{-1}) the result predicate is then combined with the derived nominal stem. As the result of the combination we
get a word that has one or more NPs with structural case on its \argstl. The case principle assigns genitive
to these NPs since it is realized in a nominal environment.


%% \chapter{Lexical rules/lexical constructions/lexical templates}

%% The original constructional idea about benefactive constructions and about resultative constructions with
%% transitive verbs is that a verb that is usually used monotransitively enters a phrasal construction
%% which licenses an additional argument. In the case of the benefactive construction it is the
%% benefactive argument and in the case of resultative constructions it is the result predicate. The
%% same conditions can be formulated as lexical rules. The input for the respective lexical rules is a
%% monotransitive verb and the output is a three-place verb with a benefactive argument or a
%% three-place verb that selects for a result predicate. All regularities that can be captured with
%% phrasal constructions can be capture with lexical constructions as well. 

%\section{Rule ordering}

%% \section{Capturing the insights of Construction Grammar}

%% Construction Grammar in its most influential form argues that phrasal patterns are the most
%% important building blocks of natural languages. For instance, Tomasello talks about
%% patterns like the one in (\mex{1}). 
%% \ea
%% {}[~Subj~V~Obj~Obj~] 
%% \z
%% Lexical approaches do not talk about the patterns directly but about
%% lexical representations that license various patterns. Ordered arguments (\textsc{arg}$_1$,
%% \textsc{arg}$_2$, \textsc{arg}$_3$) play a crucial role in the template-based approach and also in
%% other approaches as for instance in HPSG


\section{Constraints on extraction and passivization}
\label{sec-restrictions-lexical}

\citet[\page 516]{Toivonen2013a} argues that the benefactive construction is best seen as an
instantiation of the phrasal configuration in Figure~\vref{fig-benefactive-toivonen}.
She noticed that question formation involving the extraction of the benefactive NP is
excluded. The respective examples in (\ref{ex-question-formation}) are repeated here as (\mex{1})
and (\mex{2}) for convenience: 
\exewidth{(235)}
\eal
\ex[]{
I baked Linda cookies
}
\ex[*]{
Who did I bake cookies?
}
\zl
\eal
\ex[]{
The kids drew their teacher a picture.
}
\ex[*]{
Which teacher did the kids draw a picture?
}
\zl
She also discusses the example in (\ref{ex-my-sister-was-carved}) -- repeated as (\mex{1}) -- , which is judged ungrammatical by speakers of certain
dialects of English:
\ea[*]{
My sister was carved a soap statue of Bugs Bunny (by a famous sculptor).
}
\z
She observed that all these ungrammatical examples are accounted for by assuming that benefactives
are licensed in structures like the one given in Figure~\ref{fig-benefactive-toivonen}.

\largerpage
If one wanted to assume that a certain configuration is stored as chunk, one could do so without
problems in the model suggested here. Figure~\ref{fig-phrasal-configuration-English} shows what would be stored for English.
\begin{figure}
\centerline{%
\begin{forest}
[VP
  [V$'$
    [V
      [V]]
    [NP]]
  [NP]]
\end{forest}}
\caption{Stored phrasal configuration for English}\label{fig-phrasal-configuration-English}
\end{figure}
The figure shows a chunk and of course certain slots in this structure can be filled. The important
point about the figure is that the lexical rule application that maps a two-place verb to a
three-place one would be part of the stored configuration. This ensures that language-internal and crosslinguistic
generalizations are captured. However, as I have shown, storing a phrasal configuration is not what
is required here since extraction of the secondary object is possible (\ref{ex-extraction-secondary-object}) and the construction
may be realized discontinuously in coordination structures (\ref{ex-coordination-benefactive}). However, there are other ways of
blocking extraction and passivization. The passive is treated as the suppression of the subject. If
there is an object with structural case, it is the least oblique element in the passive and therefore
it gets nominative by the Case Principle, which is usually assumed. Now, if the case of the object is
lexically constrained to be accusative, the verbal lexeme can only be used in the active since
otherwise the case specification of the benefactive argument as accusative would be in conflict with
the assignment of nominative by the Case Principle. So, for speakers that allow for passivization,
the case of the subject and the two objects is just specified as structural with the actual value
being underspecified, and for speakers who do not allow for a passive, the case of the benefactive
argument is specified to be accusative.

The extraction of primary objects is marked for all verbs that take two objects irrespective of the
semantic role. For some speakers the extraction of benefactives is worse than the extraction of
other primary objects.
% For instance Gerlad Penn
If one wanted to block extraction via a hard constraint rather than assuming that performance
factors play a role here \citep{LKD73a}, one could state that the
\slashv of the primary object is the empty list \citep[\page 98]{Mueller99a} or -- if extraction out
of the primary object is to be permitted -- different from the \localv of the primary
object. Because of this specification a trace would be incompatible with this object. The same
applies to an appropriately specified lexical rule for argument extraction \citep[\page 226]{Mueller96b} or a process like \slasch
amalgamation as suggested by \citet*{BMS2001a}.

Note that this approach also predicts that constraints on extraction and passivization in
coordinated structures affect the result of coordination. The reason is that the constraints on the
selected arguments are identified in symmetric coordinations \citep[\page 202]{ps2}. Hence, the \slasch constraints and the
case constraints on the benefactive argument are effective on the mother node of verb coordinations
as well. So, the analyses that introduce constraints for extraction and passivization lexically
correctly predict that the coordination of two items is at least as restrictive as the individual
conjuncts, while in approaches that introduce the constraints on the phrasal level, coordinating
items may result in an object that can enter less restrictive phrasal rules.


For completeness it should be noted that the German benefactive construction is much less
restricted. The benefactive arguments can be extracted in German and can be used in questions:
\eal
\ex 
\gll Wem habe ich Kekse gebacken?\\
     who.\dat{} have I.\nom{} cookies.\acc{} baked\\
\ex 
\gll Welchem Lehrer haben die Kinder ein Bild gemalt?\\
     which.\dat{} teacher have the.\nom{} children a.\acc{} picture drawn\\
\zl
As (\ref{ex-benefactive-in-differentcases}) shows, the construction also interacts with the dative
passive. Hence, Toivonen's original motivation for a phrasal approach would not apply to
German. 

Since I have shown how the respective constraints can be formulated in a lexical approach,
there is now a proposal that captures both German and English and the commonalities between the two languages.


\chapter{Conclusions}
\label{sec-conclusions}

I have shown that both the benefactive and the resultative construction are more
flexible than originally suggested by the authors who proposed phrasal configurations. All
non-verbal parts of the resultative construction may be extracted or promoted by passivization. The
secondary object of benefactive constructions may be extracted and some speakers allow for
passivization.
 
I have also shown that morphology needs access to valence (adjective formation and \suffix{bar}
`able' derivation). \citet{Alsina96a} showed that a lexical analysis of the passive is possible even
for analyses that introduce the accusative object syntactically. But the examples that were
discussed in the present book involved the selection of lexemes governing an accusative in the
morphology component. If this valence information is not added to lexical items but dependents are introduced by phrasal
constructions instead, there is no way to account for the insights regarding morphological rules.

Furthermore, I have argued that either the c"=structure does not add any constraints in a
template-based phrasal approach or the relation between active and passive variants of a construction
is not covered. I also showed that the phrasal analysis of English benefactive and resultative
constructions does not carry over to languages that are assumed to have different c-structures. As
was the case for the phrasal GPSG approach to valence, partial phrases that play a role in
coordination, partial fronting, and also certain accounts of fronting are problematic for
pattern-based approaches to argument structure. 

I have shown that all these problems disappear and crosslinguistic generalizations regarding the
benefactive, resultative and many other constructions can be captured if one returns
to the traditional lexical analysis, which assumed that all arguments are introduced lexically. A
version of the lexical analysis was presented in Chapter~\ref{sec-lexical-approach-hpsg}. This analysis is the basis of
implemented fragments of German and English that have been developed in the CoreGram project
\citep{MuellerGrammix,MuellerCoreGram}. As was demonstrated, the lexical rule for the benefactive in
German and English is the same. The languages differ in how the second argument of ditransitive
verbs is realized since German has a morphologically marked dative case, which is absent from
English. But this is a general property of ditransitive verbs that is independent of the benefactive
rule. Lexical rules for resultative predicates are parallel for English and German. The differences
are due to the differences in the syntactic systems of the languages but this is independent of the
resultative construction.

With the system of lexical rules in place, the phrasal schemata for specifier-head structures
and head-complement structures in German and English are identical (or rather the schema for English
is a specialization of the one for German). No special construction-specific
stipulations are needed.

By having shown that approaches assuming the resultative construction and/or the benefactive
construction to be phrasal constructions run into problems, I have also shown that approaches
considering all argument structure constructions phrasal are problematic. Hence, this book is a
contribution to the general debate about argument structure constructions. It shows that phrasal
constructions (in LFG) are not suited to deal with argument structure. Instead, lexical constructions
(lexical rules) are needed. The syntactic combinations are licensed by rather abstract syntactic
rules. Nevertheless, phrasal constructions are useful and necessary in those parts of grammars that
do not interact with argument structure and valence alternations. An example of such a construction
is the N-P-N construction (\emph{student after student}), in which no head can be identified
\citep{Jackendoff2008a}. So, this book provides support for the position that a mix of the proposals
from the two major linguistic schools is needed: we need a rich lexicon and abstract schemata for
combining linguistic objects and we need specific phrasal constructions that contribute their own semantics.




%      <!-- Local IspellDict: en_US-w_accents -->




%% die Frau -> SUBJ / OBJ

%% die Frau kennt den Mann.

%% Die Frau wird geschlagen.  SUBJ / OBJ


%% ----------------------------------------------------

%% On Fri, Nov 11, 2016 at 4:25 AM, Stefan Müller <St.Mueller@hu-berlin.de> wrote:

%%     Dear Steve,

%%     I have some more questions on benefactives. I now read Ida's paper from
%%     2013. She looked at questions and "do so" and so on and claimed that
%%     question formation is impossible, which is of course explained by the
%%     assumption that the benefactive has to be verb-adjacent.

%%     Now my question:

%%     Do you know of any attested examples of the sort:

%%     Who(m) did she bake a cake?

%%     Is it possible to have frontings like:

%%     This man I would never bake a cake!

%%     He never drew his mother a picture, but his teacher, he did draw a picture.


%% These all sound bad to me, and I don't know of attested examples.  
 

%%     And if Ida is right, how would one do this in a lexical world. The
%%     easiest way is to say SLASH of the benefactive is just the empty
%%     set/list. This does it, but it feels like a hack. On the other hand the
%%     phrasal analysis feels like a hack as well.

%%     The extraction of the dative is pretty normal in German, so I would
%%     guess that extraction is blocked in English since things get intransparent.


%% Yeah, English lacks dative case.
 

%%     Is the following possible:

%%     This man, I would never give a cake!


%% Rather bad.
 


%%     Is it marked? Would be the extraction from dative shift be better?

%%     This man, I would never give a cake to!


%% This one is fine.  I think Ida is right about the facts.  
 


%%     Anything else apart from passive where we see that the dative is not
%%     positionally fixed?


%% Not that I know of.   Suppose there is a phrasal construction.  That wouldn't be a problem for lexicalist theory, would it?  

%% Steve
 


%% ---------------------------------------------------------







%% On Sat, Nov 12, 2016 at 5:06 AM, Stefan Müller <St.Mueller@hu-berlin.de> wrote:


%%     >     Now my question:
%%     >
%%     >     Do you know of any attested examples of the sort:
%%     >
%%     >     Who(m) did she bake a cake?
%%     >
%%     >     Is it possible to have frontings like:
%%     >
%%     >     This man I would never bake a cake!
%%     >
%%     >     He never drew his mother a picture, but his teacher, he did draw a
%%     >     picture.
%%     >
%%     >
%%     > These all sound bad to me, and I don't know of attested examples.

%%     OK. Learned something. These are possible in German.

%%     >
%%     >     And if Ida is right, how would one do this in a lexical world. The
%%     >     easiest way is to say SLASH of the benefactive is just the empty
%%     >     set/list. This does it, but it feels like a hack. On the other hand the
%%     >     phrasal analysis feels like a hack as well.
%%     >
%%     >     The extraction of the dative is pretty normal in German, so I would
%%     >     guess that extraction is blocked in English since things get
%%     >     intransparent.
%%     >
%%     >
%%     > Yeah, English lacks dative case.
%%     >
%%     >
%%     >     Is the following possible:
%%     >
%%     >     This man, I would never give a cake!
%%     >
%%     >
%%     > Rather bad.

%%     But then this could be a general constraint on extraction of the first
%%     object. "give" is a normal ditransitive verb and benefactive cook or
%%     other verbs could be just similar to "give" in not allowing extraction.

%%     Is the following good?

%%     Such great cakes, I would never give this strange teacher.


%% A little awkward, but grammatical.
 

%%     What did you give the teacher?


%% Perfect.
 



%%     Such great cakes, I would never bake this strange teacher.

%%     What did you bake the teacher?


%% These are more awkward than the ones above.  They probably do occur.  
%% These are better:

%%      What did you buy me?
%%      What did you buy your teacher?

%% Steve
 



%%     Thanks!

%%     >
%%     >
%%     >     Is it marked? Would be the extraction from dative shift be better?
%%     >
%%     >     This man, I would never give a cake to!
%%     >
%%     >
%%     > This one is fine.  I think Ida is right about the facts.

%%     So maybe it is the explicit marking by "to" that helps processing the
%%     stuff. Since extraction destroys the configuration, hearers do not have
%%     any chance to recover the grammatical relations.

%%     Would that be a plausible explanation? Maybe one can construct plausible
%%     contexts that make the interpretation easier and result in higher
%%     acceptability.


%%     >
%%     >
%%     >     Anything else apart from passive where we see that the dative is not
%%     >     positionally fixed?
%%     >
%%     >
%%     > Not that I know of.   Suppose there is a phrasal construction.  That
%%     > wouldn't be a problem for lexicalist theory, would it?

%%     Not at all. I am looking for things to break the phrasal analysis.

%%     Best wishes

%%             Stefan

%%     > Steve
%%     >
%%     >
%%     >
%%     >
%%     >     Best wishes
%%     >
%%     >             Stefan
%%     >
%%     >     --
%%     >     PGP welcome
%%     >
%%     >     Stefan Müller       Tel: (+49) (+30) 2093 9631
%%     >
%%     >     Institut für deutsche Sprache und Linguistik
%%     >     Dorotheenstraße 24
%%     >     10117 Berlin
%%     >
%%     >     http://hpsg.fu-berlin.de/~stefan/ <http://hpsg.fu-berlin.de/~stefan/>
%%     >
%%     >     http://langsci-press.org/
%%     >
%%     >     http://hpsg.fu-berlin.de/Projects/CoreGram.html
%%     >     <http://hpsg.fu-berlin.de/Projects/CoreGram.html>
%%     >
%%     >




%% Philipp Miller 29.11.2016

%% ?*Kim was baked a cake.
%% ?*Pat was written a recommendation letter. 

%% On 11/29/16 08:35, Stefan Müller wrote:
%% > Dear Philip,
%% >
%% > Thanks for your mail! I am doing fine and I hope you do too!
%% >
%% > It is this paper from language:
%% >
%% > @article{Hudson92a,
%% >     Author = {Richard Hudson},
%% >     Journal = {Language},
%% >     Number = {2},
%% >     Pages = {251--276},
%% >     Title = {So-Called 'Double Objects' and Grammatical Relations},
%% >     Volume = {68},
%% >     Year = {1992}}
%% >
%% >> My corpus work shows that it isn't, and one of the island contexts I
%% >> found quite a few examples of is the one you mention below. But
%% >> though I felt that the results of extraction out of these positions
%% >> were quite unacceptable,
%% >
%% > So you say there are attested examples but you find them unacceptable?

%% Sorry, I may have been unclear: I find the cases of relative and wh question extraction out of these contexts quite unacceptable, but the pseudogappings are fine. So the argument is that the remnant isn't subject to the same kinds of constraints and shouldn't undergo a similar type of analysis (i.e. movement, or whatever analysis of long-distance dependencies your theory countenances).
%% >
%% > People say that the passive with benefactives is ungrammatical for some
%% > speakers. So maybe there is a correlation ...

%% Yes, I agree that the following are really very unacceptable:

%% ?*Kim was baked a cake.
%% ?*Pat was written a recommendation letter.

%% >
%% > I try to finish a HeadLex proceedings paper today. If you are interested
%% > I will send it to you. Would be great to have some discussion about this
%% > stuff.

%% I'll try to read it quickly if you send it.


%% >
%% > Best wishes
%% >
%% >         Stefan
%% >

%% PS trying to produce a new benefactive+passive example, I came up with the recommendation letter case above (corresponding to X wrote a recommendation letter for Pat). But rereading, I realized that the sentence should be ambiguous and acceptable on the less plausible reading that Pat is the intended recipient of the recommendation letter rather than the person the letter is about and who benefits from having it to add to their file. But for some reason, it really sounds bad, and a brief check on the COCA using the string

%% was|were written a|an

%% returns no examples of a recipient subject passive.
%% I have no idea why this is (and maybe a broader corpus would provide examples), but it might be worth checking to what extent passive recipients are grammatical in general (beyond the "give" and "send" examples that typically given in papers and textbooks). 
